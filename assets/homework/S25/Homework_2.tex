\documentclass[11pt, oneside]{article}   	% use "amsart" instead of "article" for AMSLaTeX format
\usepackage{geometry}                		% See geometry.pdf to learn the layout options. There are lots.
\geometry{letterpaper}                   		% ... or a4paper or a5paper or ... 
%\geometry{landscape}                		% Activate for rotated page geometry
\usepackage[parfill]{parskip}    		% Activate to begin paragraphs with an empty line rather than an indent
\usepackage{graphicx}				% Use pdf, png, jpg, or eps§ with pdflatex; use eps in DVI mode
								% TeX will automatically convert eps --> pdf in pdflatex		
\usepackage{amssymb, enumerate}

%SetFonts

%SetFonts


\title{Math F113X: Homework Set 2}
%\author{The Author}
\date{}							% Activate to display a given date or no date

\begin{document}
\maketitle
%\section{}
%\subsection{}

%Homework assignment 1 is:

\fbox{\parbox{\textwidth}{

\begin{itemize}
\item Answer the following problems from the Voting Theory section (p.53):
\begin{quote}\# 3def, 6def, 12, 13-16*, 18, 24ab, Problem A\end{quote}
* These are really one problem. 

{\bf Problem A:}
\begin{enumerate}[(1)]
\item Explain Arrow's Impossibility Theorem in your own words in such a way that a typical adult could understand it. 

\item Now that you have learned a lot about voting schemes, what scheme would \textbf{you} choose to elect the President of the United States and why? (A short paragraph is sufficient even though we all know entire books can be written on this topic!)

\item How has learning about voting schemes changed your view of voting and elections? (A sentence or two is sufficient.)
\end{enumerate}



\item Answer the following {\bf reflection question}: What did you learn from checking your homework answers against the provided solutions?
\end{itemize}
}}

Remember to write up your homework solutions according to the homework writeup guidelines. 

Homework is graded using the following rubric for each problem (or problem part):

\begin{description}
\item[2 points:] You provided a complete answer, with supporting work, written up clearly
\item[1 point:] Some attempt at a solution, but incomplete writeup / unclear / illegible / no answer
\item[1 point:] Only an answer, with no supporting work 
\item[0 points:] Missing.
\end{description}

After you do the homework, you need to check your answers against the solutions! Then figure out your errors (if any) and revise your homework before you submit it. Finally, answer the reflection question.

Homework must be submitted on Gradescope.

\end{document}  