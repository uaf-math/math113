\documentclass[11pt, oneside]{article}   	% use "amsart" instead of "article" for AMSLaTeX format
\usepackage[margin = 1in]{geometry}                		% See geometry.pdf to learn the layout options. There are lots.
\geometry{letterpaper}                   		% ... or a4paper or a5paper or ... 
%\geometry{landscape}                		% Activate for rotated page geometry
\usepackage[parfill]{parskip}    		% Activate to begin paragraphs with an empty line rather than an indent
\usepackage{graphicx, array,hyperref}				% Use pdf, png, jpg, or eps§ with pdflatex; use eps in DVI mode
								% TeX will automatically convert eps --> pdf in pdflatex		
\usepackage{amssymb, enumerate, tikz, multicol}

%SetFonts

%SetFonts


\title{Math F113X: Homework Set 10}
%\author{The Author}
\date{}							% Activate to display a given date or no date

\begin{document}

\maketitle

\vspace{-1.5cm}


\fbox{\parbox{\textwidth}{

\begin{itemize}
\item Complete problems \#13-18 from the Encryption section.
\item Complete problems A  and B on Encryption.
\item Complete problems \#1-10 from Section 2.1.6 from the version of the text linked here:\\
\url{https://spot.pcc.edu/math/mathinsociety/chap_two_intro_spreadsheet.html}
\item Next, answer the {\bf reflection question}: What did you learn from checking your homework answers against the provided solutions?
\end{itemize}
}}

\textbf{Problem A} Vegen\`{e}re Cipher
	\begin{enumerate}
	\item[(a)] Encrypt the plaintext below using a Vegen\`{e}re cipher with keyword NUMBER.\\
	plaintext: WHY WAS MATH PROF LATE\\
	%JBK XEJ ZUFI TIBZ XBXV
	\item[(b)] Decrypt the plaintext below using a Vegen\`{e}re cipher with keyword NUMBER.\\\\
	ciphertext: GBQZ XFBE FII IUIYCYJ \\
	%They took the rhombus.
	\end{enumerate}

\textbf{Problem B} Double Transposition
	\begin{enumerate}
	\item[(a)] Encrypt the plaintext below using double transposition and first keyword WARM and second keyword MUSHY.\\
	plaintext: WE PREDICT THE TRIPOD WILL FALL\\
	%RCPFO ELPHE LTRAW TDLII TDEIW L
	\item[(b)] Decrypt the ciphertext below assuming double transposition was used where the first keyword was WARM and second keyword was MUSHY.\\
	ciphertext:  TAADP ENEBM TYNNR HEFES DEMEO UNCOW OR\\
	%ONMAYSECONDBETWEENTHREEANDFOURPM
	\end{enumerate}
\vfill
\hrulefill

Remember to write up your homework solutions according to the homework writeup guidelines. 

Homework is graded using the following rubric for each problem (or problem part):

\begin{description}
\item[2 points:] You provided a complete answer, with supporting work, written up clearly
\item[1 point:] Some attempt at a solution, but incomplete writeup / unclear / illegible / no answer
\item[1 point:] Only an answer, with no supporting work 
\item[0 points:] Missing.
\end{description}

After you do the homework, you need to check your answers against the solutions! Then figure out your errors (if any) and revise your homework before you submit it. Finally, answer the reflection question.

Homework must be submitted on Gradescope.

\end{document}  