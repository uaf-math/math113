\documentclass[11pt, oneside]{article}   	% use "amsart" instead of "article" for AMSLaTeX format
\usepackage[margin = 1in]{geometry}                		% See geometry.pdf to learn the layout options. There are lots.
\geometry{letterpaper}                   		% ... or a4paper or a5paper or ... 
%\geometry{landscape}                		% Activate for rotated page geometry
\usepackage[parfill]{parskip}    		% Activate to begin paragraphs with an empty line rather than an indent
\usepackage{graphicx, array}				% Use pdf, png, jpg, or eps§ with pdflatex; use eps in DVI mode
								% TeX will automatically convert eps --> pdf in pdflatex		
\usepackage{amssymb, enumerate, tikz, multicol}

%SetFonts

%SetFonts


\title{Math F113X: Homework Set 9}
%\author{The Author}
\date{}							% Activate to display a given date or no date

\begin{document}

\maketitle

\vspace{-1.5cm}


\fbox{\parbox{\textwidth}{

\begin{itemize}
\item Complete problems \#1-12 from the Encryption section.
\item Complete problems A and B.
\item Next, answer the {\bf reflection question}: What did you learn from checking your homework answers against the provided solutions?
\end{itemize}
}}

\vfill
\fbox{\textbf{Problem A:}} Suppose you find the encrypted message below. 
\begin{quote} \textbf{Znk qke oy zu ynolz he ykbkt. Yu znk rkzzkx g muky zu znk rkzzkx m.} \end{quote}
	\begin{enumerate}
	\item[a.] Fill out the frequency table below for the symbols in the encrypted text.\\
	\begin{tabular}{c|c||c|c||c|c}
	letter&frequency&letter&frequency&letter&frequency\\
	\hline \hline
	A or a&&J or j&&S or s&\\
	\hline
	B or b&&K or k&&T or t&\\
	\hline
	C or c&&L or l&&U or u&\\
	\hline
	D or d&&M or m&&V or v&\\
	\hline
	E or e&&N or n&&W or w&\\
	\hline
	F or f&&O or o&&X or x&\\
	\hline
	G or g&&P or p&&Y or y&\\
	\hline
	H or h&&Q or q&&Z or z&\\
	\hline
	I or i&&R or r&&&\\
	\hline
	\end{tabular}
	\item[b.] Identify the most frequent symbols and guess which letters they represent based on typical frequency tables of English text and supposing the encryption method is a substitution cipher. (See Example 5 from the text.) \\
	\begin{tabular}{c||c|c|c|c|c}
	frequency&1st&2nd&3rd&4th&5th\\
	\hline
	encrypted symbol&&&&&\\
	\hline
	likely/possible&&&&&\\
	decrypted letter&&&&&\\
	\end{tabular}
	\item[c.] Suppose the encryption method is a \textbf{Caesar cipher}. Determine the shift and decrypt the message.
	
	\end{enumerate}
\fbox{\textbf{Problem B:}} \textbf{This is a challenging problem. Be patient with yourself.} Suppose you find the encrypted message below. Its letter frequency is in the following table. 
\begin{quote} \textbf{RXVKWAMM METBB FTLA VX BTH WAMCARGYVK TV AMGTJBYMEFAVG XZ WABYKYXV, XW CWXEYJYGYVK GEA ZWAA AIAWRYMA GEAWAXZ; XW TJWYSKYVK GEA ZWAASXF XZ MCAARE, XW XZ GEA CWAMM; XW GEA WYKEG XZ GEA CAXCBA CATRATJBP GX TMMAFJBA, TVS GX CAGYGYXV GEA KXDAWVFAVG ZXW T WASWAMM XZ KWYADTVRAM.} \end{quote}
	 
	\begin{tabular}{c|c||c|c||c|c}
	letter&frequency&letter&frequency&letter&frequency\\
	\hline \hline
	A&38&J&5&S&4\\
	\hline
	B&8&K&9&T&12\\
	\hline
	C&8&L&1&U&0\\
	\hline
	D&2&M&15&V&13\\
	\hline
	E&12&N&0&W&20\\
	\hline
	F&5&O&0&X&21\\
	\hline
	G&17&P&1&Y&14\\
	\hline
	H&1&Q&0&Z&9\\
	\hline
	I&1&R&6&&\\
	\hline
	\end{tabular}
	\begin{enumerate}
	\item[a.] Identify the most frequent symbols and guess which letters they represent based on typical frequency tables of English text and supposing the encryption method is a substitution cipher. (See Example 5 from the text.) \\
	\begin{tabular}{c||c|c|c|c|c}
	frequency&1st&2nd&3rd&4th&5th\\
	\hline
	encrypted symbol&&&&&\\
	\hline
	likely/possible&&&&&\\
	decrypted letter&&&&&\\
	\end{tabular}
	\item[b.] Based on your work in parts a and b, explain why this is \textbf{unlikely} to be a \textbf{Caesar cipher}. 
	\item[c.] Fill in the decryption key below and decrypt  the message. Note that to do this will require ad hoc and guess-and-check strategies.\\
	
\begingroup
\setlength{\tabcolsep}{4pt}
\renewcommand{\arraystretch}{0.5}
	\hspace*{-0.2in}\begin{tabular}{c|c|c|c|c|c|c|c|c|c|c|c|c|c|c|c|c|c|c|c|c|c|c|c|c|c|c}
	encrypted &&&&&&&&&& &&&&&&&&&& &&&&&&\\
	symbol&A&B&C&D&E&F&G&H&I&J&K&L&M&N&O&P&Q&R&S&T&U&V&W&X&Y&Z\\ \hline
	decrypted &&&&&&&&&& &&&&&&&&&& &&&&&&\\
	symbol&&&&&&&&&& &&&&&&&&&& &&&&&&\\ 
	\end{tabular}
	
\endgroup

\item[d.] What aspects of the encryption method made it much easier to decrypt?
	\end{enumerate}

\hrulefill

Remember to write up your homework solutions according to the homework writeup guidelines. 

Homework is graded using the following rubric for each problem (or problem part):

\begin{description}
\item[2 points:] You provided a complete answer, with supporting work, written up clearly
\item[1 point:] Some attempt at a solution, but incomplete writeup / unclear / illegible / no answer
\item[1 point:] Only an answer, with no supporting work 
\item[0 points:] Missing.
\end{description}

After you do the homework, you need to check your answers against the solutions! Then figure out your errors (if any) and revise your homework before you submit it. Finally, answer the reflection question.

Homework must be submitted on Gradescope.

\end{document}  