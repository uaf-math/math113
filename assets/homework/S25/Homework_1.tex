\documentclass[11pt, oneside]{article}   	% use "amsart" instead of "article" for AMSLaTeX format
\usepackage{geometry}                		% See geometry.pdf to learn the layout options. There are lots.
\geometry{letterpaper}                   		% ... or a4paper or a5paper or ... 
%\geometry{landscape}                		% Activate for rotated page geometry
\usepackage[parfill]{parskip}    		% Activate to begin paragraphs with an empty line rather than an indent
\usepackage{graphicx}				% Use pdf, png, jpg, or eps§ with pdflatex; use eps in DVI mode
								% TeX will automatically convert eps --> pdf in pdflatex		
\usepackage{amssymb}

%SetFonts

%SetFonts


\title{Math F113X: Homework Set 1}
%\author{The Author}
\date{}							% Activate to display a given date or no date

\begin{document}
\maketitle
%\section{}
%\subsection{}

%Homework assignment 1 is:

\fbox{\parbox{\textwidth}{
\begin{itemize}
\item Answer the following problems from the Voting Theory section (p.53):
\begin{quote}\# 1, 3abc, 6abc, 8, 10, 17, 20\end{quote}
\item Answer the following {\bf reflection question}: What did you learn from checking your homework answers against the provided solutions?
\end{itemize}
}}

Remember to write up your homework solutions according to the homework writeup guidelines. 

Homework is graded using the following rubric for each problem (or problem part):

\begin{description}
\item[2 points:] You provided a complete answer, with supporting work, written up clearly
\item[1 point:] Some attempt at a solution, but incomplete writeup / unclear / illegible / no answer
\item[1 point:] Only an answer, with no supporting work 
\item[0 points:] Missing.
\end{description}

After you do the homework, you need to check your answers against the solutions! Then figure out your errors (if any) and revise your homework before you submit it. Finally, answer the reflection question.

Homework must be submitted on Gradescope.

\end{document}  