\documentclass[11pt, oneside]{article}   	% use "amsart" instead of "article" for AMSLaTeX format
\usepackage[margin = 1in]{geometry}                		% See geometry.pdf to learn the layout options. There are lots.
\geometry{letterpaper}                   		% ... or a4paper or a5paper or ... 
%\geometry{landscape}                		% Activate for rotated page geometry
\usepackage[parfill]{parskip}    		% Activate to begin paragraphs with an empty line rather than an indent
\usepackage{graphicx}				% Use pdf, png, jpg, or eps§ with pdflatex; use eps in DVI mode
								% TeX will automatically convert eps --> pdf in pdflatex		
\usepackage{amssymb, enumerate}

%SetFonts

%SetFonts


\title{Math F113X: Homework Set 4}
%\author{The Author}
\date{}							% Activate to display a given date or no date

\begin{document}
\maketitle
%\section{}
%\subsection{}

%Homework assignment 1 is:
%\vspace{-1.5cm}


\fbox{\parbox{\textwidth}{

\begin{itemize}
\item Answer the following problems from the Fair Division  section:
\begin{quote}Problem A (below), \# 4, 6, 8, 10, 11, 14, 16 \end{quote}

\item Answer the following {\bf reflection question}: What did you learn from checking your homework answers against the provided solutions?
\end{itemize}
}}

Problem A: Ahmed, Emily, Tiana, and Levi buy a large pizza for \$24. It is half pepperoni and half veggie and cut into 8 slices.
\begin{enumerate}
	\item[a.] Suppose Ahmed doesn't care whether the slice is pepperoni or veggie. What is the value a slice of veggie for Ahmed?
	\item[b.] Suppose Emily doesn't eat meat of any kind. What is the value a slice of veggie for Emily? What is the value a slice of pepperoni for Emily?
	\item[c.] Suppose Tiana values a slice of pepperoni twice as much as a slice of veggie. What is the value a slice of veggie for Tiana? What is the value a slice of pepperoni for Tiana?
	\item[d.] Are two slices of veggie a fair share for Tiana? Justify your answer.
	\item[e.] Are two slices of veggie a fair share for Emily? Justify your answer.
\end{enumerate}

\hrulefill

Remember to write up your homework solutions according to the homework writeup guidelines. 

Homework is graded using the following rubric for each problem (or problem part):

\begin{description}
\item[2 points:] You provided a complete answer, with supporting work, written up clearly
\item[1 point:] Some attempt at a solution, but incomplete writeup / unclear / illegible / no answer
\item[1 point:] Only an answer, with no supporting work 
\item[0 points:] Missing.
\end{description}

After you do the homework, you need to check your answers against the solutions! Then figure out your errors (if any) and revise your homework before you submit it. Finally, answer the reflection question.

Homework must be submitted on Gradescope.

\end{document}  