\documentclass[11pt, oneside]{article}   	% use "amsart" instead of "article" for AMSLaTeX format
\usepackage[margin = 1in]{geometry}                		% See geometry.pdf to learn the layout options. There are lots.
\geometry{letterpaper}                   		% ... or a4paper or a5paper or ... 
%\geometry{landscape}                		% Activate for rotated page geometry
\usepackage[parfill]{parskip}    		% Activate to begin paragraphs with an empty line rather than an indent
\usepackage{graphicx, ulem}				% Use pdf, png, jpg, or eps§ with pdflatex; use eps in DVI mode
								% TeX will automatically convert eps --> pdf in pdflatex		
\usepackage{amssymb, enumerate}

%SetFonts

%SetFonts


\title{Math F113X: Homework Set 4}
%\author{The Author}
\date{}							% Activate to display a given date or no date

\begin{document}
\maketitle
%\section{}
%\subsection{}

%Homework assignment 1 is:
%\vspace{-1.5cm}


\fbox{\parbox{\textwidth}{

Answer the following problems from the Fair Division section:
\begin{quote} 6, A, 8, B, 11, 14, 16 \end{quote}
Note: It will be easier on the grader and on YOU if you do the problems in the order listed. 
}}

\hrulefill
\textbf{Problem A:} Two people, Aeden (A) and Beyonc\'{e} (B), are splitting a box of 6 labubu toys from the Macaron Series worth \$400. The chart below shows how each person values each toy, given in dollars.

\begin{tabular}{| c || c | c | c | c | c | c |}
\hline
& \multicolumn{6}{c}{Labubu Toy}\\
& Soymilk (S)&Lychee (L)&Grape (G) &Coconut (C) &Toffee (T) & Sesame Bean (SB) \\
\hline
Aeden &100&100&50&50&50&50\\
Beyonc\'{e}&100&80&80&80&40&20\\
\hline
\end{tabular}
	\begin{enumerate}
	\item Divide the box of toys using Divider-Chooser with Aeden as the Divider and Beyonc\'{e} as the Chooser. Indicate the final assignment of toys to people and the relative values of each share.
	\item Divide the box of toys using Divider-Chooser with Beyonc\'{e}  as the Divider and Aeden as the Chooser. Indicate the final assignment of toys to people and the relative values of each share.

	\end{enumerate}
	
\textbf{Problem B:} Three people, Aimee (A), Benito (B), and Cicely (C), are dividing a piece of land using the lone-divider method. The values of the three pieces of land in the eyes of each player are shown below.

\begin{tabular}{| c || c | c | c |}
\hline
& piece 1 & piece 2 & piece 3 \\
A&30\%&30\% &40\%\\
B&25\%&25\%&50\%\\
C &33.3\%&33.3\%&33.3\%\\
\hline
\end{tabular}

	\begin{enumerate}
	\item Who was the divider and how can you tell?
	\item Explain what happens in the next step of the lone-divider method and why?
	\item Suppose Cicely takes piece 1. What is the remaining land to Aimee and Benito?
	\item Suppose Aimee and Benito split the combined land using the Divider-Chooser method what is the minimum share of the original land each would receive? Would it be fair?
	\end{enumerate}

Remember to write up your homework solutions according to the homework writeup guidelines. 

Homework is graded using the following rubric for each problem (or problem part):

\begin{description}
\item[2 points:] You provided a complete answer, with supporting work, written up clearly
\item[1 point:] Some attempt at a solution, but incomplete writeup / unclear / illegible / no answer
\item[1 point:] Only an answer, with no supporting work 
\item[0 points:] Missing.
\end{description}

After you do the homework, you need to check your answers against the solutions! Then figure out your errors (if any) and revise your homework before you submit it. Finally, answer the reflection question.

Homework must be submitted on Gradescope.

\end{document}  