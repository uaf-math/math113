\documentclass[11pt, oneside]{article}   	% use "amsart" instead of "article" for AMSLaTeX format
\usepackage{geometry}                		% See geometry.pdf to learn the layout options. There are lots.
\geometry{letterpaper}                   		% ... or a4paper or a5paper or ... 
%\geometry{landscape}                		% Activate for rotated page geometry
\usepackage[parfill]{parskip}    		% Activate to begin paragraphs with an empty line rather than an indent
\usepackage{graphicx}				% Use pdf, png, jpg, or eps§ with pdflatex; use eps in DVI mode
								% TeX will automatically convert eps --> pdf in pdflatex		
\usepackage{amssymb, amsmath}

%SetFonts

%SetFonts



\begin{document}
\begin{center} {\huge{Homework Guidelines}} \end{center}

\section{Philosophy}

\textbf{This is where the majority of your learning will occur.}

Working through the homework thoughtfully is one of the best ways to l earn math. It's like practicing a sport or musical instrument -- you wouldn't expect to perform well in a game or concert without putting in the hours of practice beforehand. The same applies to quizzes, midterms, and projects in a math class: they tend to go better when you've engaged deeply with the homework leading up to them.

\textbf{If you don't have questions, you're not doing it right!}

I would expect \emph{every} student to have questions on \emph{every} assignment such as:	\begin{itemize}
	\item I understand what steps to do but I don't understand why I am doing them.
	\item I got the same answer but I did it a totally different way. Is my way ok or was it a fluke?
	\item Once I saw what I was supposed to do, I could do it. What part of the problem should have indicated to me what I was supposed to do?
	\item How did you do that step and why?
	\end{itemize}

\section{A Healthy Algorithm for Learning}

\textbf{A handful of simple actions can make homework more productive.}

\begin{enumerate}
\item[0.] Dedicate three 1-hour blocks between class meetings to work on assigned homework.
\item[1.] Start the assignment asap and finish it before the deadline.
\item[2.] For each part of each problem,
	\begin{enumerate}
	\item[a.] Attempt it first by yourself. You must write something.
	\item[b.] Use the solutions/teacher/tutor/friend/AI to ensure your answer \emph{and your work} is correct. 
	\item[c.] Write down your questions and ask them.
	\item[d.] \textbf{After completing each problem, indicate your level of mastery.}
	\end{enumerate}
\end{enumerate}

For 2d, I have seen students use symbols like {\Large{\textbf{\checkmark,$\thicksim$, X }}} to indicate ``mastery", "a bit wobbly", and ``really don't understand".
\section{Minimum Standards}

\textbf{You will not earn full-credit on the homework unless you meet the expectations below.}

\begin{enumerate}
\item All parts of all problems should be numbered/labeled and written in order.
\item You should start at the top of the page and work your way down.
\item There must be work to support your answer and your answer should be easy to find.

\vspace{.08in}

\begin{quote} 
Standards 1, 2, and 3 above mean you homework should look like:\\

\vspace{.08in}

\begin{tabular}{ll}
1. & work goes here followed by \fbox{answer.}\\
&\\
2.a. & work goes here followed by \fbox{answer.}\\
&\\
2.b. & work goes here and it's a lot\\
& so more work is here and it's also a lot\\
& finally I can state my \fbox{answer.}\\
\end{tabular}
\end{quote}
\vspace{.08in}
\item You work should be \textbf{correct} or \textbf{corrected}. 
\item Your work should be your own. Copying solutions is unacceptable.
\end{enumerate}

\end{document}  