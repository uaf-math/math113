%\documentclass[12pt]{report}
%
%\usepackage{graphicx}
%\usepackage{amsthm}
%\usepackage{latexsym}
%\usepackage{amsmath,amsthm}
%\usepackage{enumerate}
%\usepackage{soul}
%
%\usepackage[parfill]{parskip}
%
%\usepackage[margin=0.7in]{geometry}
%\usepackage{tikz}
%\usepackage{amsmath}
%\usetikzlibrary{calc,trees,positioning,arrows,fit,shapes,calc}
%\usepackage{pgfplots}
%\pgfplotsset{width=9cm,compat=1.9}
%\tikzset{
%  jumpdot/.style={mark=*,solid},
%  excl/.append style={jumpdot,fill=white},
%  incl/.append style={jumpdot,fill=blue},
%}
%\renewcommand{\arraystretch}{2.25}
%\newcommand{\f}[2]{\displaystyle\frac{#1}{#2}}
%
%\newcommand{\green}[1]{\textcolor{red}{#1}}
%
%\usepackage{array}
%\newcolumntype{L}[1]{>{\raggedright\let\newline\\\arraybackslash\hspace{0pt}}m{#1}}
%\newcolumntype{C}[1]{>{\centering\let\newline\\\arraybackslash\hspace{0pt}}m{#1}}
%\newcolumntype{R}[1]{>{\raggedleft\let\newline\\\arraybackslash\hspace{0pt}}m{#1}}
%
%\topmargin -1in
%\textheight 9.5in
%\oddsidemargin -0.3in
%\evensidemargin \oddsidemargin
%\pagestyle{empty}
%%\marginparwidth 0.5in
%\textwidth 7in
%\parindent 0in
%
%\renewcommand{\arraystretch}{0.9}
%
%\begin{document}
% 
%
%\parindent=0em
%
%\begin{center}
%{\bf Math F113X: Numbers and Society Project Option 3:} \\
%{\bf Creating an Encryption System} \\
%\end{center}
%\vspace{0.5cm}

\documentclass[12pt]{article}

\usepackage[margin = .8in]{geometry}
\usepackage{amsmath}
\usepackage{graphicx}
\usepackage[parfill]{parskip}
\usepackage{multicol, enumerate, tabularx}

\usepackage{adjustbox}

\usepackage{fancyhdr}
\pagestyle{fancy}

\lhead{Math F113X: Numbers and Society}
\rhead{Date: \hspace{1in}}

\usepackage{tikz}
\usetikzlibrary{calc,trees,positioning,arrows,fit,shapes,through, backgrounds}
\usetikzlibrary{patterns}

\usetikzlibrary{decorations.markings}
\usetikzlibrary{arrows}

\usepackage{pgfplots}

\usepackage{longtable}
\usepackage{tabularx}

\newcommand{\ds}{\displaystyle}
\newcommand{\ans}[1][1in]{\rule{#1}{.5pt}}

\newcommand{\points}[1]{(#1 points.)}		% Trying to be lazy.

\usepackage{array}
\newcolumntype{L}[1]{>{\raggedright\let\newline\\\arraybackslash\hspace{0pt}}m{#1}}
\newcolumntype{C}[1]{>{\centering\let\newline\\\arraybackslash\hspace{0pt}}m{#1}}
\newcolumntype{R}[1]{>{\raggedleft\let\newline\\\arraybackslash\hspace{0pt}}m{#1}}
\newcommand{\red}[1]{\textcolor{red}{#1}}

\newcommand{\be}{\begin{enumerate}}
\newcommand{\ee}{\end{enumerate}}

\newcommand{\bi}{\begin{itemize}}
\newcommand{\ei}{\end{itemize}}

%\topmargin -1in
%\textheight 9.5in
%\oddsidemargin -0.3in
%\evensidemargin \oddsidemargin
%\pagestyle{empty}
%%\marginparwidth 0.5in
%\textwidth 7in
%\parindent 0in

%--------------------------------------------------------------------------------------------------------------------------------------------------------------------------
%						Document
%--------------------------------------------------------------------------------------------------------------------------------------------------------------------------


\begin{document}
%\pagestyle{fancy}
%\begin{center}
%{\Large  Worksheet 15 (Scheduling 1): \\Priority Lists and Decreasing Time Algorithm}
%\end{center}



{\Large
\begin{center}
 {\sc Final Project: Creating an Encryption System} %\\
\end{center}
}

\section{Description of the project}

\subsection*{ Summary} For this project, you will be creating your own
encryption system. Then you will use
your encryption system to encrypt several messages of various
lengths. You will write out
instructions for decrypting your message and test these out on a
friend.


\subsection{Creating a 2-step Encryption System}
\begin{enumerate}
\item {\bf Create an encryption system that involves at least 2 steps}.
  Here are the requirements:
  \begin{enumerate}
  \item One step must be a {\bf substitution cipher}, meaning that you
    are replacing the
    characters in your message with different characters, following
    some established
    mapping (such as an alphanumeric Caesar cipher with a given
    shift).
  \item One step must be a {\bf transposition cipher}, meaning that
    you are changing the
    order of characters to obscure the message (such as tabular
    transposition).
  \item Each step needs to be compatible with numerical digits as
    well as English letters.
  \item Each step need to be {\bf reversible}, meaning that you can
    write down a step-by-step process to decrypt a message.
  \end{enumerate}
  
\item {\bf Give your encryption system a name}.
\item {\bf Write out an explanation of both encryption steps}.
  Use words and terms that someone
  who is not in the class would be able to understand.
\item {\bf Write out a step-by-step guide for decrypting a message
    with your system}. Use words
  and terms that someone who is not in the class would be able to
  understand and apply.
\end{enumerate}


\subsection*{Encrypting Messages}
\begin{enumerate}
  \item {\bf Test your encryption system by encrypting the
      following three short messages:}.
    \begin{enumerate}
    \item Encrypt the word {\bf PASSWORD}.
    \item Encrypt the phrase {\bf THANK YOU FOR YOUR HELP}.
    \item Encrypt the number {\bf 9074747332}.
    \end{enumerate}
  \item {\bf Create a long message} in English of at least 50
    words. Feel free to use a famous quote, a song lyric, a message of
    your own, or anything else, so long as it is coherent English.
  \item {\bf Use your encryption system to encrypt the long message you
      just created}.
\end{enumerate}


\subsection*{ Further Analysis}
\begin{enumerate}
\item {\bf Test your encrypted system with a friend} by providing your
  decryption instructions and
  your three encrypted short messages (not the long one). See if they
  can decode them.
\item {\bf Answer each of the following questions:}
  \begin{enumerate}
  \item How did your friend do? Were your decryption instructions
    clear enough?
  \item How hard do you think it would be for someone to decrypt
    these messages
    without knowing your encryption process? Is your encryption system
    secure?
  \end{enumerate}
\end{enumerate}

%\newpage

\section{Project Deliverables}

\subsection*{How to submit}
\begin{itemize}

\item Your project must have a {\bf Title Page},  containing your name and which project you are completing.

\item Your final project must be submitted in class during the final exam period. (Note that you can print out materials at the Student Success Center!)

\item Your final project must be typed.

\item You should use sentences to describe what each piece of your project is doing and what you are computing: one of your classmates should be able to read your project and understand what you are doing.

\item Your final project must be stapled together.
\end{itemize}

%{\bf 3 files} on Canvas format. Each file may contain multiple pages, but there should be precisely three files uploaded.
\subsection*{What to submit}

Your final project will have three sections. Each section should start on a separate page.

%Please submit the following {\bf 3 files} in .pdf format on Canvas.Each file may contain multiple pages, but there should beprecisely three files uploaded.

\begin{itemize}
\item A section titled {\bf My Encryption Method} containing:
  \begin{itemize}
  \item The name of your encryption system
  \item The explanation of both steps of your encryption process
  \item A step-by-step guide for encrypting a message with your
    system
  \item A step-by-step guide for decrypting a message with your
    system
  \end{itemize}
\item A section titled {\bf Encrypted Messages} containing:
  \begin{itemize}
  \item The encrypted form of the word PASSWORD
  \item The encrypted form of the sentence THANK YOU FOR YOUR HELP
  \item The encrypted form of the number 9074747332
  \item Your long encrypted message of at least 50 words.
  \item On a separate page, your original unencrypted long message.  \end{itemize}
\item A section titled {\bf Further Analysis} containing 
  \begin{itemize}
  \item Your answer to the question: \begin{quote}``How did your friend do? Were your
    decryption instructions clear enough?''\end{quote}
  \item You answer to the question: \begin{quote} ``How hard do you think it would be
    for someone to
    decrypt these messages without knowing your encryption process? Is
    your
    encryption system secure?''\end{quote}
  \end{itemize}
\end{itemize}

%\newpage
\section{Grading}

Your project will be graded out of 100 points using the following rubric:

\begin{enumerate}

\item Your encryption system (15 points)
\be
\item A clear description of your substitution
\item A clear description of your transposition, including a key word and its use
\item A clear description of how to encrypt a message using your system
\item A name for your encryption system
\item A system understandable by someone who is not in Math F113X
\item An encryption system compatible with numbers and letters 
\ee

\item Your encrypted messages (15 points)
\be
\item Correct encrypted form of the word PASSWORD
  \item Correct encrypted form of the sentence THANK YOU FOR YOUR HELP
  \item Correct encrypted form of the number 9074747332
  \item A long encrypted message of at least 50 words.
  \item The original unencrypted (plain text) long message, on a separate page. 
\ee

\item Decryption check (20 points)

\be
\item A clear description of how to decrypt your messages
\item A step-by-step guide for decrypting messages understandable by someone who is not in Math F113X
\item Encryption and decryption instructions such that the grader can successfully implement. 
\ee

\item How did your friend do? (20 points)
\be
\item Explain whether or not your friend successfully decrypted all three short messages using the step-by-step decryption guide.
\item State how long  it took your friend to decrypt your messages 
\item Assess the clarity of your step-by-step decryption guide and describe how it could be improved.
\ee

\item Answer to ``How hard do you think it
  would be for someone to decrypt these
  messages...'' 
  (20 pts)
  \be
  \item Provide an analysis of your encryption system
  \item Talk about at least one strength of your encryption system that would make it
 difficult to hack 
 \item Talk about at least one possible weakness of your encryption system that may make
 it vulnerable to hacking
 \item Give an overall assessment of the security of your
 encryption system's security
 \ee
 
 \item Grammar, mechanics, and following directions (10 points)
\bi
\item Use sufficient words and \textbf{complete sentences} in your discussions
\item Use correct grammar and mechanics in your writing
\item Use words and headings to make it clear what you are answering where
\item Computations should be presented clearly and legibly
\item Follow the directions
\ei

\ee

%\begin{tabular}{|m{7cm}|m{10cm}|}
%  \hline
%  {\bf Setup (10 pts)} & Clear description of your encryption system, using:
%  -- a substitution (explain)
%  
% -- a transposition, including a key word (explain)
  
%  +4 points if the grader cannot
%                                  determine how many times the
%                                  most common letter appears in your
%                                  long message (Test 1) \\
%                                & +3 points if the grader cannot
%                                  determine your substitution
%                                  cipher step from your 3 encrypted
%                                  short messages (Test 2) \\
%  The grader will try to 
%  decrypt your messages without
%  reading the encryption/decryption instructions.
%                                & +3 points if the grader cannot
%                                  determine your transposition
%                                  cipher step from your 3 encrypted
%                                  short messages (Test 3) \\
% \\ \hline
 % {\bf Encryption Check (30 pts)} & 
%  %+5 points if the explanation of
%                                    %your encryption system can be
%                                    understood by someone who is not
%                                    in Math F113X \\
%                                & +5 points if you used a substitution
%                                  cipher and transposition
%                                  cipher \\
%  The grader will use your
%  explanation to
%  check the three encrypted short
%  messages
%                                & +5 points if your encryption system
%                                  is compatible with numbers
%                                  and letters \\
%                                & +5 points for correct encryption of
%                                  PASSWORD \\
%                                & +5 points for correct encryption of
%                                  THANK YOU FOR YOUR HELP \\
%                                & +5 points for correct encryption of
%                                  the phone number \\
%  \hline
%  {\bf Decryption Test (30 pts)} & +5 points if your step-by-step
%                                   guide for decrypting messages
%                                   allows someone to correctly decrypt
%                                   any encrypted message \\
%                                & +15 points if the grader can
%                                  successfully decrypt your long
%                                  message using your step-by-step
%                                  decryption guide \\
%  The grader will 
%  attempt to use your guide
%  to decrypt your long
%  message
%                                & +10 points if your step-by-step
%                                  guide for decrypting messages
%                                  can be understood by someone who is
%                                  not in Math F113X \\
%  \hline
%  {\bf Answer to ``How did your friend do?''
%   (10 pts)}
%                                & +2 points for using 5-10 sentences
%                                  in your answer \\
%                                & +2 points for mentioning whether or
%                                  not your friend successfully
%                                  decrypted all three short messages
%                                  using the step-by-step
%                                  decryption guide. \\
%                                & +3 points for mentioning how long it
%                                  took your friend to decrypt
%                                  your messages \\
%                                & +3 points for assessing the clarity
%                                  of your step-by-step
%                                  decryption guide \\
%  \hline
%  {\bf Answer to ``How hard do you think it
%  would be for someone to decrypt these
%  messages...'' 
%  (10 pts)}
%                                & +2 points for using 5-10 sentences
%                                  in your answer \\
%                                & +3 points for talking about at least
%                                  one strength of your
%                                  encryption system that would make it
%                                  difficult to hack \\
%                                & +3 points for talking about at least
%                                  one possible weakness of
%                                  your encryption system that may make
%                                  it vulnerable to hacking \\
%                                & +2 points for giving an overall
%                                  assessment of the security of your
%                                  encryption system’s security \\
%  \hline
%\end{tabular}

\end{document}






%%% Local Variables:
%%% mode: latex
%%% TeX-master: t
%%% End:
