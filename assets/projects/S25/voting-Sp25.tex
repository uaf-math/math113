%\documentclass[12pt]{report}
%
%\usepackage{graphicx}
%\usepackage{amsthm}
%\usepackage{latexsym}
%\usepackage{amsmath,amsthm}
%\usepackage{enumerate}
%
%\usepackage[margin=0.7in]{geometry}
%\usepackage{tikz}
%\usepackage{amsmath}
%\usetikzlibrary{calc,trees,positioning,arrows,fit,shapes,calc}
%\usepackage{pgfplots}
%\pgfplotsset{width=9cm,compat=1.9}
%\tikzset{
%  jumpdot/.style={mark=*,solid},
%  excl/.append style={jumpdot,fill=white},
%  incl/.append style={jumpdot,fill=blue},
%}
%\renewcommand{\arraystretch}{2.25}
%\newcommand{\f}[2]{\displaystyle\frac{#1}{#2}}
%
%\newcommand{\green}[1]{\textcolor{red}{#1}}
%
%\usepackage{array}
%\newcolumntype{L}[1]{>{\raggedright\let\newline\\\arraybackslash\hspace{0pt}}m{#1}}
%\newcolumntype{C}[1]{>{\centering\let\newline\\\arraybackslash\hspace{0pt}}m{#1}}
%\newcolumntype{R}[1]{>{\raggedleft\let\newline\\\arraybackslash\hspace{0pt}}m{#1}}
%
%\topmargin -1in
%\textheight 9.5in
%\oddsidemargin -0.3in
%\evensidemargin \oddsidemargin
%\pagestyle{empty}
%%\marginparwidth 0.5in
%\textwidth 7in
%\parindent 0in
%
%\renewcommand{\arraystretch}{1.5}
%
%\begin{document}
% 
%
%\parindent=0em

\documentclass[12pt]{article}

\usepackage[margin = .8in]{geometry}
\usepackage{amsmath}
\usepackage{graphicx}
\usepackage[parfill]{parskip}
\usepackage{multicol, enumerate, tabularx}

\usepackage{adjustbox}

\usepackage{fancyhdr}
\pagestyle{fancy}

\lhead{Math F113X: Numbers and Society}
\rhead{Date: \hspace{1in}}

\usepackage{tikz}
\usetikzlibrary{calc,trees,positioning,arrows,fit,shapes,through, backgrounds}
\usetikzlibrary{patterns}

\usetikzlibrary{decorations.markings}
\usetikzlibrary{arrows}

\usepackage{pgfplots}

\usepackage{longtable}
\usepackage{tabularx}

\newcommand{\ds}{\displaystyle}
\newcommand{\ans}[1][1in]{\rule{#1}{.5pt}}

\newcommand{\points}[1]{(#1 points.)}		% Trying to be lazy.

\usepackage{array}
\newcolumntype{L}[1]{>{\raggedright\let\newline\\\arraybackslash\hspace{0pt}}m{#1}}
\newcolumntype{C}[1]{>{\centering\let\newline\\\arraybackslash\hspace{0pt}}m{#1}}
\newcolumntype{R}[1]{>{\raggedleft\let\newline\\\arraybackslash\hspace{0pt}}m{#1}}
\newcommand{\red}[1]{\textcolor{red}{#1}}

\newcommand{\be}{\begin{enumerate}}
\newcommand{\ee}{\end{enumerate}}

\newcommand{\bi}{\begin{itemize}}
\newcommand{\ei}{\end{itemize}}

%\topmargin -1in
%\textheight 9.5in
%\oddsidemargin -0.3in
%\evensidemargin \oddsidemargin
%\pagestyle{empty}
%%\marginparwidth 0.5in
%\textwidth 7in
%\parindent 0in

%--------------------------------------------------------------------------------------------------------------------------------------------------------------------------
%						Document
%--------------------------------------------------------------------------------------------------------------------------------------------------------------------------


\begin{document}
%\pagestyle{fancy}
%\begin{center}
%{\Large  Worksheet 15 (Scheduling 1): \\Priority Lists and Decreasing Time Algorithm}
%\end{center}



{\Large
\begin{center}
 Final Project: Applying Voting Methods to a Homemade Survey %\\
\end{center}
}
%\vspace{0.5cm}

%\begin{description}

\section{Description of the project}

\subsection*{Summary} For this project, you will conduct a %ranked-choice
survey with four choices for survey-takers to rank. You will calculate the winner using the
voting methods we have learned in this class and write a short
essay explaining which method you think is best for your survey.
\vspace{0.5cm}

\subsection*{Creating a survery and compiling results}
\begin{enumerate}
\item \emph{Create a survey of your own with 4 choices}. This can
  be an online survey using a platform like LimeSurvey or Survey
  Monkey, or you can make a paper ballot. The instructions on your
  survey should be clear, and you need to ask voters to rank the 4
  options from best to worst. (In other words, this is a preference
  ballot.)
\item \emph{Collect results for your survey} from at least 25
    people.
\item \emph{Make a preference schedule} using your survey
  results.
\end{enumerate}
\vspace{0.5cm}

\subsection*{Determining the winners}
\begin{enumerate}
  \item \emph{Determine the winner using the Plurality
      method}. Also determine who finished second, third, and fourth.
      
  \item \emph{Determine the winner using Instant Runoff Voting}.
    Use the fewest number of last place votes to break any
    ties.
  \item \emph{Determine the winner using Borda Count}.
    Also determine who finished second, third, and fourth. 
    Use the fewest number of last place votes to break any
    ties.
  \item \emph{Determine the winner using Copeland's
      method}. Also determine who finished second, third, and fourth.
    Use the fewest number of last place votes to break any
    ties.
  \item \emph{Determine the winner using your own original
      method}. For example, you could decide a winner using fewest
    last place methods, or an instant runoff voting system where the
    candidate with the most last place votes is eliminated. Clearly
    explain your method and show all your work.
\end{enumerate}


\subsection*{Further Analysis}
%\begin{enumerate}
%\item 
Based on the objective of your survey, which of the voting
  methods you used do think is most fair overall? Which is the least
  fair? Write a short answer (1-2 paragraphs with at
  least 10 sentences total) explaining your answer. This is not an
  explanation of how the methods work, but why your chosen method seems
  most/least fair to you. Mention all methods that you used, including
  your own made up method.


%\end{description}

%\newpage

%{\bf SUBMITTING YOUR PROJECT}

%\vspace{0.5cm}
\section{Project Deliverables}

\subsection*{How to submit}
\begin{itemize}

\item Your project must have a {\bf Title Page},  containing your name and which project you are completing.

\item Your final project must be submitted in class during the final exam period. (Note that you can print out materials at the Student Success Center!)

\item Your final project must be typed.

\item You should use sentences to describe what each piece of your project is doing and what you are computing: one of your classmates should be able to read your project and understand what you are doing.

\item Your final project must be stapled together.
\end{itemize}

%{\bf 3 files} on Canvas format. Each file may contain multiple pages, but there should be precisely three files uploaded.
\subsection*{What to submit}

Your final project will have three sections. Each section should start on a separate page.

\begin{enumerate}

\item A  section titled {\bf Winners} containing:
  \begin{itemize}
  \item Your preference schedule
  \item Your results using the Plurality method, with supporting work
  \item Your results using Instant Runoff Voting, with supporting work 
  \item Your results using Borda Count, with supporting work
  \item Your results using Copeland's Method, with supporting work
  \item Your explanation of your own method along with your results and supporting work
  \end{itemize}
  
  Each subsection above should be indicated with a clear heading.
  
\item A section titled {\bf Best Method} containing your answer, with justification, to
  the following question: 
  \begin{quote}``Which of the voting methods you used do you think is
  most fair overall? Which is the least fair?'' \end{quote}

\item A section titled {\bf Survey Results} with each voter's
  preference ballot. You may present these raw results using a table or spreadsheet, or include the actual ballots.

\end{enumerate}

%\newpage

%{\bf GRADING}
\section{Grading}

Your project will be graded out of 100 points using the following rubric:

\begin{enumerate}
\item Survey Results (10 points)
\bi
\item Your survey should have at least 25 ballots.
\item You should include the raw data as part of your project submission
\ei
\item Preference Schedule (10 points)
\bi
\item The numbers corresponding to each preference should sum to the total number of voters
\item You must correctly identify how many people chose each preference
\ei

\item Plurality Method (10 pts)

\bi
\item Correct identification of the plurality winner
\item Correct identification of the 2nd place, 3rd place, and 4th place winner
\item Supporting computations
\ei

\item Instant Runoff Voting (10 pts)
\bi
\item Clear description of what happens in each round (including redistribution of votes)
\item Clear communication about who gets eliminated in each round
\item Correct determination of the IRV winner 
\item Supporting computations
\ei

\item Borda Count (10 points)
\bi
\item Clear description of comptuations
\item Correct determination of the Borda Count winner and  2nd place, 3rd place, and 4th place 
\item Supporting computations
\ei

\item Copeland's Method (10 points)
\bi
\item List all head-to-head comparisons
\item Correctly determine the winner of all head-to-head comparisons
\item Provide computations supporting the determination of the winner using Copeland's method
\item correct determination of the winner and  2nd place, 3rd place, and 4th place 
\ei

\item Your own made up method (10 pts)
\bi
\item Provide a clear explanation of what your method is and how to implement it
\item Explain how your method is clearly different from the other four methods
\item Apply your method to determine the winner,  and  2nd place, 3rd place, and 4th place 
\item Supporting computations
\ei

\item Paragraph on most fair and least fair method (20 pts)
\bi
\item Clearly describe the objective of your survey
\item Describe the results of each voting method and how the results are similar or different
\item Decide which method is most fair and least fair
\item Provide justification on why you made those determinations
\item Include mathematical analysis to support that justificatoin
\ei

\item Grammar, mechanics, and following directions (10 points)
\bi
\item Use sufficient words and complete sentences in your discussions
\item Use correct grammar and mechanics in your writing
\item Use words and headings to make it clear what you are answering where
\item Computations should be presented clearly and legibly
\item Follow the directions
\ei

%& +3 points for a clear
%                                           explanation of your method \\
%                                & +3 points if your method
%                                  differentiates from our other 4
%                                  methods \\
%                                & +4 points for correctly choosing the
%                                  winner as well as 2nd place,
%                                  3rd place, and 4th place \\


\end{enumerate}

%\begin{tabular}{|m{7cm}|m{10cm}|}
%  \hline
%  {\bf Survey Results (10 pts)} & +10 points for obtaining {\bf at least 25
%                                  preference ballots} \\
%  \hline
%  {\bf Preference Schedule (10 pts)} & + 5 points if the numbers
%                                       corresponding to each
%                                       preference add up to the total
%                                       number of surveys
%                                       collected (it should!) \\
%                                & +5 points for correctly identifying
%                                  how many people chose each preference \\
%  \hline
%  {\bf Plurality Method (10 pts)} & +8 points for identifying the
%                                    number of first place votes for each
%                                    choice \\
%  & +2 points for correctly choosing the winner as well as 2nd place,
%    3rd place, and 4th place \\
%  \hline
%  {\bf Instant Runoff Voting (10 pts)} & +2 points for identifying
%                                        which choice gets eliminated first \\
%  & +4 points for correctly redistributing the eliminated choice’s votes
%    and identifying which choice gets eliminated next \\
%  & +4 points for correctly redistributing the eliminated choices’ and
%    determining the winner \\
%  \hline
%  {\bf Borda Count (10 pts)} & +8 points for correctly calculating
%                                  each choice’s Borda score \\
%  & +2 points for correctly choosing the winner as well as 2nd place,
%    3rd place, and 4th place \\
%  \hline
%  {\bf Copeland's Method (10 pts)} & +2 points for listing out all
%                                     direct comparisons (also called
%                                     head-to-head comparisons) \\
%                                & +6 points for correctly identifying
%                                  who wins in each direct
%                                  comparison \\
%                                & +2 points for correctly choosing the
%                                  winner as well as 2nd place,
%                                  3rd place, and 4th place \\
%  \hline
%  {\bf Your own made up method (10 pts)} & +3 points for a clear
%                                           explanation of your method \\
%                                & +3 points if your method
%                                  differentiates from our other 4
%                                  methods \\
%                                & +4 points for correctly choosing the
%                                  winner as well as 2nd place,
%                                  3rd place, and 4th place \\
%  \hline
%  {\bf Paragraph on most fair and least fair method (20 pts)} & +4
%                                                                points
%                                                                for
%                                                                a detailed answer with justification \\
%                                & +4 points for clarity and proper use
%                                  of vocabulary. \\
%                                & +3 points for explaining the
%                                  objective of your survey \\
%                                & +5 points for mentioning each method \\
%                                & +4 points for using good
%                                  mathematical analysis in your
%                                  explanation \\
%  \hline
%  
%  {\bf Instructions, Grammar and Mechanics} & +10 all instructions followed \\ 
%  & ???
%\end{tabular}

\end{document}






%%% Local Variables:
%%% mode: latex
%%% TeX-master: t
%%% End:
