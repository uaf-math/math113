

\documentclass[12pt]{article}

\usepackage[margin = .8in]{geometry}
\usepackage{amsmath}
\usepackage{graphicx}
\usepackage[parfill]{parskip}
\usepackage{multicol, enumerate, tabularx}

\usepackage{adjustbox}

\usepackage{fancyhdr}
\pagestyle{fancy}

\renewcommand{\emph}[1]{\textsf{\textbf{#1}}}

\lhead{Math F113X: Numbers and Society}
\rhead{Date: \hspace{1in}}

\usepackage{tikz}
\usetikzlibrary{calc,trees,positioning,arrows,fit,shapes,through, backgrounds}
\usetikzlibrary{patterns}

\usetikzlibrary{decorations.markings}
\usetikzlibrary{arrows}

\usepackage{pgfplots}

\usepackage{longtable}
\usepackage{tabularx}

\newcommand{\ds}{\displaystyle}
\newcommand{\ans}[1][1in]{\rule{#1}{.5pt}}

\newcommand{\points}[1]{(#1 points.)}		% Trying to be lazy.

\usepackage{array}
\newcolumntype{L}[1]{>{\raggedright\let\newline\\\arraybackslash\hspace{0pt}}m{#1}}
\newcolumntype{C}[1]{>{\centering\let\newline\\\arraybackslash\hspace{0pt}}m{#1}}
\newcolumntype{R}[1]{>{\raggedleft\let\newline\\\arraybackslash\hspace{0pt}}m{#1}}
\newcommand{\red}[1]{\textcolor{red}{#1}}

\newcommand{\be}{\begin{enumerate}}
\newcommand{\ee}{\end{enumerate}}

\newcommand{\bi}{\begin{itemize}}
\newcommand{\ei}{\end{itemize}}

%\topmargin -1in
%\textheight 9.5in
%\oddsidemargin -0.3in
%\evensidemargin \oddsidemargin
%\pagestyle{empty}
%%\marginparwidth 0.5in
%\textwidth 7in
%\parindent 0in

%--------------------------------------------------------------------------------------------------------------------------------------------------------------------------
%						Document
%--------------------------------------------------------------------------------------------------------------------------------------------------------------------------


\begin{document}


{\Large
\begin{center}
 {\sc Final Project: Encryption} %\\
\end{center}
}

\section{Description of the project}

\subsection*{Summary} For this project, you will 
\be
\item Choose a plaintext of at least 50 characters and  encrypt it using
\be
\item A shift (Caesar) cipher with your choice of shift;
\item A transposition cipher with a key of your choice, of at least 5 letters;
\item a Vigen\`{e}re cipher, using the same key;
\item a two-step encryption system that you create (``Your Encryption Method''), described in the next section.
\ee
\item Then you will write out careful instructions for decrypting a message using your 2-step system and test these instructions out on a friend, by seeing if they can successfully decrypt the plaintext that you encrypted using Your Encryption Method.
\item Finally, you will analyze Your Encryption Method.
\ee


\subsection*{Creating a 2-step Encryption System}
\begin{enumerate}
\item \emph{ Create an encryption system that involves at least 2 steps}.
  Here are the requirements:
  \begin{enumerate}
  \item One step must be a \emph{substitution cipher}, meaning that you
    are replacing the
    characters in your message with different characters, following some established mapping (such as an alphanumeric Caesar cipher with a given shift).
  \item One step must be a \emph{transposition cipher}, meaning that you are changing the order of characters to obscure the message (such as tabular
    transposition).
    \item You need to use a different substitution cipher and transposition keyword than you use in other parts of this project.
% \item Each step needs to be compatible with numerical digits as well as English letters.
  \item Each step need to be \emph{reversible}, meaning that you can
    write down a step-by-step process to decrypt a message.
  \end{enumerate}
  
\item Give your encryption system a name.
\item Write out an explanation of both encryption steps.
  Use words and terms that someone
  \emph{who is not in the class} can understand.
\item Write out a step-by-step guide for decrypting a message with your system. Use words and terms that someone \emph{who is not in the class} would be able to understand and apply.
\end{enumerate}


%\subsection*{Encrypting Messages}
%\begin{enumerate}
%  \item {\bf Test your encryption system by encrypting the
%      following three short messages:}.
%    \begin{enumerate}
%    \item Encrypt the word {\bf PASSWORD}.
%    \item Encrypt the phrase {\bf THANK YOU FOR YOUR HELP}.
%    \item Encrypt the number {\bf 9074747332}.
%    \end{enumerate}
%  \item {\bf Create a long message} in English of at least 50
%    words. Feel free to use a famous quote, a song lyric, a message of
%    your own, or anything else, so long as it is coherent English.
%  \item {\bf Use your encryption system to encrypt the long message you
%      just created}.
%\end{enumerate}


\subsection*{ Further Analysis}
\begin{enumerate}
\item \emph{Test your decryption instructions with a friend} by providing your decryption instructions and your ciphertext that was encrypted using Your Encryption Method.
\item \emph{Answer each of the following questions:}
  \begin{enumerate}
  \item How did your friend do? Were your decryption instructions
    clear enough?
  \item How hard do you think it would be for someone to decrypt
    these messages
    without knowing your encryption process? Is your encryption system
    secure?
  \end{enumerate}
\end{enumerate}

%\newpage

\section{Project Deliverables}

\subsection*{How to submit}
\begin{itemize}

\item Your project must have a {\bf Title Page},  containing your name and which project you are completing.

\item Your final project must be submitted in class during the final exam period. (Note that you can print out materials at the Student Success Center!)

\item Your final project must be typed.

\item You should use sentences to describe what each piece of your project is doing and what you are computing: one of your classmates should be able to read your project and understand what you are doing.

\item Your final project must be stapled.
\end{itemize}

%{\bf 3 files} on Canvas format. Each file may contain multiple pages, but there should be precisely three files uploaded.
\subsection*{What to submit}

Your final project must contain the following sections. Each section should start on a separate page.

%Please submit the following {\bf 3 files} in .pdf format on Canvas.Each file may contain multiple pages, but there should beprecisely three files uploaded.


\begin{itemize}

\item A section titled \emph{My Encryption Method} containing:
  \begin{itemize}
  \item The name of your encryption system
%  \item The explanation of both steps of your encryption process
  \item A step-by-step guide for \emph{encrypting} a message with your system, clearly explaining how both steps work
  \item A clear, step-by-step guide for \emph{decrypting} a message with your system, that is understandable by someone who is not taking/has not taken Math F113X
  \end{itemize}
  
\item A section titled \emph{Encrypted Messages} containing:
  \begin{itemize}
  \item Your plaintext
  \item Your plaintext encrypted using a shift cipher. You must clearly say what shift you are using. \emph{Prohibited shifts}: 0 ($A \to A$), $1$ ($A \to B$) and $-1$ ($A \to Z$).
  \item A choice of keyword of at least 5 letters.
  \item Your plaintext encrypted using a transposition cipher using your keyword. Include supporting work for the encryption.
  \item Your plaintext encrypted using a Vigen\`ere cipher using your keyword. Include supporting work for the encryption.
  \item Your plaintext encrypted using Your Encryption Method. Include supporting work for the encryption.
\end{itemize}

\item A section titled \emph{Further Analysis} containing 
  \begin{enumerate}
  \item Your answer to the question: \begin{quote}``How did your friend do? Were your
    decryption instructions clear enough?''\end{quote}
    This should include answers to the following:
    \be
    \item A description of your process of getting your friend to do the decryption (what happened?)
    \item Whether or not your friend successfully decrypted your plaintext using the step-by-step decryption guide.
    \item How long it took
    \item An assessment of the clarity of your decryption guide and how it might need to be improved 
    \ee
  \item Your answer to the question: \begin{quote} ``How hard do you think it would be
    for someone to
    decrypt these messages without knowing your encryption process? Is
    your
    encryption system secure?''\end{quote}
    
    You should address the following:
    \be
  \item Talk about at least one strength of your encryption system that would make it difficult to hack 
 \item Talk about at least one possible weakness of your encryption system that may make it vulnerable to hacking
 \item Give an overall assessment of the security of your
 encryption system
 \ee
  \end{enumerate}
  These answers should be written in complete sentences, and they should fully explain your answer to the questions, with justification.
  
\end{itemize}

%\newpage
\section{Grading}

Your project will be graded out of 100 points using the following rubric:

\begin{enumerate}

\item \emph{My encryption system} (15 points)
\be
\item Substitution described clearly
\item  Transposition described clearly, including a key word and its use
\item Named encryption system
\item Clear encryption instructions provided that is understandable by someone who is not in Math F113X 
\ee

\item \emph{Encrypted messages} (25 points)
\be
\item The plaintext, shift, and keyword  used are clear
\item Plaintext was correctly encrypted using the shift cipher
 \item Plaintext was correctly encrypted using a transposition cipher, with supporting work
  \item Plaintext was correctly encrypted using a Vigen\`{e}re cipher, with supporting work
  \item Plaintext was correctly encrypted using Your Encryption Method, with supporting work
\ee

\item \emph{Decryption check} (10 points)

\be
%\item A clear description of how to decrypt your messages
\item A clear step-by-step guide for decrypting messages was provided 
\item The guide is understandable by someone who is not in Math F113X
\ee

\item \emph{How did your friend do?} (20 points)
\be
\item Clear explanation of whether or not your friend successfully decrypted your plaintext using the step-by-step decryption guide.
\item Stated how long  it took your friend to decrypt your messages 
\item Provided assessment of the clarity of the step-by-step decryption guide and described how it could be improved.
\ee

\item \emph{Answer to ``How hard do you think it
  would be for someone to decrypt these
  messages...'' }
  (20 pts)
  \be
  \item Analyzed Your Encryption System
  \item Discussed least one strength of Your Encryption System that would make it
 difficult to hack 
 \item Discussed of at least one possible weakness of Your Encryption System that may make
 it vulnerable to hacking
 \item Overall assessment of the security of Your Encryption System \ee
 
 \item \emph{Grammar, mechanics, and following directions} (10 points)
\bi
\item Used sufficient words and \emph{complete sentences} in your discussions
\item Used correct grammar and mechanics in your writing
\item Used words and headings to make it clear what you are answering where
\item Computations are presented clearly and legibly
\item Followed the directions
\ei

\ee

%\begin{tabular}{|m{7cm}|m{10cm}|}
%  \hline
%  {\bf Setup (10 pts)} & Clear description of your encryption system, using:
%  -- a substitution (explain)
%  
% -- a transposition, including a key word (explain)
  
%  +4 points if the grader cannot
%                                  determine how many times the
%                                  most common letter appears in your
%                                  long message (Test 1) \\
%                                & +3 points if the grader cannot
%                                  determine your substitution
%                                  cipher step from your 3 encrypted
%                                  short messages (Test 2) \\
%  The grader will try to 
%  decrypt your messages without
%  reading the encryption/decryption instructions.
%                                & +3 points if the grader cannot
%                                  determine your transposition
%                                  cipher step from your 3 encrypted
%                                  short messages (Test 3) \\
% \\ \hline
 % {\bf Encryption Check (30 pts)} & 
%  %+5 points if the explanation of
%                                    %your encryption system can be
%                                    understood by someone who is not
%                                    in Math F113X \\
%                                & +5 points if you used a substitution
%                                  cipher and transposition
%                                  cipher \\
%  The grader will use your
%  explanation to
%  check the three encrypted short
%  messages
%                                & +5 points if your encryption system
%                                  is compatible with numbers
%                                  and letters \\
%                                & +5 points for correct encryption of
%                                  PASSWORD \\
%                                & +5 points for correct encryption of
%                                  THANK YOU FOR YOUR HELP \\
%                                & +5 points for correct encryption of
%                                  the phone number \\
%  \hline
%  {\bf Decryption Test (30 pts)} & +5 points if your step-by-step
%                                   guide for decrypting messages
%                                   allows someone to correctly decrypt
%                                   any encrypted message \\
%                                & +15 points if the grader can
%                                  successfully decrypt your long
%                                  message using your step-by-step
%                                  decryption guide \\
%  The grader will 
%  attempt to use your guide
%  to decrypt your long
%  message
%                                & +10 points if your step-by-step
%                                  guide for decrypting messages
%                                  can be understood by someone who is
%                                  not in Math F113X \\
%  \hline
%  {\bf Answer to ``How did your friend do?''
%   (10 pts)}
%                                & +2 points for using 5-10 sentences
%                                  in your answer \\
%                                & +2 points for mentioning whether or
%                                  not your friend successfully
%                                  decrypted all three short messages
%                                  using the step-by-step
%                                  decryption guide. \\
%                                & +3 points for mentioning how long it
%                                  took your friend to decrypt
%                                  your messages \\
%                                & +3 points for assessing the clarity
%                                  of your step-by-step
%                                  decryption guide \\
%  \hline
%  {\bf Answer to ``How hard do you think it
%  would be for someone to decrypt these
%  messages...'' 
%  (10 pts)}
%                                & +2 points for using 5-10 sentences
%                                  in your answer \\
%                                & +3 points for talking about at least
%                                  one strength of your
%                                  encryption system that would make it
%                                  difficult to hack \\
%                                & +3 points for talking about at least
%                                  one possible weakness of
%                                  your encryption system that may make
%                                  it vulnerable to hacking \\
%                                & +2 points for giving an overall
%                                  assessment of the security of your
%                                  encryption system’s security \\
%  \hline
%\end{tabular}

\end{document}






%%% Local Variables:
%%% mode: latex
%%% TeX-master: t
%%% End:
