% !TEX TS-program = pdflatexmk
\documentclass[12pt]{article}

% Layout.
\usepackage[top=1in, bottom=0.75in, left=1in, right=1in, headheight=1in, headsep=6pt]{geometry}

% Fonts.
\usepackage{mathptmx}
\usepackage[scaled=0.86]{helvet}
\renewcommand{\emph}[1]{\textsf{\textbf{#1}}}

% Misc packages.
\usepackage{amsmath,amssymb,latexsym}
\usepackage{graphicx,hyperref}
\usepackage{array}
\usepackage{xcolor}
\usepackage{multicol}
\usepackage{tabularx,colortbl,booktabs,xparse}
\usepackage{enumitem}

% Rotation: \rot[<angle>][<width>]{<stuff>}
\NewDocumentCommand{\rot}{O{45} O{1em} m}{\makebox[#2][l]{\rotatebox{#1}{#3}}}%

\usepackage{fancyhdr}
\pagestyle{fancy} 
\lhead{\large\sf\textbf{MATH F113X: Fair Division}}
\chead{\large\sf\textbf{lecture notes}}
\rhead{\large\sf\textbf{Sealed Bids}}

\begin{document}
Goal: Understand the method of \emph{Sealed Bids}.
\begin{enumerate}
\item Sealed bids: general idea
\vspace{1in}
\item Steps:
\begin{enumerate}
\item List items to be divided
\item Individuals submit a sealed bid for each item
\item Determine each person's total bid
\item Determine each person's fair share:  \: (their total bid) / (\# people)
\item Award each item to the highest bidder
\item Determine each person's initial allocation of money to holding pot: 
\begin{itemize}
\item if award $-$ fair share is positive, individual pays difference \emph{into} pot
\item if award $-$ fair share is negative, individual gets difference \emph{from} pot
\end{itemize}
\item Divide remaining surplus in pot equally
\end{enumerate}
\item \fbox{Example} ITEMS: Vase, Earrings, Ice Skates. PEOPLE: Sam, Omar

\begin{tabular}{|p{2in} || p{2in} |p{2in} | p{1in}}
\hline
& Sam & Omar \\ \hline \hline
vase & \$10 & \$5 \\[12 pt] \hline
earrings & \$4 & \$15 \\[12 pt] \hline
ice skates & \$20 & \$16 \\[12 pt] \hline
total bid & & \\[12 pt] \hline
\textbf{fair share} & & \\[12 pt] \hline
 who gets what (award)&& \\[12pt] \hline
 \textbf{award value}&& \\[12pt] \hline
 (award value) $-$ (fair share) && \\[12 pt] \hline
 pays in / receives && \\[12 pt] \hline
 total surplus &\multicolumn{2}{c|}{\quad} \\[12 pt] \hline
 share of surplus && \\[12 pt] \hline
 \textbf{Final Allocation} && \\[24 pt] \hline

\end{tabular}

\newpage

\item Same Example with different bids: %ITEMS: Vase, Earrings, Ice Skates. PEOPLE: Sam, Omar

\begin{tabular}{|p{2in} || p{2in} |p{2in} | p{1in}}
\hline
& Sam & Omar \\ \hline \hline
vase & \$10 & \$5 \\[12 pt] \hline
earrings & \$4 & \$1 \\[12 pt] \hline
ice skates & \$20 & \$10 \\[12 pt] \hline
total bid & & \\[12 pt] \hline
\textbf{fair share} & & \\[12 pt] \hline
 who gets what (award)&& \\[12pt] \hline
 \textbf{award value}&& \\[12pt] \hline
 (award value) $-$ (fair share) && \\[12 pt] \hline
 pays in / receives && \\[12 pt] \hline
 total surplus &\multicolumn{2}{c|}{\quad} \\[12 pt] \hline
 share of surplus && \\[12 pt] \hline
 \textbf{Final Allocation} && \\[24 pt] \hline

\end{tabular}

\end{enumerate}
\end{document}