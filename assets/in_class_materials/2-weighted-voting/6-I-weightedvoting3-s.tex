% !TEX TS-program = pdflatexmk
\documentclass[12pt]{article}

% Layout.
\usepackage[top=1in, bottom=0.75in, left=1in, right=1in, headheight=1in, headsep=6pt]{geometry}

% Fonts.
\usepackage{mathptmx}
\usepackage[scaled=0.86]{helvet}
\renewcommand{\emph}[1]{\textsf{\textbf{#1}}}

% Misc packages.
\usepackage{amsmath,amssymb,latexsym}
\usepackage{graphicx,hyperref}
\usepackage{array}
\usepackage{xcolor}
\usepackage{multicol}
\usepackage{tabularx,colortbl,booktabs,xparse}
\usepackage{enumitem}

% Rotation: \rot[<angle>][<width>]{<stuff>}
\NewDocumentCommand{\rot}{O{45} O{1em} m}{\makebox[#2][l]{\rotatebox{#1}{#3}}}%

\usepackage{fancyhdr}
\pagestyle{fancy} 
\lhead{\large\sf\textbf{MATH F113X: Weighted Voting}}
\chead{\large\sf\textbf{lecture notes}}
\rhead{\large\sf\textbf{Wrap-up}}

\begin{document}
\begin{center} Some Solutions \end{center}
\begin{enumerate}
\item In earlier notes, we found seven possible coalitions with players $P_1$, $P_2$, and $P_3.$ 
	\begin{enumerate}
	\item List them again below.\\
	
	$P_1$\\
	$P_2$\\
	$P_3$\\
	$P_1,P_2$\\
	$P_1,P_3$\\
	$P_2,P_3$\\
	$P_1,P_2,P_3$\\
	
	total number: 7
	\vfill
	\item Suppose the system has a fourth player, $P_4$. Determine how many coalitions in this case. Try to answer the question without actually listing all of them.\\
	
	Observation 1: All of the coalitions above are also coalitions with players $P_1$, $P_2$,  $P_3$, and $P_4.$ So we get the 7 above. \\
	
	Observation 2: Every time we add $P_4$ to a coalition above, we get another one. For example, adding $P_4$ to the first three in the list gives: \\
	
	$P_1,P_4$\\
	$P_2,P_4$\\
	$P_3,P_4$\\
	
	Observation 3: We missed out coalition $P_4$\\
	
	total number: $7+7+1=15$

	\vfill
	\item What if there is a fifth player, $P_5$?\\
	
	guess: $15 + 15 +1 = 31$
	\vfill
	\item Make a conjecture about how many coalitions are possible with $n$ players, $P_1, P_2, P_3, \cdots P_n.$ How would you argue that your count is correct?\\
	
	What is the pattern you see? 
	
	$7=8-1=2^3 -1$\\
	$15=16-1=2^4-1$\\
	$31=32-1=2^5-1$\\
	guess: With $n$ players there are $2^n-1$ different coalitions.
	 \vfill
	 \item What does this suggest about the mechanics of calculating the Banzhaf Power Index for a weighted voting system with a lot of players?
	 
	 $2^n$ grows really really fast. \\
	 While $2^5=32,$ $2^{10}= 1024,$ and $2^{20} > 1,000,000$. \\
	 Listing \emph{all} coalitions is, in general, not practical.
	 \vfill
	\end{enumerate}
%\newpage
\item What is the Banzhaf Power Index
	\begin{enumerate}
	\item when there is a dictator\\
	
	dictator has 100\%, all others have 0\%\\
	\vfill
	\item for a dummy player\\
	
	always has 0\%
	\vfill
	\item if all players have an equal number of votes\\
	
	All players have equal power. Might has well have 1 player 1vote.
	\vfill
	\item if player $P_1$ has double the number of votes as player $P_2$?\\
	
	All you can conclude is that $P_2$ will not have \emph{more} power than $P_1.$ But beyond that there isn't enough information. We have seen examples in which such players have equal power and we can construct others where $P_1$ is a dictator: $[10: 10, 5, 2]$
	\vfill
	\end{enumerate}
\end{enumerate}
\end{document}