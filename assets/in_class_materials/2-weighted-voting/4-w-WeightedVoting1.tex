\documentclass[12pt]{article}

\usepackage[margin = .8in]{geometry}
\usepackage{amsmath}
\usepackage{graphicx}
\usepackage{multicol}

\usepackage{fancyhdr}
\pagestyle{fancy}

\lhead{Math F113X: Numbers and Society}
\rhead{}

\usepackage{tikz}
\usetikzlibrary{calc,trees,positioning,arrows,fit,shapes,calc}
\usepackage{pgfplots}

\usepackage{longtable}
\usepackage{tabularx}

\newcommand{\ds}{\displaystyle}
\newcommand{\ans}[1][1in]{\rule{#1}{.5pt}}

\newcommand{\be}{\begin{enumerate}}
\newcommand{\ee}{\end{enumerate}}

\newcommand{\points}[1]{(#1 points.)}		% Trying to be lazy.

\usepackage{array}
\newcolumntype{L}[1]{>{\raggedright\let\newline\\\arraybackslash\hspace{0pt}}m{#1}}
\newcolumntype{C}[1]{>{\centering\let\newline\\\arraybackslash\hspace{0pt}}m{#1}}
\newcolumntype{R}[1]{>{\raggedleft\let\newline\\\arraybackslash\hspace{0pt}}m{#1}}
\newcommand{\red}[1]{\textcolor{red}{#1}}

%\topmargin -1in
%\textheight 9.5in
%\oddsidemargin -0.3in
%\evensidemargin \oddsidemargin
%\pagestyle{empty}
%%\marginparwidth 0.5in
%\textwidth 7in
%\parindent 0in

%--------------------------------------------------------------------------------------------------------------------------------------------------------------------------
%						Document
%--------------------------------------------------------------------------------------------------------------------------------------------------------------------------


\begin{document}
%\pagestyle{fancy}
\begin{center}
{\Large  Worksheet 4:  Weighted Voting Systems	 	}
\end{center}



%\noindent \textbf{Group Names:} \hrulefill \\
%-------------------------------------------------------------------------------------------------------------
%						Assignment
%-----------------------------------------------------------------------------------------------------
\begin{enumerate}

\item Consider the weighted voting system [35: 10,10,9,5,5]
\be
\item How many players are there? \ans
\vfill
\item What is the total number (weight) of votes? \ans
\vfill
\item What is the quota in this system? \ans
\vfill
\item Find all winning coalitions for this system. (Hint: There aren't very many...)
\vfill
\item Is there a dictator? Justify your answer.
\vfill
\item Do any players have veto power?  Justify your answer.
\vfill
\item Are there any dummy players?  Justify your answer.
\vfill
\item Is it possible to change the quota in this voting system such that it has a dictator? (Note that you are not allowed to change the voting weights.)
\vfill
\ee

\newpage
\item Five friends decide to start a business. They decide on a weighted voting system where the weight is determined by the number of hours worked per week. Bill worked 15 hours, Tammy worked 8 hours,
    Dara worked 7 hours, Priyanka worked 3 hours, and Ross worked 2 hours. Any
    decision that their company makes requires  a \emph{majority} of the votes.
    
 \be

  \item   What is the total weight of this voting system?
  \vfill
  
  \item What is the quota? (show your work) 
  \vfill
 
  
 \item Write the $[q: w_1, w_2, \cdots, w_n]$ notation for this voting system.\hrulefill
  
 \vfill

\item Determine {all} winning coalitions with \emph{at most 3} players. List the players and the total weight of each coalition. (Hint: There are 10.)
\vspace{2.5in}

\item For each coalition above, circle the players that are \emph{critical} to that coalition. 
\item Is there a dictator? Justify your answer.
\vfill
\item Do any players have veto power?  Justify your answer.
\vfill
\item Are there any dummy players?  Justify your answer.
\vfill
\item Is it possible to change the value of the quota such that Bill has veto power? (Note that you are not allowed to change the voting weights.)
\vfill

\ee
\ee
\end{document}

Then use that information to determine whether there are any dictators, dummies, or players with veto power. (explain)

\vfill
  
%  \item Are there any dictators? Show calculations explaining which player is a dictator, or else showing why there are no dictators.
%  
%  \vfill
%  
%    
%\item Are there dummies? Show calculations explaining which player(s) is/are a dummy, or else showing why there are none.
%
% \vfill
%
%\item Are there players with veto power? Show calculations explaining which player(s) have veto power, or else show why there are none.
%Construct a weighted voting system to represent each of the following scenarios. Then, determine if there are any dictators, dummies, or people with veto power.

 

    \vfill
   % \ee
    
    \newpage
    
\item  Consider the previous scenario, but suppose making a decision requires {\bf a majority}. %Then determine if there are any dictators, dummies, or people with veto power.
    
    \begin{enumerate}
    
    \item What is the new quota? \ans
    
    \vspace{1cm}
    
    \item new Weighted Voting System: \hrulefill
    
    
   \item Determine all winning coalitions using two or three players
   
   \vfill
   
   \vfill
    
       \item Dictators? (explain why/why not) 
    
\vfill    
     \item Dummies? (explain why/why not) 
     
\vfill     
     \item Players with veto power? (explain why/why not) 


  \vfill
    
    \ee
%    \newpage

%  \item A condo community is voting to approve a 10K loan to fix 
%    structural and foundation issues with the building. If passed, the
%    homeowners are responsible for paying off the loan over the next five
%    years. This would result in an increase in the homeowners' monthly HOA
%    fee until the loan is paid off. There are 2 homeowners and 2 board members. 
%    
%    For this measure to pass both homeowners and one board member must vote yes. %Determine a weighted  voting system to represent this scenario. %Time permitting, decide if there are any dictators, dummies, or people with veto power.
%    
%    Assume the homeowners (H) are listed first and the board members (B) second. Explain why the weighted voting system [5: 2, \ 2, \ 1, \ 1] satisfies this setup, by showing:
%    \be
%    \item Does every coalition HHB make quota?
%    \vfill
%     \item Does every other coalition NOT make quota?
%     \vfill
%%

%    \vfill
%    
%    \vfill
    
    
\ee
\ee

\end{document}

%-------------------------------------------------------------------------------------------------------------------------------------------------------------------------------------------------------------------

%%% Local Variables:
%%% mode: latex
%%% TeX-master: t
%%% End:
