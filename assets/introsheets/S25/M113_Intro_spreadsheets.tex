

\documentclass[12pt]{article}

\usepackage[margin = .8in]{geometry}
\usepackage{amsmath}
\usepackage{graphicx}
\usepackage{multicol, enumerate, tabularx}

\usepackage[parfill]{parskip}

\usepackage{adjustbox, soul}

\usepackage{fancyhdr}
\pagestyle{fancy}

\lhead{Math F113X: Numbers and Society}
%\rhead{Date: \hspace{1in}}

\usepackage{tikz}
\usetikzlibrary{calc,trees,positioning,arrows,fit,shapes,through, backgrounds}
\usetikzlibrary{patterns}

\usetikzlibrary{decorations.markings}
\usetikzlibrary{arrows}

\usepackage{pgfplots}

\usepackage{longtable}
\usepackage{tabularx}

\newcommand{\ds}{\displaystyle}
\newcommand{\ans}[1][1in]{\rule{#1}{.5pt}}

\newcommand{\points}[1]{(#1 points.)}		% Trying to be lazy.

\usepackage{array}
\newcolumntype{L}[1]{>{\raggedright\let\newline\\\arraybackslash\hspace{0pt}}m{#1}}
\newcolumntype{C}[1]{>{\centering\let\newline\\\arraybackslash\hspace{0pt}}m{#1}}
\newcolumntype{R}[1]{>{\raggedleft\let\newline\\\arraybackslash\hspace{0pt}}m{#1}}
\newcommand{\red}[1]{\textcolor{red}{#1}}

\newcommand{\be}{\begin{enumerate}}
\newcommand{\ee}{\end{enumerate}}

%\topmargin -1in
%\textheight 9.5in
%\oddsidemargin -0.3in
%\evensidemargin \oddsidemargin
%\pagestyle{empty}
%%\marginparwidth 0.5in
%\textwidth 7in
%\parindent 0in

%--------------------------------------------------------------------------------------------------------------------------------------------------------------------------
%						Document
%--------------------------------------------------------------------------------------------------------------------------------------------------------------------------


\begin{document}
%\pagestyle{fancy}
\begin{center}
{\Large  Introduction to Spreadsheets and Simple Interest}
\end{center}
\begin{enumerate}

\item Getting started with spreadsheets. 
\be
\item Rectangles in spreadsheets are called \emph{cells} and they are identified by their row (numbers)  and column (letters) . The upper left rectangle is \verb`A1`.
\item To calculate something, click in a cell and start the calculation with \verb`=`. For example, to add 3+4, click in a cell and type \verb`=3+4` and then hit return. To multiply, you must type \verb`*`.
\ee

\item Some starting examples: compute the following
\be
\item To add 3 + 4, enter \verb`=3+4`
\item To subtract 100-76, enter \verb`=100-76`
\item To multiply 4 times 18, enter \verb`=4*18`
\item To divide 0.05 by 12, enter \verb`=0.05/12`
\item To calculate , enter \verb`=5^25`
\ee

\item Use a spreadsheet to compute an 18\% tip on a \$35.75 bill. 
\be
\item What is 18\% as a decimal? \ans
\item What should we enter into the spreadsheet? \ans
\ee

\item Suppose we wanted to be able to make a tip calculator, where you could enter the bill, and enter a tip percent, and have it automatically compute the additional tip and the total. We will use \emph{cell references}.

\be
\item In cell \verb`A1`, type \verb`Bill Amount`
\item In cell \verb`B1`, type \verb`Tip Total`
\item In cell \verb`C1`, type \verb`Bill Total`
\item In cell \verb`A2`, type \verb`35.75`
\item In cell \verb`B2`, type \verb`=0.18*A2`, (or type \verb`=0.18*` and then click on  cell \verb`A2`)
\item In cell \verb`C2`, type \verb`=A2 + B2` (or click on the corresponding cells)
\ee

\includegraphics[width = 4in]{SpreadsheetPic1.pdf}

What happens if you change the bill amount? \ans

What would you need to change if you wanted to give a 20\% tip? \ans

\newpage

\begin{minipage}{.4\textwidth}
\item We can use \emph{Fill Down} to quickly recalculate changes in values.

\be
\item Change the value in cell \verb`A2` to 10.
\item In cell \verb`A3` enter 20.
\item Select both cells and drag down until you get to 110.
\item Drag down the values in cells \verb`B2` and \verb`C2`.
\item How much is the tip on a \$110 meal? \ans 

How much is the final bill? \ans
\ee
\end{minipage}
\hfill
\begin{minipage}{.4\textwidth}
\includegraphics[width = 2in]{SpreadsheetPic2}
\end{minipage}

\item Simple Interest
\be

\item {\bf Definition:} Interest is only earned (or paid) on the original amount. (Imagine you take the interest each year and just put it in your wallet.)

\item {\bf Example:} You invest \$500 and you earn 6\% interest every year for 5 years (only on the original \$500).  

Calculating simple interest with a spreadsheet:


%\be
%\item How much interest do you earn in the first year? \ans
%\item How much interest do you earn in the second year? \ans
%\item How much interest do you earn in the third year? \ans
%\item How much interest do you earn in \emph{all five years}? \ans
%\item How much money do you have at the end of the five years? \ans
%\item Can you write that as a single formula? \ans[3in]
%\ee

%\item The formula to compute simple interest is 
%\[ \text{amount} = \text{principal} + \text{principal}*\text{rate}*\text{time} = \text{principal}*(1+\text{rate}*\text{time})\]
%
%where \verb`amount` = final amount and \verb`principal` = starting amount
\be
\item Make a new sheet in your spreadsheet, called \verb`Interest`.
\item In \verb`A1` type \verb`Simple Interest`
\item In \verb`A2` \verb`Year`. Type in 1 and 2 and fill down to get to year 5.
\item In \verb`B2` type \verb`principal` (principal = starting amount of money).
\item In \verb `C2` type  \verb`interest`
\item In \verb `C1` type \verb`0.06` (this is where we are strong our interest)
\item Type \verb`500` into \verb`B3`
\item Type \verb`=$B$3*$C$1` into \verb`C3`. The \verb`$` fix the row and column references.
\item Fill down \verb`C3` until year 5
\item In \verb`A8` type \verb`total`
\item In \verb`C8` type \verb`=sum(C3:C7)` (or type \verb`=sum(` and then click on cell \verb`C3` and drag down to \verb`C7`) 
\item In \verb`A9` type \verb`grand total` and then in \verb`B9` type \verb`=B3+C8`
\ee

\item How much interest was earned each year? \ans

\item How much interest was earned in total? \ans

\item How much money did you have at the end of 5 years? \ans

\item What happens if you change the interest rate? What if you change the principal? Experiment. (Then change back to principal = \$500 and interest = 6\%)

\ee
\newpage

\item Compound Interest
\be

\item {\bf Definition:} Interest is earned at a certain rate and then reinvested with the principal

\item {\bf Example:} You invest \$500 and you earn 6\% interest, compounded every year for 5 years.  

Calculating compound interest with a spreadsheet:

\be
\item Copy the simple interest calculation starting in column \verb`E1`. We will modify to compute compound interest:
\item In \verb`E1` type \verb`Compound Interest`
\item Type \verb`=$F$3*$G$1` into \verb`C3`. The \verb`$` fix the row and column references.
\item Type \verb`=F3+G3` into \verb`F4`.  What are we doing? \ans
\item Type \verb`=F4*$G$1` into \verb`G4`
\item Click on both cells \verb`F4` and \verb`G4` and fill them both down simultaneously.
\item In \verb`F8` type \verb`total`
\item In \verb`G8` type \verb`=sum(G3:G7)` (or click and drag) 
\item In \verb`E9` type \verb`grand total` and then in \verb`F9` type \verb`=F3+G8`
\ee

\item How much interest was earned in total? \ans

\item How much money did you have at the end of 5 years? \ans

\item Use a spreadsheet to calculate how much more interest was earned through compound interest vs simple interest. \ans

\ee


\end{enumerate}
\end{document}

%-------------------------------------------------------------------------------------------------------------------------------------------------------------------------------------------------------------------

%%% Local Variables:
%%% mode: latex
%%% TeX-master: t
%%% End:
