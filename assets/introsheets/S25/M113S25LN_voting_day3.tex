% !TEX TS-program = pdflatexmk
\documentclass[12pt]{article}

% Layout.
\usepackage[top=1in, bottom=0.75in, left=1in, right=1in, headheight=1in, headsep=6pt]{geometry}

% Fonts.
\usepackage{mathptmx}
\usepackage[scaled=0.86]{helvet}
\renewcommand{\emph}[1]{\textsf{\textbf{#1}}}

% Misc packages.
\usepackage{amsmath,amssymb,latexsym}
\usepackage{graphicx,hyperref}
\usepackage{array}
\usepackage{xcolor}
\usepackage{multicol}
\usepackage{tabularx,colortbl,booktabs,xparse}
\usepackage{enumitem}

% Rotation: \rot[<angle>][<width>]{<stuff>}
\NewDocumentCommand{\rot}{O{45} O{1em} m}{\makebox[#2][l]{\rotatebox{#1}{#3}}}%

\usepackage{fancyhdr}
\pagestyle{fancy} 
\lhead{\large\sf\textbf{MATH 113: Voting Theory}}
\chead{\large\sf\textbf{lecture notes}}
\rhead{\large\sf\textbf{Day 3}}

\begin{document}
\begin{enumerate}
\item Review of Voting Methods so far\\
	\begin{enumerate}
	\item Plurality Voting
	\vfill
	\item Instant Runoff Voting IRV or Ranked Choice Voting (RCV)
	\vfill
	\end{enumerate}
\item Borda Count
	\begin{enumerate}
	\item \textbf{description:}
	\vspace{1.5in}
	\item \textbf{example:}
	\begin{tabular}{l || c |c|c|c|c}
\# votes&3&4&2&1&1\\
\hline
1st choice&A&B&C&C&D\\
2nd choice&C&C&D&B&C\\
3rd choice&B&D&B&A&B\\
4th choice&D&A&A&D&A\\
\end{tabular}
	\vfill
	\end{enumerate}
\newpage
\item Copeland's Method
	\begin{enumerate}
	\item \textbf{description:}
	\vspace{1.5in}
	\item \textbf{example:}
	\begin{tabular}{l || c |c|c|c|c}
\# votes&3&4&2&1&1\\
\hline
1st choice&A&B&C&C&D\\
2nd choice&C&C&D&B&C\\
3rd choice&B&D&B&A&B\\
4th choice&D&A&A&D&A\\
\end{tabular}
	\vfill
	
	\end{enumerate}

\end{enumerate}
\end{document}