\documentclass[12pt]{article}

\usepackage[margin = .8in]{geometry}
\usepackage{amsmath}
\usepackage{graphicx}
\usepackage{multicol, enumerate, tabularx}

\usepackage{adjustbox, soul, setspace}

\usepackage{fancyhdr}
\pagestyle{fancy}

\lhead{Math F113X: Numbers and Society}
\rhead{Lecture Notes}

\usepackage{tikz}
\usetikzlibrary{calc,trees,positioning,arrows,fit,shapes,through, backgrounds}
\usetikzlibrary{patterns}

\usetikzlibrary{decorations.markings}
\usetikzlibrary{arrows}

\usepackage{pgfplots}

\usepackage{longtable}
\usepackage{tabularx}

\newcommand{\ds}{\displaystyle}
\newcommand{\ans}[1][1in]{\rule{#1}{.5pt}}

\newcommand{\points}[1]{(#1 points.)}		% Trying to be lazy.

\usepackage{array}
\newcolumntype{L}[1]{>{\raggedright\let\newline\\\arraybackslash\hspace{0pt}}m{#1}}
\newcolumntype{C}[1]{>{\centering\let\newline\\\arraybackslash\hspace{0pt}}m{#1}}
\newcolumntype{R}[1]{>{\raggedleft\let\newline\\\arraybackslash\hspace{0pt}}m{#1}}
\newcommand{\red}[1]{\textcolor{red}{#1}}

\newcommand{\be}{\begin{enumerate}}
\newcommand{\ee}{\end{enumerate}}

\begin{document}
\begin{center}
{\large  Finance Section 2: Simple Interest, Compound Interest, APR, Future Value, Effective Rate}
\end{center}
Goals: 
\begin{itemize}
\item How to use formulas for Simple Interest over Time and Annual Percentage Rate compounded at various frequencies.
\item Comparing how different interest rates and compounding frequencies compare to each other.
\item Understand the difference between APY and the Effective Annual Interest Rate (EAR).
\end{itemize}

\begin{enumerate}
\item  Simple One-Time Interest
	\be
	\item Suppose Liz borrows \$1000 and agrees to pay it back in a year with 5\% simple interest. 
		\be
		\item How much \textbf{interest} will she pay?
		\vfill
		\item How much will she owe in total at the end of the year?
		\vfill
		\ee
	\item Suppose a loan is obtained under the conditions of simple one-time interest. If $P$ represents the \textbf{principal} or \textbf{present value} and $r$ represents the interest rate in decimal form, write a formula for interest, $I$.
		\vfill
	\item Under the conditions above, write a formula for the end amount, $A$, also called the \textbf{future value} of load.
		\vfill
	\ee
\newpage
\item Simple Interest over Time
\be
	\item Suppose Liz gets a \$1000 load with 5\% simple interest assessed annually.  
		\be
		\item How much \textbf{interest} will she pay \textit{if she pays it back in 6 months}?
		\vfill
		\item  How much \textbf{in total} will she pay \textit{if she pays it back in one year and 3 months}?
		\vfill
		\ee
	\item Suppose a loan is obtained under the conditions of simple interest over time. If $P$ represents the \textbf{principal} or \textbf{present value}, $r$ represents the annual interest rate in decimal form, t represents time as measured in years, write a formula for interest, $I$.
		\vfill
	\item Under the conditions above, write a formula for the end amount, $A$, also called the \textbf{future value} of load.
		\vfill
		\ee
\newpage
\item Annual Percentage Rate (APR) compounded at various frequencies.
	\be
	\item Suppose a savings account advertises an annual percentage rate of $7\%$ compounded semiannually with a \$5000 minimum balance. Assuming a minimum balance, how much money would this account have (assuming no additional deposits and no withdrawals) after
		\be
		\item 6 months
		\vfill
		\item 1 year
		
		\vspace{2in}
		\vfill
		\item 10 years
		\vfill
		\item \textbf{For each of the time periods above}, determine the interest accumulated and, then, the percent of return. (This is called the Effective Annual Interest Rate or EAR.) What do you notice?
		\vfill
		
		\vfill
		\ee
\newpage
	\item Suppose a different account offers $6.9\%$ compounded daily with a \$5000 minimum balance. Again assuming a minimum deposit, no additional deposits and no withdrawals, how much would this account have after 10 years? What do you observe?
	\vfill
	\ee

\end{enumerate}


\end{document}
