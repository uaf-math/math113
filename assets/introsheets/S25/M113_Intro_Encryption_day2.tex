

\documentclass[12pt]{article}

\usepackage[margin = .8in]{geometry}
\usepackage{amsmath}
\usepackage{graphicx}
\usepackage{multicol, enumerate, tabularx}

\usepackage[parfill]{parskip}

\usepackage{adjustbox, soul}

\usepackage{fancyhdr}
\pagestyle{fancy}

\lhead{Math F113X: Numbers and Society}
%\rhead{Date: \hspace{1in}}

\usepackage{tikz}
\usetikzlibrary{calc,trees,positioning,arrows,fit,shapes,through, backgrounds}
\usetikzlibrary{patterns}

\usetikzlibrary{decorations.markings}
\usetikzlibrary{arrows}

\usepackage{pgfplots}

\usepackage{longtable}
\usepackage{tabularx}

\newcommand{\ds}{\displaystyle}
\newcommand{\ans}[1][1in]{\rule{#1}{.5pt}}

\newcommand{\points}[1]{(#1 points.)}		% Trying to be lazy.

\usepackage{array}
\newcolumntype{L}[1]{>{\raggedright\let\newline\\\arraybackslash\hspace{0pt}}m{#1}}
\newcolumntype{C}[1]{>{\centering\let\newline\\\arraybackslash\hspace{0pt}}m{#1}}
\newcolumntype{R}[1]{>{\raggedleft\let\newline\\\arraybackslash\hspace{0pt}}m{#1}}
\newcommand{\red}[1]{\textcolor{red}{#1}}

\newcommand{\be}{\begin{enumerate}}
\newcommand{\ee}{\end{enumerate}}

%\topmargin -1in
%\textheight 9.5in
%\oddsidemargin -0.3in
%\evensidemargin \oddsidemargin
%\pagestyle{empty}
%%\marginparwidth 0.5in
%\textwidth 7in
%\parindent 0in

%--------------------------------------------------------------------------------------------------------------------------------------------------------------------------
%						Document
%--------------------------------------------------------------------------------------------------------------------------------------------------------------------------


\begin{document}
%\pagestyle{fancy}
\begin{center}
{\Large  Intro Cryptography (Day 2)}
\end{center}
\begin{enumerate}
\item Review \emph{shift ciphers}: What is the key in the shift cipher below? KEY: \ans\\

\hspace*{-2.2cm}
{\scriptsize
\begin{tabular}{|c|c|c|c|c|c|c|c|c|c|c|c|c|c|c|c|c|c|c|c|c|c|c|c|c|c|c|}
\hline
&0&1&2&3&4&5&6&7&8&9&10&11&12&13&14&15&16&17&18&19&20&21&22&23&24&25\\ \hline \hline
in& A&B&C&D&E&F&G&H&I&J&K&L&M&N&O&P&Q&R&S&T&U&V&W&X&Y&Z\\  \hline
out&L &M&N&O&P&Q&R&S&T&U&V&W&X&Y&Z&A&B&C&D&E&F&G&H&I&J&K\\[12pt]
\hline
\end{tabular}
}
\item A shift cipher is a particular type of substitution cipher. Below is a substitution cipher that is \textbf{not} a shift cipher.\\


{\scriptsize
\begin{tabular}{|c|c|c|c|c|c|c|c|c|c|c|c|c|c|c|c|c|c|c|}
\hline
&0&1&2&3&4&5&6&7&8&9&10&11&12&13&14&15&16&17\\ \hline \hline
in& A&B&C&D&E&F&G&H&I&J&K&L&M&N&O&P&Q&R\\  \hline
out&E&F&G&L&8&A&R&Q&T&U&V&P&B&D&N&O&H&M\\
\hline \hline
&18&19&20&21&22&23&24&25&26&27&28&29&30&31&32&33&34&35\\ \hline \hline
in&S&T&U&V&W&X&Y&Z&0&1&2&3&4&5&6&7&8&9\\ \hline
out&I&1&4&7&J&6&5&K&9&3&2&0&C&X&S&Z&Y&W \\
[12pt]
\hline
\end{tabular}
}
	\begin{enumerate}
	\item Encrypt: MEET APRIL 3\\
	
	\item Decrypt: F4IE19Z09\\
	
	\item What key is needed to decrypt a message using this encryption scheme?\\
	
	\item Which substitution is, in general, harder to break, a shift cipher or one that is not a shift cipher?\\
	
	\item What strategies would you use to try to break a substitution cipher that is not a shift cipher?
	\vfill
	\end{enumerate}
\item Another encryption scheme is called a \textbf{transposition cipher}. \\

Encode \fbox{JEWEL FOUND} using a transposition cipher of rows of 4 letters.\\

\vfill	
\end{enumerate}
\end{document}

%-------------------------------------------------------------------------------------------------------------------------------------------------------------------------------------------------------------------

%%% Local Variables:
%%% mode: latex
%%% TeX-master: t
%%% End:
