\documentclass[12pt]{article}

\usepackage[margin = .8in]{geometry}
\usepackage{amsmath}
\usepackage{graphicx}
\usepackage{multicol, enumerate, tabularx}

\usepackage{adjustbox, soul, setspace}

\usepackage{fancyhdr}
\pagestyle{fancy}

\lhead{Math F113X: Numbers and Society}
\rhead{Lecture Notes}

\usepackage{tikz}
\usetikzlibrary{calc,trees,positioning,arrows,fit,shapes,through, backgrounds}
\usetikzlibrary{patterns}

\usetikzlibrary{decorations.markings}
\usetikzlibrary{arrows}

\usepackage{pgfplots}

\usepackage{longtable}
\usepackage{tabularx}

\newcommand{\ds}{\displaystyle}
\newcommand{\ans}[1][1in]{\rule{#1}{.5pt}}

\newcommand{\points}[1]{(#1 points.)}		% Trying to be lazy.

\usepackage{array}
\newcolumntype{L}[1]{>{\raggedright\let\newline\\\arraybackslash\hspace{0pt}}m{#1}}
\newcolumntype{C}[1]{>{\centering\let\newline\\\arraybackslash\hspace{0pt}}m{#1}}
\newcolumntype{R}[1]{>{\raggedleft\let\newline\\\arraybackslash\hspace{0pt}}m{#1}}
\newcommand{\red}[1]{\textcolor{red}{#1}}

\newcommand{\be}{\begin{enumerate}}
\newcommand{\ee}{\end{enumerate}}

\begin{document}
\begin{center}
{\large  Finance Section 4: Introduction to Loan Calculation}
\end{center}
Goal: Think about loan calculations intuitively.

\begin{enumerate}
\item  (Basic Example:) Suppose you take out a loan for \$10,000 with an annual interest rate of 5\% compounded annually and you are going to make payments annually.
	\begin{enumerate}
	\item What is the \textit{minimum} annual payment that ensures you will eventually pay of the loan? (Round your answer to the nearest dollar.)
	\vfill
	\item Suppose you decided to have annual payments of \$600, how much do you owe at the end of one year? (So one year of interest added and one year of payments subtracted.)
	\vfill
	\item How much do you owe at the end of the second year? The third year?
	\vfill
	\item Use a spreadsheet to determine how long it will take to pay off the loan. (Use a spreadsheet to do this! Be thoughtful about how you set this up so you can change the parameters!)
	\vfill
	\item How much did the loan cost you? How much was interest?
	\vfill
	\item What happens if you increase you annual payments to \$800?
	\vfill
	\end{enumerate}
\newpage
\item (Compounded Monthly Example:) Suppose you take out a loan for \$10,000 with an annual interest rate of 5\% compounded \textbf{monthly} and you are going to make payments monthly.
	\begin{enumerate}
	\item What is the \textit{minimum} annual payment that ensures you will eventually pay of the loan? (Round your answer to the nearest dollar.)
	\vfill
	\item If you make a \$200 payment each month, how many \textbf{years} will it take to pay off the loan and how much did you pay in total?
	\vfill
	\item If you make a \$299.71 payment each month, how many \textbf{years} will it take to pay off the loan?
	\vfill
	
	\end{enumerate}
NOTE: We will show you how to compute the payment value in 2c!
\end{enumerate}
\end{document}
