% !TEX TS-program = pdflatexmk
\documentclass[12pt]{article}

% Layout.
\usepackage[top=1in, bottom=0.75in, left=1in, right=1in, headheight=1in, headsep=6pt]{geometry}

% Fonts.
\usepackage{mathptmx}
\usepackage[scaled=0.86]{helvet}
\renewcommand{\emph}[1]{\textsf{\textbf{#1}}}
\newcommand{\ans}[1][1in]{\rule{#1}{.5pt}}

% Misc packages.
\usepackage{amsmath,amssymb,latexsym}
\usepackage{graphicx,hyperref}
\usepackage{array}
\usepackage{xcolor}
\usepackage{multicol,tikz}
\usepackage{tabularx,colortbl,booktabs,xparse}
\usepackage{enumitem}

% Rotation: \rot[<angle>][<width>]{<stuff>}
\NewDocumentCommand{\rot}{O{45} O{1em} m}{\makebox[#2][l]{\rotatebox{#1}{#3}}}%

\usepackage{fancyhdr}
\pagestyle{fancy} 
\lhead{\large\sf\textbf{MATH 113: Fair Division}}
\chead{\large\sf\textbf{lecture notes}}
\rhead{\large\sf\textbf{Day 3}}

\begin{document}
Goal: Lone-Divider, Introduce Sealed Bids
\begin{enumerate}
\item Abel, Barbie, and Chris are splitting a the cake worth \$36. Abel is the Divider who determines the three pieces. Barbie and Chris value the pieces according to the following table:

\begin{center}
\begin{tabular}{| c | c | c | c |}
\hline
& piece 1& piece 2 & piece 3\\ \hline \hline
Abel& \$12 & \$12 & \$12 \\ \hline
Barbie & \$18 & \$12 & \$6 \\ \hline
Chris & \$11 & \$18 & \$7\\ \hline
\end{tabular}
\end{center}
	\begin{enumerate}
	\item How much value is a fair share of the cake? \ans\\
	\vfill
	\item Which pieces represent a fair share  for Barbie? \ans[2in]\\
	\vfill
\item Which pieces represent a fair share  for Chris? \ans[2in]\\
\vfill
\item Is it possible to distribute the pieces of cake to the three people so that everyone gets a piece that is a fair share for them? If so, explain how to do so; if not, explain what happens next.
\vfill

\vfill
	\end{enumerate}
\item Method of Sealed Bids (pg 103 for full description)
	\begin{enumerate}
	\item[1)] Each player submits a sealed bid on each item. 
	\item[2)] Calculate \textbf{total value} and \textbf{fair share} for each player.
	\item[3)] Award each item to highest bidder.
	\item[4)] For each player, determine the difference between the value received and fair share. (owed to holding pile or received from holding pile)
	\item[5a)] Calculate the surplus and divide it equally.
	\item[5b)] Determine the final allocation.
	\end{enumerate}
\newpage
\item Example 1. Anand and Bert are dividing the property below. Their value of each item is in the table below.

\begin{tabular}{l|l|p{2in}|p{2in}|}
&items&Anand&Bert\\
\hline
1)&couch&\$300&\$200\\
&coffee maker&\$50&\$100\\
&framed artwork&\$50&\$150\\
\hline
2)&Total Value&&\\
&&&\\
&Fair Share&&\\
\hline
3)& Award \& Sum&&\\
&&&\\
&&&\\
\hline
4)& (Award)-(Fair Share)&&\\
&mark owed or &&\\
&received&&\\
\hline
5a)& total surplus &&\\
&&&\\
&per person&&\\
\hline
5b)&Final Allocation&&\\
&&&\\
&&&\\
\end{tabular}
\item Example 2. What happens if Anand takes a different strategy and bids high in order to ensure receiving the three items?

\begin{tabular}{l|l|p{2in}|p{2in}|}
&items&Anand&Bert\\
\hline
1)&couch&\$400&\$200\\
&coffee maker&\$200&\$100\\
&framed artwork&\$100&\$60\\
\hline
2)&Total Value&&\\
&&&\\
&Fair Share&&\\
\hline
3)& Award \& Sum&&\\
&&&\\
&&&\\
\hline
4)& (Award)-(Fair Share)&&\\
&mark owed or &&\\
&received&&\\
\hline
5a)& total surplus &&\\
&&&\\
&per person&&\\
\hline
5b)&Final Allocation&&\\
&&&\\
&&&\\
\end{tabular}


\end{enumerate}
\end{document}