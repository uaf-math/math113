% !TEX TS-program = pdflatexmk
\documentclass[12pt]{article}

% Layout.
\usepackage[top=1in, bottom=0.75in, left=1in, right=1in, headheight=1in, headsep=6pt]{geometry}

% Fonts.
\usepackage{mathptmx}
\usepackage[scaled=0.86]{helvet}
\renewcommand{\emph}[1]{\textsf{\textbf{#1}}}

% Misc packages.
\usepackage{amsmath,amssymb,latexsym}
\usepackage{graphicx,hyperref}
\usepackage{array}
\usepackage{xcolor}
\usepackage{multicol}
\usepackage{tabularx,colortbl,booktabs,xparse}
\usepackage{enumitem}

% Rotation: \rot[<angle>][<width>]{<stuff>}
\NewDocumentCommand{\rot}{O{45} O{1em} m}{\makebox[#2][l]{\rotatebox{#1}{#3}}}%

\usepackage{fancyhdr}
\pagestyle{fancy} 
\lhead{\large\sf\textbf{MATH 113: Voting Theory}}
\lhead{\large\sf\textbf{lecture notes}}
\rhead{\large\sf\textbf{Day 1}}

\begin{document}
\begin{enumerate}
\item \textbf{Majority vs Plurality, Preference Schedule}

Ten Alaskans are asked to vote on the ``best" of four Alaskan villages.\\

Voters: Bishop, Claman, Dunbar, Giessel, Hughes, Kawasaki, Myers, Olson, Tobin, Wilson\\
Villages: \textbf{A}dak, \textbf{B}ettles, \textbf{C}hevak, \textbf{D}iomede\\

	\begin{enumerate}
	\item Given the vote below. Who wins? Did they win a \textbf{majority?}\\
	\begin{tabular}{l || cccccccccc }
voter&\rot{Bishop}&\rot{Claman}&\rot{Dunbar}&\rot{Giessel}&\rot{Hughes}&\rot{Kawasaki}&\rot{Myers}&\rot{Olson}&\rot{Tobin}&\rot{Wilson}\\
\hline
vote&A&A&A&A&B&B&B&C&C&D\\
	\end{tabular}

\vspace{2in}
	\item Suppose, in a different world, they voters voted this way. Now what?\\
\begin{tabular}{l || cccccccccc }
voter&\rot{Bishop}&\rot{Claman}&\rot{Dunbar}&\rot{Giessel}&\rot{Hughes}&\rot{Kawasaki}&\rot{Myers}&\rot{Olson}&\rot{Tobin}&\rot{Wilson}\\
\hline
vote&A&A&A&B&B&B&C&C&C&D\\

\end{tabular}
\vfill
	\item One option is to \textbf{collect more information.} See the new vote tally.\\
	
	\begin{tabular}{l || cccccccccc }
voter&\rot{Bishop}&\rot{Claman}&\rot{Dunbar}&\rot{Giessel}&\rot{Hughes}&\rot{Kawasaki}&\rot{Myers}&\rot{Olson}&\rot{Tobin}&\rot{Wilson}\\
\hline
1st choice&A&A&A&B&B&B&C&C&C&D\\
2nd choice&C&C&C&C&C&C&D&D&B&C\\
3rd choice&B&B&B&D&D&D&B&B&A&B\\
4th choice&D&D&D&A&A&A&A&A&D&A\\
\end{tabular}
\vfill
\item Observe that the vote tally in part (c) can be usefully summarized as follows:\\

\begin{tabular}{l || c }
\# votes&\quad \hspace{4in} \quad\\
\hline
1st choice&\\
2nd choice&\\
3rd choice&\\
4th choice&\\
\end{tabular}

\end{enumerate}

\newpage

\item \textbf{Fairness Criteria}\\
\vfill

\item \textbf{Condorcet Criteria}\\
\vfill

\item Show that Chevak (C) is the Condorcet Winner in the vote tally summarized in part 1d.\\

\begin{tabular}{l || c |c|c|c|c}
\# votes&3&3&2&1&1\\
\hline
1st choice&A&B&C&C&D\\
2nd choice&C&C&D&B&C\\
3rd choice&B&D&B&A&B\\
4th choice&D&A&A&D&A\\
\end{tabular}

\vfill
\end{enumerate}

\end{document}