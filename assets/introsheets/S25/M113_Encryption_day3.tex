

\documentclass[12pt]{article}

\usepackage[margin = .8in]{geometry}
\usepackage{amsmath}
\usepackage{graphicx}
\usepackage{multicol, enumerate, tabularx}

\usepackage[parfill]{parskip}

\usepackage{adjustbox, soul}

\usepackage{fancyhdr}
\pagestyle{fancy}

\lhead{Math F113X: Numbers and Society}
%\rhead{Date: \hspace{1in}}

\usepackage{tikz}
\usetikzlibrary{calc,trees,positioning,arrows,fit,shapes,through, backgrounds}
\usetikzlibrary{patterns}

\usetikzlibrary{decorations.markings}
\usetikzlibrary{arrows}

\usepackage{pgfplots}

\usepackage{longtable}
\usepackage{tabularx}

\newcommand{\ds}{\displaystyle}
\newcommand{\ans}[1][1in]{\rule{#1}{.5pt}}

\newcommand{\points}[1]{(#1 points.)}		% Trying to be lazy.

\usepackage{array}
\newcolumntype{L}[1]{>{\raggedright\let\newline\\\arraybackslash\hspace{0pt}}m{#1}}
\newcolumntype{C}[1]{>{\centering\let\newline\\\arraybackslash\hspace{0pt}}m{#1}}
\newcolumntype{R}[1]{>{\raggedleft\let\newline\\\arraybackslash\hspace{0pt}}m{#1}}
\newcommand{\red}[1]{\textcolor{red}{#1}}

\newcommand{\be}{\begin{enumerate}}
\newcommand{\ee}{\end{enumerate}}

%\topmargin -1in
%\textheight 9.5in
%\oddsidemargin -0.3in
%\evensidemargin \oddsidemargin
%\pagestyle{empty}
%%\marginparwidth 0.5in
%\textwidth 7in
%\parindent 0in

%--------------------------------------------------------------------------------------------------------------------------------------------------------------------------
%						Document
%--------------------------------------------------------------------------------------------------------------------------------------------------------------------------


\begin{document}
%\pagestyle{fancy}
\begin{center}
{\Large  Cryptography (Day 3)}
\end{center}
\begin{enumerate}
\item {\bf Progressive Caesar Cipher}: Start with a Caesar cipher, and then shift one letter with each character you are encrypting.



\makeatletter
\newcommand{\Letter}[1]{\@Alph{#1}}
\makeatother

\def\r{.6}
\begin{tikzpicture}
\draw[step=\r] (0,0) grid (27*\r,6*\r);
%\foreach \i in {1,2,...,26}{\draw (0,0) -- (\i*\r, 1);}
\foreach \i in {1,2,...,26}{\path (\i*\r+1/2*\r, 6*\r-1/2*\r) node {\textbf{\Letter{\i}}};}%
\foreach \i in {17,18,...,21}{\path (1/2*\r, 21*\r-\i*\r+1/2*\r) node {\textbf{\Letter{\i}}};}%
\draw[line width = .75mm] (0,5*\r) -- (27*\r, 5*\r);
\draw[line width = .75mm] (\r,0) -- (\r, 6*\r);
\foreach \i in {1,2,...,26}
	\foreach \j in {17,18,...,21}{
		{\path let \n1 = {int(mod((\j-1)+(\i-1), 26)+1} in (\i*\r+1/2*\r, 21*\r-\j*\r + 1/2*\r) node {\Letter{\n1}};
		}}
\end{tikzpicture}

\be
\item Example: Begin with the shift cipher $A \to Q$, and encrypt the word \so{HELLO}

\vfill
\item Example: Suppose the ciphertext was encrypted with the above scheme. Decrypt the word \so{FRJMS}.

\vfill

\ee

\item {\bf Vigen\`ere Cipher} Choose a keyword, and use that keyword to determine a shift cipher for each letter. (Repeat the keyword over and over.)

\begin{tikzpicture}
\draw[step=\r] (0,0) grid (27*\r,4*\r);
%\foreach \i in {1,2,...,26}{\draw (0,0) -- (\i*\r, 1);}
\foreach \i in {1,2,...,26}{\path (\i*\r+1/2*\r, 4*\r-1/2*\r) node {\textbf{\Letter{\i}}};}%
\foreach \i/\k in {4/1,15/2,7/3}{
\path (1/2*\r, 3*\r-\k*\r+1/2*\r) node {\textbf{\Letter{\i}}};
}%
\draw[line width = .75mm] (0,3*\r) -- (27*\r, 3*\r);
\draw[line width = .75mm] (\r,0) -- (\r, 4*\r);
\foreach \i in {1,2,...,26}
	\foreach \j/\k in {4/1,15/2,7/3}{
		{\path let \n1 = {int(mod((\j-1)+(\i-1), 26)+1} in (\i*\r+1/2*\r, 3*\r-\k*\r + 1/2*\r) node {\Letter{\n1}};
		}}
\end{tikzpicture}

\be
\item Example: use the keyword {\tt DOG} and encrypt the phrase WHERESMYFOOD.

\vfill

\item Example: Suppose the keyword is {\tt DOG}. Decrypt the ciphertext \so{LZONSYQCC}. 



\vfill

\ee

\newpage
\item{\bf Double Transposition Cipher:} Use two keywords (often of different lengths). Do a transposition cipher with the first keyword to produce a first ciphertext. If there are spaces, ignore them. Then encrypt that ciphertext (as though it were plaintext) using the second keyword.

\begin{description}
\item[Encryption:]\
\be[i.]
\item Write the plaintext in rows.
\item Permute the columns using the keyword.
\item Read off the {\bf columns} to form the new ``plaintext'' 
\item Write it in rows in the second grid.
\item Permute the columns using the second keyword.
\item Read off the columns to form the final ciphertext.
\ee

\item[Decryption:]\
\be[i.]
\item Count the number of characters. Determine how many ``extra'' characters there will be for the second keyword. 
\item Write your second grid with the permuted keyword. 
\item Fill in the columns of the grid, making sure to put the extra characters in the appropriate columns. For example, if your second keyword is {\tt SHORT} and you have two ``extra'' characters, then columns S and H get one more letter than the other columns.
\item (Un)permute the columns
\item Read off the {\bf rows}.
\item Using the number of characters, determine the number of ``extra'' characters for the first keyword.
\item Fill in the columns of the grid, filling in the extra long columns using the sorted keyword. For example, if your first keyword is {\tt KEY} and you have two ``extra'' characters, then columns K and E get one more letter than the other columns.
\item Unpermute the columns.
\item Read off the {\bf rows}
\ee
\end{description}


\newpage

\be

\item Example: Encrypt the word TRANSPOSITION using the first key {\tt KEY} and the second key {\tt SHORT}.

\vfill


\item Decrypt the ciphertext \so{ETETS PIHXC R} assuming it was encrypted using the scheme above.

\vfill
\ee


\end{enumerate}
\end{document}

%-------------------------------------------------------------------------------------------------------------------------------------------------------------------------------------------------------------------

%%% Local Variables:
%%% mode: latex
%%% TeX-master: t
%%% End:
