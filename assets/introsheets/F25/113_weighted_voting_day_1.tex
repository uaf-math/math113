% !TEX TS-program = pdflatexmk
\documentclass[12pt]{article}

% Layout.
\usepackage[top=1in, bottom=0.75in, left=1in, right=1in, headheight=1in, headsep=6pt]{geometry}

% Fonts.
\usepackage{mathptmx}
\usepackage[scaled=0.86]{helvet}
\renewcommand{\emph}[1]{\textsf{\textbf{#1}}}

% Misc packages.
\usepackage{amsmath,amssymb,latexsym}
\usepackage{graphicx,hyperref}
\usepackage{array}
\usepackage{xcolor}
\usepackage{multicol}
\usepackage{tabularx,colortbl,booktabs,xparse}
\usepackage{enumitem}

% Rotation: \rot[<angle>][<width>]{<stuff>}
\NewDocumentCommand{\rot}{O{45} O{1em} m}{\makebox[#2][l]{\rotatebox{#1}{#3}}}%

\usepackage{fancyhdr}
\pagestyle{fancy} 
\lhead{\large\sf\textbf{MATH F113X: Weighted Voting}}
\chead{\large\sf\textbf{lecture notes}}
\rhead{\large\sf\textbf{Day 1}}

\begin{document}
\begin{enumerate}
\item Context for Weighted Voting\\

\vfill

\item Notation and Terminology for Weighted Voting\\

\begin{itemize}

\item players

\vfill

\item weight

\vfill

\item quota

\vfill

\item coalition

\vfill

\item winning coalition

\vfill


\item critical player
\vfill

\end{itemize}

\item Example: [25: 11, 11, 10, 8]
\vfill

\newpage

\item Reasonable Limits on the Quota\\
\vspace{1in}

\item A Look at Power
	\begin{enumerate}
	\item A Dictator \hspace{1cm} [25:  \rule{.5cm}{.5pt}\ , 10, 3, 2], total weight = 40
	\vfill
	\item Having Veto Power \hspace{1cm} [25: \rule{.5cm}{.5pt}\ , 11, 5, 3], total weight = 40
	\vfill
	\item A Dummy \hspace{1cm} [25: 20, 18, \rule{.5cm}{.5pt}\ , \rule{.5cm}{.5pt}\ ], total weight = 40
	\vfill
	\end{enumerate}
	
\end{enumerate}
\end{document}