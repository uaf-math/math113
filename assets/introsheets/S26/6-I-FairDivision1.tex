% !TEX TS-program = pdflatexmk
\documentclass[12pt]{article}

% Layout.
\usepackage[top=1in, bottom=0.75in, left=1in, right=1in, headheight=1in, headsep=6pt]{geometry}

% Fonts.
\usepackage{mathptmx}
\usepackage[scaled=0.86]{helvet}
\renewcommand{\emph}[1]{\textsf{\textbf{#1}}}

% Misc packages.
\usepackage{amsmath,amssymb,latexsym}
\usepackage{graphicx,hyperref}
\usepackage{array}
\usepackage{xcolor}
\usepackage{multicol}
\usepackage{tabularx,colortbl,booktabs,xparse}
\usepackage{enumitem}

% Rotation: \rot[<angle>][<width>]{<stuff>}
\NewDocumentCommand{\rot}{O{45} O{1em} m}{\makebox[#2][l]{\rotatebox{#1}{#3}}}%

\usepackage{fancyhdr}
\pagestyle{fancy} 
\lhead{\large\sf\textbf{MATH F113X: Fair Division}}
\chead{\large\sf\textbf{lecture notes}}
\rhead{\large\sf\textbf{Intro + Divider-Chooser}}

\begin{document}
%Goal: Understand what is meant by \textbf{a fair share} and how to calculate it.
\begin{enumerate}
\item The context of Fair Division
\vfill
\item The definition of \textbf{a fair share}:\\
\vfill
\item Divider-Chooser in a Nutshell\\
\vfill
\newpage
\hspace*{-.7in}
\begin{tabular}{rl}
\textbf{\fbox{{\large{Example}}}\hspace{.7in} What is being divided:}& 6 muffins \\
&2 apple-walnut (A)\\
&2 blueberry (B)\\
&2 cheese \& jalepeno (C)\\[6pt]
%&\\
\textbf{Cost:}& the package of 6 cost \$12\\[6pt]
%&\\
\textbf{Parties (who is involved):}& Yuri (Y), and Zariah (Z) \\[6pt]
%&\\
\textbf{Preferences:}& Y likes all the flavors equally.\\
& Z likes A twice as much as B or C\\
\end{tabular}

	\begin{enumerate}
	\item \textbf{Ignoring all preferences}, what is the value of a muffin? \underline{\hspace{1in}}\\
	
	\item \textbf{In a dollar amount}, what would be \textbf{the value of} a fair share in this case? \underline{\hspace{1in}}\\
	
	\item Fill out the table below indicating for each party (X,Y, or Z), the dollar amount \textbf{they} would assign to each muffin. The total value should always sum to \$12. (!!)\\
	
	\begin{tabular}{c || c | c| c| c | c| c| c}
	party&\quad A \quad  \quad&\quad A \quad  \quad&\quad B \quad\quad&\quad B \quad\quad&\quad C\quad\quad&\quad C\quad\quad&total value\\
	\hline\hline
	&&&&&& \\
	Y&&&&&&\\
	&&&&&& \\
	\hline
	&&&&&& \\
	Z&&&&&&\\
	&&&&&& \\
	\end{tabular}
\item Complete Divider-Chooser with  {Yuri} as the divider and Zariah as the chooser.\\
\vfill  
\item Complete Divider-Chooser with  {Zariah} as the divider and Yuri as the chooser.\\
\vfill
\end{enumerate} 
\end{enumerate}
\end{document}