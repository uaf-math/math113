\documentclass[12pt]{article}

\usepackage[margin = .8in]{geometry}
\usepackage{amsmath}
\usepackage{graphicx}
\usepackage{multicol, enumerate, tabularx}

\usepackage{adjustbox, soul}

\usepackage{fancyhdr}
\pagestyle{fancy}

\lhead{Math F113X: Numbers and Society}
\rhead{Date: \hspace{1in}}

\usepackage{tikz}
\usetikzlibrary{calc,trees,positioning,arrows,fit,shapes,through, backgrounds}
\usetikzlibrary{patterns}

\usetikzlibrary{decorations.markings}
\usetikzlibrary{arrows}

\usepackage{pgfplots}

\usepackage{longtable}
\usepackage{tabularx}

\newcommand{\ds}{\displaystyle}
\newcommand{\ans}[1][1in]{\rule{#1}{.5pt}}

\newcommand{\points}[1]{(#1 points.)}		% Trying to be lazy.

\usepackage{array}
\newcolumntype{L}[1]{>{\raggedright\let\newline\\\arraybackslash\hspace{0pt}}m{#1}}
\newcolumntype{C}[1]{>{\centering\let\newline\\\arraybackslash\hspace{0pt}}m{#1}}
\newcolumntype{R}[1]{>{\raggedleft\let\newline\\\arraybackslash\hspace{0pt}}m{#1}}
\newcommand{\red}[1]{\textcolor{red}{#1}}

\newcommand{\be}{\begin{enumerate}}
\newcommand{\ee}{\end{enumerate}}

%\topmargin -1in
%\textheight 9.5in
%\oddsidemargin -0.3in
%\evensidemargin \oddsidemargin
%\pagestyle{empty}
%%\marginparwidth 0.5in
%\textwidth 7in
%\parindent 0in

%--------------------------------------------------------------------------------------------------------------------------------------------------------------------------
%						Document
%--------------------------------------------------------------------------------------------------------------------------------------------------------------------------


\begin{document}
%\pagestyle{fancy}
\begin{center}
{\Large  Worksheet 17 (Cryptography 1): Shift  Ciphers}
\end{center}



\noindent \textbf{Group Names:} \hrulefill \\
%-------------------------------------------------------------------------------------------------------------
%						Assignment
%-----------------------------------------------------------------------------------------------------
\begin{description}
\item[encryption] the process of transforming information in a way that, ideally, only authorized parties can decode
\item[decryption] Undoing a transformed message using a set of instructions (for example, a key and an algorithm)
\item[plaintext] readable/legible version of a message (unencrypted or already decrypted/decoded)
\item[ciphertext] unreadable/illegible version of a message (already encrypted/encoded)
\item[key] variable value (word or number), often kept secret, altering the result of encryption/decryption
\end{description}

\begin{enumerate}

\item Julius Caesar communicated with his troops using an encryption scheme where each letter in the message was shifted three letters over. 


{\scriptsize
\hspace*{-1.5cm}
\begin{tabular}{|c|c|c|c|c|c|c|c|c|c|c|c|c|c|c|c|c|c|c|c|c|c|c|c|c|c|c|}
\hline
&0&1&2&3&4&5&6&7&8&9&10&11&12&13&14&15&16&17&18&19&20&21&22&23&24&25\\ \hline \hline
in& A&B&C&D&E&F&G&H&I&J&K&L&M&N&O&P&Q&R&S&T&U&V&W&X&Y&Z\\  \hline
out&D&E&F&G&H&I&J&K&L&M&N&O&P&Q&R&S&T&U&V&W&X&Y&Z&A&B&C\\
\hline
\end{tabular}
}


\be

\item Use this cipher to encrypt the plaintext THE QUICK BROWN FOX.

%WKH TXLFN EURZQ IRA

\vfill

\item Decrypt the ciphertext \so{EHZDUH WKH LGHV RI PDUFK}

%BEWARE THE IDES OF MARCH

\vfill

%\item Choose a short phrase to encrypt (3 - 4 words). Encrypt it using the Caesar cipher, and then see if a classmate can decrypt it. They will simultaneously give you a phrase to decrypt.
%
%\vfill


\ee

\item In general, a \emph{shift cipher} shifts the letters of the alphabet over a certain number of steps, say $n$. For the classical Caesar cipher, $n = 3$. (Sometimes all shift ciphers are called Caesar ciphers.)

\be
\item Fill in the table for how to encrypt and decrypt the alphabet using a shift of 13. (On the internet, this is called `ROT-13' and is sometimes used to obscure movie spoilers, etc.) 


\hspace*{-2.2cm}
{\scriptsize
\begin{tabular}{|c|c|c|c|c|c|c|c|c|c|c|c|c|c|c|c|c|c|c|c|c|c|c|c|c|c|c|}
\hline
&0&1&2&3&4&5&6&7&8&9&10&11&12&13&14&15&16&17&18&19&20&21&22&23&24&25\\ \hline \hline
in& A&B&C&D&E&F&G&H&I&J&K&L&M&N&O&P&Q&R&S&T&U&V&W&X&Y&Z\\  \hline
out& & & &&&&&&&&&&&&&&&&&&&&&&&\\[12pt]
\hline
\end{tabular}
}

\item Decrypt the text {\LARGE \so{GURJURRYFBAGUROHF}}

\vfill


\item How many different shift ciphers are there? \ans
\ee

\vfill

\newpage

\item Consider the ciphertext

{\linespread{1.5}
\selectfont
\Large
%\begin{quote}
\so{NV KYV GVFGCV FW KYV LEZKVU JKRKVJ ZE FIUVI KF WFID R DFIV GVIWVTK LEZFE }
%\end{quote}
\par
}

%\bigskip

\vspace{-.5cm}

Suppose you know that this was encrypted using a shift cipher, but you don't know what the shift step is.  By looking at the short words in the ciphertext, guess what the shift is and decrypt the message. 


\hspace*{-1.5cm}
{\scriptsize
\begin{tabular}{|c|c|c|c|c|c|c|c|c|c|c|c|c|c|c|c|c|c|c|c|c|c|c|c|c|c|c|}
\hline
&0&1&2&3&4&5&6&7&8&9&10&11&12&13&14&15&16&17&18&19&20&21&22&23&24&25\\ \hline \hline
in& A&B&C&D&E&F&G&H&I&J&K&L&M&N&O&P&Q&R&S&T&U&V&W&X&Y&Z\\  \hline
out& & & &&&&&&&&&&&&&&&&&&&&&&&\\[12pt]
\hline
\end{tabular}
}



\vspace{1cm}

\item If you don't have spaces to help you determine word lengths, you can use \emph{frequency analysis} to help guess which letters map to which other letters. 


Consider the following ciphertext. It has been encrypted using some sort of shift cipher.% Count the number of appearances of each letter in the ciphertext to construct a \emph{frequency table}.
%(The text is repeated twice so you can cross out letters on the right-hand side for counting purposes.)



The letters of the English language sorted by frequency (say, number of appearances in 40,000 words) are

\hspace*{-1cm}
{\footnotesize
\begin{tabular}{|c|c|c|c|c|c|c|c|c|c|c|c|c|c|c|c|c|c|c|c|c|c|c|c|c|c|}
\hline
%&0&1&2&3&4&5&6&7&8&9&10&11&12&13&14&15&16&17&18&19&20&21&22&23&24&25\\ \hline \hline
 E&T&A&I&O&N&S&R&H&D&L&U&C&M&F&Y&W&G&P&B&V&K&X&Q&J&Z\\  %\hline
% & & &&&&&&&&&&&&&&&&&&&&&&&\\[12pt]
\hline
\end{tabular}
}

Here are frequencies of the letters of the ciphertext (this is a \emph{frequency table}).

\hspace*{-1cm}
{\scriptsize
\begin{tabular}{|c|c|c|c|c|c|c|c|c|c|c|c|c|c|c|c|c|c|c|c|c|c|c|c|c|c|c|}
\hline
%&0&1&2&3&4&5&6&7&8&9&10&11&12&13&14&15&16&17&18&19&20&21&22&23&24&25\\ \hline \hline
letter& A&B&C&D&E&F&G&H&I&J&K&L&M&N&O&P&Q&R&S&T&U&V&W&X&Y&Z\\  \hline \hline
count&0&9&5&14&3&0&2&0&3&0&11&3&2&3&16&2&2&9&9&0&0&4&1&7&4&3\\
\hline
\end{tabular}
}

Use the frequency table to help you guess the shift. Then decrypt the message.

\hspace*{-1cm}
{\scriptsize
\begin{tabular}{|c|c|c|c|c|c|c|c|c|c|c|c|c|c|c|c|c|c|c|c|c|c|c|c|c|c|c|}
\hline
%&0&1&2&3&4&5&6&7&8&9&10&11&12&13&14&15&16&17&18&19&20&21&22&23&24&25\\ \hline \hline
in& A&B&C&D&E&F&G&H&I&J&K&L&M&N&O&P&Q&R&S&T&U&V&W&X&Y&Z\\  \hline
out& & & &&&&&&&&&&&&&&&&&&&&&&&\\[12pt]
\hline
\end{tabular}
}

%\bigskip

%\begin{tabular}{c | c}
%\begin{minipage}{.5\linewidth}


\vspace{-1.5cm}

{\Large
\linespread{1.5}
\selectfont
%\begin{quote}
\so{

DRKDDROIKBOOXNYGONLIDROSBMBO

KDYBGSDRMOBDKSXEXKVSOXKLVOBS

QRDCDRKDKWYXQDROCOKBOVSPOVS

LOBDIKXNDROZEBCESDYPRKZZSXOCC

}
%\end{quote}
}
%\end{minipage}&
%\begin{minipage}{.5\linewidth}
%\begin{quote}
%DRKDDROIKBOOXNYG
%
%ONLIDROSBMBOKDYB
%
%GSDRMOBDKSXEXKVS
%
%OXKLVOBSQRDCDRKD
%
%KWYXQDROCOKBOVSP
%
%OVSLOBDIKXNDROZE
%
%BCESDYPRKZZSXOCC
%\end{quote}
%\end{minipage}
%\end{tabular}

\bigskip




\end{enumerate}
\end{document}

%-------------------------------------------------------------------------------------------------------------------------------------------------------------------------------------------------------------------

%%% Local Variables:
%%% mode: latex
%%% TeX-master: t
%%% End:
