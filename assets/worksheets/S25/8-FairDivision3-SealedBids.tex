\documentclass[12pt]{article}

\usepackage[margin = .8in]{geometry}
\usepackage{amsmath}
\usepackage{graphicx}
\usepackage{multicol, enumerate}

\usepackage{fancyhdr}
\pagestyle{fancy}

\lhead{Math F113X: Numbers and Society}
\rhead{Date: \hspace{1in}}

\usepackage{tikz}
\usetikzlibrary{calc,trees,positioning,arrows,fit,shapes,calc}
\usetikzlibrary{patterns}
\usepackage{pgfplots}

\usepackage{longtable}
\usepackage{tabularx}

\newcommand{\ds}{\displaystyle}
\newcommand{\ans}[1][1in]{\rule{#1}{.5pt}}

\newcommand{\points}[1]{(#1 points.)}		% Trying to be lazy.

\usepackage{array}
\newcolumntype{L}[1]{>{\raggedright\let\newline\\\arraybackslash\hspace{0pt}}m{#1}}
\newcolumntype{C}[1]{>{\centering\let\newline\\\arraybackslash\hspace{0pt}}m{#1}}
\newcolumntype{R}[1]{>{\raggedleft\let\newline\\\arraybackslash\hspace{0pt}}m{#1}}
\newcommand{\red}[1]{\textcolor{red}{#1}}

\newcommand{\be}{\begin{enumerate}}
\newcommand{\ee}{\end{enumerate}}

%\topmargin -1in
%\textheight 9.5in
%\oddsidemargin -0.3in
%\evensidemargin \oddsidemargin
%\pagestyle{empty}
%%\marginparwidth 0.5in
%\textwidth 7in
%\parindent 0in

%--------------------------------------------------------------------------------------------------------------------------------------------------------------------------
%						Document
%--------------------------------------------------------------------------------------------------------------------------------------------------------------------------


\begin{document}
%\pagestyle{fancy}
\begin{center}
{\Large  Worksheet 8 (Fair Division 3): The Method of Sealed Bids}
\end{center}



\noindent \textbf{Group Names:} \hrulefill \\
%-------------------------------------------------------------------------------------------------------------
%						Assignment
%-----------------------------------------------------------------------------------------------------
\begin{enumerate}
\item Jamal, Maggie, and Kendra are dividing an estate consisting of a house, a cabin, and
a boat.  Their valuations (in thousands) are shown below. We will use the method of sealed bids to determine the final allocation.

Their sealed bids are shown below. To determine the total allocation, we will fill in the various rows of the table. Instructions are below.


\begin{tabularx}{\linewidth}{| p{1.5in} | X | X | X|}
\hline
& {\bf Jamal} & {\bf Maggie} & {\bf Kendra } \\ \hline \hline
House &\$250 & \$280 & \$300\\ \hline
Cabin & \$170 &\$200  & \$255 \\ \hline
Boat & \$50& \$40& \$45 \\ \hline
Total bid & & & \\[10pt] \hline
Fair Share & & & \\[10pt] \hline
%3& Winning Bid(s) / who & & & \\[12pt] \hline
 total value of won items  & & & \\[10pt] \hline
 owed to estate  & & & \\[10pt] \hline
 estate owes  & & & \\[10pt] \hline
 share of surplus  & & & \\[10pt] \hline
 Final allocation & & & \\[20pt]
\hline
\end{tabularx}

\begin{enumerate}
\item Fill in the total amount bid by each person.
\item Fill in the Fair Share for each person: compute {\bf $(\text{total bid})/\text{(\# of people)}$}.
\item Circle the winning bid for each item.
\item Enter the total value of all the ``won'' items.
\item For each person, compute {\bf$(\text{total value of won items}) - (\text{fair share}).$}
If the total value is more than the fair share, that amount is \emph{ owed to the estate}. If the total value is less than the fair share, then the estate owes you that much. 
\vfill
\item Compute {\bf $ (\text{total amount owed to estate}) - (\text{total amount estate owes}).$} This is the \emph{total surplus}. Divide that quantity by the number of people to determine each person's surplus.

\vfill
\item Determine the final  allocation. Make sure to list any items the person got, any money they are paid, and any money they pay in. 
\vfill

\end{enumerate}

\newpage
\item 
\be
\item Fill in following table.

\begin{tabularx}{\linewidth}{| p{1.5in} | X | X | X|}
\hline
& {\bf Jamal} & {\bf Maggie} & {\bf Kendra } \\ \hline \hline
House &\$200 & \$75 & \$10\\ \hline
Cabin & \$200 &\$75  & \$10 \\ \hline
Boat & \$200& \$75& \$10 \\ \hline
Total bid & & & \\[10pt] \hline
Fair Share & & & \\[10pt] \hline
%3& Winning Bid(s) / who & & & \\[12pt] \hline
 total value of won items  & & & \\[10pt] \hline
 owed to estate  & & & \\[10pt] \hline
 estate owes  & & & \\[10pt] \hline
 share of surplus  & & & \\[10pt] \hline
 Final allocation & & & \\[20pt]
\hline
\end{tabularx}

\item Describe what happens when one person bids really high for everything and one person bids really low for everything.

\vfill
\ee

\item 
\be
\item Fill in the following table.

\begin{tabularx}{\linewidth}{| p{1.5in} | X | X | X|}
\hline
& {\bf Jamal} & {\bf Maggie} & {\bf Kendra } \\ \hline \hline
House &\$0 & \$100 & \$200\\ \hline
Cabin & \$300 &\$100  & \$50 \\ \hline
Boat & \$0& \$100& \$50 \\ \hline
Total bid & & & \\[10pt] \hline
Fair Share & & & \\[10pt] \hline
%3& Winning Bid(s) / who & & & \\[12pt] \hline
 total value of won items  & & & \\[10pt] \hline
 owed to estate  & & & \\[10pt] \hline
 estate owes  & & & \\[10pt] \hline
 share of surplus  & & & \\[10pt] \hline
 Final allocation & & & \\[20pt]
\hline
\end{tabularx}



\item Describe what happens in this situation.

\vfill
\ee
\end{enumerate}

\end{document}

%-------------------------------------------------------------------------------------------------------------------------------------------------------------------------------------------------------------------

%%% Local Variables:
%%% mode: latex
%%% TeX-master: t
%%% End:
