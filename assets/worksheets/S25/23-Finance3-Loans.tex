\documentclass[12pt]{article}

\usepackage[margin = .8in]{geometry}
\usepackage{amsmath}
\usepackage{graphicx}
\usepackage{multicol, enumerate, tabularx}

\usepackage{adjustbox, soul, setspace}

\usepackage[parfill]{parskip}

\usepackage{fancyhdr}
\pagestyle{fancy}

\lhead{Math F113X: Numbers and Society}
\rhead{Date: \hspace{1in}}

\usepackage{tikz}
\usetikzlibrary{calc,trees,positioning,arrows,fit,shapes,through, backgrounds}
\usetikzlibrary{patterns}

\usetikzlibrary{decorations.markings}
\usetikzlibrary{arrows}

\usepackage{pgfplots}

\usepackage{longtable}
\usepackage{tabularx}

\newcommand{\ds}{\displaystyle}
\newcommand{\ans}[1][1in]{\rule{#1}{.5pt}}

\newcommand{\points}[1]{(#1 points.)}		% Trying to be lazy.

\usepackage{array}
\newcolumntype{L}[1]{>{\raggedright\let\newline\\\arraybackslash\hspace{0pt}}m{#1}}
\newcolumntype{C}[1]{>{\centering\let\newline\\\arraybackslash\hspace{0pt}}m{#1}}
\newcolumntype{R}[1]{>{\raggedleft\let\newline\\\arraybackslash\hspace{0pt}}m{#1}}
\newcommand{\red}[1]{\textcolor{red}{#1}}

\newcommand{\be}{\begin{enumerate}}
\newcommand{\ee}{\end{enumerate}}

%\topmargin -1in
%\textheight 9.5in
%\oddsidemargin -0.3in
%\evensidemargin \oddsidemargin
%\pagestyle{empty}
%%\marginparwidth 0.5in
%\textwidth 7in
%\parindent 0in

%--------------------------------------------------------------------------------------------------------------------------------------------------------------------------
%						Document
%--------------------------------------------------------------------------------------------------------------------------------------------------------------------------


\begin{document}
%\pagestyle{fancy}
\begin{center}
{\Large  Worksheet 23 (Finance 3): Credit Cards and Mortgages}
\end{center}



\noindent \textbf{Group names:} \hrulefill \\
%-------------------------------------------------------------------------------------------------------------
%						Assignment
%-----------------------------------------------------------------------------------------------------
%Definitions:
%
%\begin{tabular}{| l | l |}
%\hline
%$P$ = principal / starting amount & $r$ = annual interest rate (APR) \\ \hline
%$I$ = interest & $n$ = number of compounding periods per year\\ \hline
%$A$ = final amount & $t$ = number of years\\ \hline
%\end{tabular}
%


%
%
%\begin{tabular}{ c  c}
%\fbox{Simple Interest} & \fbox{Compound Interest}\\
%$A = P(1 + rt)$ & $A = P(1 + \frac{r}{n})^{nt}$
%\end{tabular}

Instructions: You must use spreadsheets for the problems on the first page. You are encouraged to use spreadsheets for the back page, but you can also use a calculator if you choose.

\be

\item An Alaska Airlines Bank of America Credit Card charges an APR of 16.99\% on purchases. The minimum monthly payment is \$25 or 1\% of the balance on the account, whichever is larger\footnote{The balance is usually compounded daily, even though the payments are monthly. We are pretending that the balance is compounded monthly for this problem.}.

Suppose you have \$1500 in purchases on the credit card and you put no more purchases on the card.

\be
\item What is the minimum monthly payment for this balance? \ans

\vfill

\item How much interest is charged in the first month? \ans

\vfill

\item If you pay the minimum monthly payment in month 1, how much is your balance in month 2? \ans

\vfill

\item Suppose you pay \$100/month to your credit card bill. 
 Use a spreadsheet to determine how long it will take you to pay off the balance.
 



\be
\item How many months did it take to pay off your balance? \ans 

How many years? \ans


\item How much money did you pay in total? \ans 

\item How much of your payment was interest? \ans

\vfill

\ee
\item Suppose you only paid \$25/month to the bill.

\be
\item How many months did it take to pay off your balance? \ans 

How many years? \ans


\item How much money did you pay in total? \ans 

\item How much of your payment was interest? \ans

\vfill

\ee
\ee

\newpage
\item The Chase Freedom Unlimited credit card has a variable APR of 18.99\% to 28.49\%, based on your creditworthiness and other factors.\footnote{Apparently, for the Chase Freedom Unlimited credit card, the minimum payment is the greater of \$40 or 1\%..}

\be
\item If you had a balance of \$1500, you were charged an APR of 28.49\%, and you made monthly payments of \$25, how much would you owe at the end of the first month? \ans

\item How much would you owe at the end of the first year? \ans

\item What does that say about how long it would take to pay off your balance, if you only paid \$25/month?

\vfill

\ee



\item The formulas for computing information about a loan are as follows:


\begin{tabular}{| l | l |}
\hline
$P$ = principal / starting amount & $r$ = annual interest rate (APR) \\ \hline
$I$ = interest & $n$ = number of compounding periods per year\\ \hline
$A$ = final amount & $t$ = length of the loan, in years\\ \hline
& $d$ = regular loan payment \\ \hline
\end{tabular}


\begin{center}
\begin{tabularx}{.8\linewidth}{|X | X|} 
\hline
Loan amount given payment & payment given loan amount\\ \hline
$\displaystyle P = \frac{d\left(1 - \left(1+\frac{r}{n}\right)^{(-nt)}\right)}{\left(\frac{r}{n}\right)}$ & $\displaystyle d = \frac{P\left(\frac{r}{n}\right)}{\left(1 - \left(1+\frac{r}{n}\right)^{(-nt)}\right)}$\\
\hline
\end{tabularx}
\end{center}

 Suppose you want to pay off the \$1500 Alaska Airlines credit card charge (APR of 16.99\%) in a certain amount of time, given monthly payments and compounding ($n = 12$). You can think of the \$1500 as a loan. What does your monthly payment need to be (find $d$) to pay off your credit card in:
\be
\item 3 years? \ans

\vfill

Check your answer by doing the previous fill-down computation. You should get that you owe 0 after \ans months.


\item 1 year? \ans

\vfill

Check your answer by doing the previous fill-down computation. You should get that you owe 0 after \ans months.

\item 6 months? \ans

\vfill

Check your answer by doing the previous fill-down computation. %You should get that you owe 0 after \ans months.

\ee

\newpage

\item Suppose you want to buy a house that costs \$200,000 (after your downpayment). Suppose you can get a 30-year fixed rate mortgage at a 6.940\% APR (this is a current rate).  

\be
\item What will your monthly payment be? (You know $P$, you want $d$; assume $n = 12$.)
\ans

\vfill

\item How would your previous answer change if mortgage rates went down to 4\% APR? \ans 
 
\vfill 

\item With that same 30-year fixed rate mortgage at a 6.940\% APR, if you can afford a monthly mortgage payment of \$800, how much money can you afford to spend on a house? (You know $d$, you want $P$.) \ans

\vfill 

\item How much of a mortgage can you take out if you can afford a monthly payment of \$1000? \ans

\vfill

\item If you had a mortgage on a \$200,000 house with a monthly payment of \$1323, and instead you paid \$1500 every month, how quickly would you be able to repay the mortgage?

\vfill 
\ee

\ee
\end{document}

%-------------------------------------------------------------------------------------------------------------------------------------------------------------------------------------------------------------------

%%% Local Variables:
%%% mode: latex
%%% TeX-master: t
%%% End:
