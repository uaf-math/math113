\documentclass[12pt]{article}

\usepackage[margin = .8in]{geometry}
\usepackage{amsmath}
\usepackage{graphicx}
\usepackage{multicol, enumerate, tabularx}

\usepackage{adjustbox, soul, setspace}

\usepackage{fancyhdr}
\pagestyle{fancy}

\lhead{Math F113X: Numbers and Society}
\rhead{Date: \hspace{1in}}

\usepackage{tikz}
\usetikzlibrary{calc,trees,positioning,arrows,fit,shapes,through, backgrounds}
\usetikzlibrary{patterns}

\usetikzlibrary{decorations.markings}
\usetikzlibrary{arrows}

\usepackage{pgfplots}

\usepackage{longtable}
\usepackage{tabularx}

\newcommand{\ds}{\displaystyle}
\newcommand{\ans}[1][1in]{\rule{#1}{.5pt}}

\newcommand{\points}[1]{(#1 points.)}		% Trying to be lazy.

\usepackage{array}
\newcolumntype{L}[1]{>{\raggedright\let\newline\\\arraybackslash\hspace{0pt}}m{#1}}
\newcolumntype{C}[1]{>{\centering\let\newline\\\arraybackslash\hspace{0pt}}m{#1}}
\newcolumntype{R}[1]{>{\raggedleft\let\newline\\\arraybackslash\hspace{0pt}}m{#1}}
\newcommand{\red}[1]{\textcolor{red}{#1}}

\newcommand{\be}{\begin{enumerate}}
\newcommand{\ee}{\end{enumerate}}

%\topmargin -1in
%\textheight 9.5in
%\oddsidemargin -0.3in
%\evensidemargin \oddsidemargin
%\pagestyle{empty}
%%\marginparwidth 0.5in
%\textwidth 7in
%\parindent 0in

%--------------------------------------------------------------------------------------------------------------------------------------------------------------------------
%						Document
%--------------------------------------------------------------------------------------------------------------------------------------------------------------------------


\begin{document}
%\pagestyle{fancy}
\begin{center}
{\Large  Worksheet 21 (Finance 1): Spreadsheets and Interest}
\end{center}



\noindent \textbf{Group names:} \hrulefill \\
%-------------------------------------------------------------------------------------------------------------
%						Assignment
%-----------------------------------------------------------------------------------------------------



\begin{enumerate}

\item Use a spreadsheet to calculate the following quantities:
\be
\item $471+5619$ = \ans
\item $36*987$= \ans
\item $45^{3}$= \ans (Use syntax \verb`=45^3`)
\item $12!$= \ans (Use syntax \verb`=fact(12)`)
\item Use a spreadsheet to calculate your total bill if the dinner cost \$72.50 and you want to give an 18\% tip. \ans

\ee

\singlespacing

\item Use a spreadsheet to calculate the following if you invested \$4500  for 10 years, assuming {\bf simple interest}.  (Use the process from the lecture notes.)

\doublespacing
\be
\item how much interest would be accrued if you invested 2\%? \ans

\item How much money did you have at the end of 10 years? \ans

\item How much interest would you accrue if you invested at 3\%? \ans

\item How much money did you have at the end of 10 years? \ans

\item How much additional interest did you get in going from 2\% to 3\%? \ans

\ee

\singlespacing
\item Answer the previous question if you are using {\bf compound interest} (compounded annually) instead.

\doublespacing

\be
\item how much interest would be accrued if you invested 2\%? \ans

\item How much money did you have at the end of 10 years? \ans

\item How much interest would you accrue if you invested at 3\%? \ans

\item How much money did you have at the end of 10 years? \ans

\item How much additional interest did you get in going from 2\% to 3\%? \ans

\ee

\newpage

\singlespacing
\item Now we will make a more clever interest calculator. 

\be[i.]
\item Make a new sheet in your spreadsheet, called \verb`Interest`.
\item In \verb`A1` type \verb`Simple Interest`
\item In row 2 make cells A through E contain the words 

\begin{tabular}{c c c c c}
year&principal&	interest&	total interest&	grand total\\
\end{tabular}
\item Start column \verb`A2` with 1 and then 2 and fill down until year 10.
\item In \verb `C1` type \verb`0.06` (this is where we are storing our interest)
\item Type \verb`500` into \verb`B3`
\item Type \verb`=B$3*C$1` into \verb`C3`. 
\item In \verb`D3` type \verb`=sum(C$3:C3)`. 
\item In \verb`E3` type \verb`=B$3+D3`.
\item Fill down \verb`C3, D3, E3` until year 10
\ee

 To convert the long decimals to things that look like currency, you can click on the column header and then click the button that looks like \verb`$` in the toolbar.

\doublespacing

\be
\item How much interest did you accrue in Year 10? \ans
\item How much total money did you accrue in Year 10? \ans
\item Fill down more columns if necessary. At what year will you have more than \$1200? \ans
\item At what year will you have accrued more than \$500 in interest? \ans
\ee



\singlespacing

\item Answer the previous questions assuming you invested \$825 at 2.75\% interest using simple interest:

\doublespacing

\be
\item What two cells do you need to update to make this change?

Cell  \ans should change to \ans

 Cell \ans should change to \ans


 

\item How much interest did you accrue in Year 10? \ans
\item How much total money did you accrue in Year 10? \ans
\item Fill down more columns if necessary. At what year will you have more than \$1200? \ans
\item At what year will you have accrued more than \$500 in interest? \ans

\ee


\newpage

\singlespacing
\item We will next make a more clever compound interest calculator by copying the simple interest calculator and updating some cells. 

\be[(i)]
\item Select your simple interest spreadsheet cells, copy them, click on cell \verb`G1` and hit paste.
\item Change the title to {\bf Compound Interest}
\item Change your cell values so  \verb`I1`  contains \verb`0.06` and \verb`H3`  contains 500.
\item Type \verb`=H3*I$1` into \verb`I3`. 
\item Type \verb`=H3+I3` into \verb`H4`.
\item Type \verb`=H4*I$1` into \verb`I4`.
\item Now select cells \verb`H4, I4` and fill them down. Columns J and K should update automatically.
\item If you like, select columns {\tt H,I,J,K} and turn them into currency by clicking the \$ in the toolbar.
\ee

Answer the following questions about compound interest, investing \$500 compounded annually at a rate of 6\%:



\be
\item How much interest did you accrue in Year 10? \ans
\item How much total money did you accrue in Year 10? \ans
\item How much more money did you accrue in Year 10 than from simple interest? \ans
\item Fill down more columns if necessary. At what year will you have more than \$1200? \ans
\item How did this answer differ from your simple interest answer? \ans
\item At what year will you have accrued more than \$500 in interest? \ans
\item How did this answer differ from your simple interest answer? \ans


\ee

\item Now answer the previous questions about compound interest assuming you invested \$825 at 2.75\% interest:
\be
\item What two cells do you need to update to make this change?

Cell  \ans\ should change to \ans

 Cell \ans\ should change to \ans


\item How much interest did you accrue in Year 10? \ans
\item How much total money did you accrue in Year 10? \ans
\item Fill down more columns if necessary. At what year will you have more than \$1200? \ans
\item At what year will you have accrued more than \$500 in interest? \ans

\ee
 
 \ee


\end{document}

%-------------------------------------------------------------------------------------------------------------------------------------------------------------------------------------------------------------------

%%% Local Variables:
%%% mode: latex
%%% TeX-master: t
%%% End:
