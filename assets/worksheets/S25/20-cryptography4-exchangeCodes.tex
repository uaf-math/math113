\documentclass[12pt]{article}

\usepackage[margin = .8in]{geometry}
\usepackage{amsmath}
\usepackage{graphicx}
\usepackage{multicol, enumerate, tabularx}

\usepackage{adjustbox, soul, setspace}

\usepackage{fancyhdr}
\pagestyle{fancy}

\lhead{Math F113X: Numbers and Society}
\rhead{Date: \hspace{1in}}

\usepackage{tikz}
\usetikzlibrary{calc,trees,positioning,arrows,fit,shapes,through, backgrounds}
\usetikzlibrary{patterns}

\usetikzlibrary{decorations.markings}
\usetikzlibrary{arrows}

\usepackage{pgfplots}

\usepackage{longtable}
\usepackage{tabularx}

\newcommand{\ds}{\displaystyle}
\newcommand{\ans}[1][1in]{\rule{#1}{.5pt}}

\newcommand{\points}[1]{(#1 points.)}		% Trying to be lazy.

\usepackage{array}
\newcolumntype{L}[1]{>{\raggedright\let\newline\\\arraybackslash\hspace{0pt}}m{#1}}
\newcolumntype{C}[1]{>{\centering\let\newline\\\arraybackslash\hspace{0pt}}m{#1}}
\newcolumntype{R}[1]{>{\raggedleft\let\newline\\\arraybackslash\hspace{0pt}}m{#1}}
\newcommand{\red}[1]{\textcolor{red}{#1}}

\newcommand{\be}{\begin{enumerate}}
\newcommand{\ee}{\end{enumerate}}

%\topmargin -1in
%\textheight 9.5in
%\oddsidemargin -0.3in
%\evensidemargin \oddsidemargin
%\pagestyle{empty}
%%\marginparwidth 0.5in
%\textwidth 7in
%\parindent 0in

%--------------------------------------------------------------------------------------------------------------------------------------------------------------------------
%						Document
%--------------------------------------------------------------------------------------------------------------------------------------------------------------------------


\begin{document}
%\pagestyle{fancy}
\begin{center}
{\Large  Worksheet 20 (Cryptography 4): What did they say?}
\end{center}



\noindent \textbf{Your name:} \hrulefill \\
%-------------------------------------------------------------------------------------------------------------
%						Assignment
%-----------------------------------------------------------------------------------------------------

\begin{center}
\fbox{Plaintext}
\end{center}

\begin{enumerate}

\item 
Choose one of the following three ciphers (circle): 

Progressive Caesar Cipher \hfill Vigeni\`ere Cipher \hfill Double Transposition Cipher

\item Choose an appropriate keyword 
\be
\item PCC: choose a starting shift: \hrulefill
\item Vigeni\`ere: Choose a 5-letter word \hrulefill
\item Double Transposition: Choose two words with different lengths, of at most 6 letters

\hrulefill
\ee

\item Choose a phrase of at least 20 characters

Phrase: \hrulefill

\item Encrypt your phrase using your method. Separate it into blocks of 5 characters.
\vfill

\item Then fill in the second sheet and exchange it with someone else in the class.
\ee
%\hrulefill

\newpage

\noindent \textbf{The encrypter's name:} \hrulefill \\

\noindent \textbf{The decrypter's name:} \hrulefill \\

\begin{center}
\fbox{Ciphertext}
\end{center}

Decrypt the following text using the given information:
\doublespacing



Cipher method: \ans[3in]

\bigskip

Secret Key: \ans[3in]

\bigskip

Ciphertext: \hrulefill

\hrulefill

\hrulefill

\vfill

\vfill


\end{document}

%-------------------------------------------------------------------------------------------------------------------------------------------------------------------------------------------------------------------

%%% Local Variables:
%%% mode: latex
%%% TeX-master: t
%%% End:
