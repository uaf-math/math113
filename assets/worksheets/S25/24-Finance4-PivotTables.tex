\documentclass[12pt]{article}

\usepackage[margin = .8in]{geometry}
\usepackage{amsmath}
\usepackage{graphicx}
\usepackage{multicol, enumerate, tabularx}

\usepackage{adjustbox, soul, setspace}

\usepackage[parfill]{parskip}

\usepackage{fancyhdr}
\pagestyle{fancy}

\lhead{Math F113X: Numbers and Society}
\rhead{Date: \hspace{1in}}

\usepackage{tikz}
\usetikzlibrary{calc,trees,positioning,arrows,fit,shapes,through, backgrounds}
\usetikzlibrary{patterns}

\usetikzlibrary{decorations.markings}
\usetikzlibrary{arrows}

\usepackage{pgfplots}

\usepackage{longtable}
\usepackage{tabularx}

\newcommand{\ds}{\displaystyle}
\newcommand{\ans}[1][1in]{\rule{#1}{.5pt}}

\newcommand{\points}[1]{(#1 points.)}		% Trying to be lazy.

\usepackage{array}
\newcolumntype{L}[1]{>{\raggedright\let\newline\\\arraybackslash\hspace{0pt}}m{#1}}
\newcolumntype{C}[1]{>{\centering\let\newline\\\arraybackslash\hspace{0pt}}m{#1}}
\newcolumntype{R}[1]{>{\raggedleft\let\newline\\\arraybackslash\hspace{0pt}}m{#1}}
\newcommand{\red}[1]{\textcolor{red}{#1}}

\newcommand{\be}{\begin{enumerate}}
\newcommand{\ee}{\end{enumerate}}

%\topmargin -1in
%\textheight 9.5in
%\oddsidemargin -0.3in
%\evensidemargin \oddsidemargin
%\pagestyle{empty}
%%\marginparwidth 0.5in
%\textwidth 7in
%\parindent 0in

%--------------------------------------------------------------------------------------------------------------------------------------------------------------------------
%						Document
%--------------------------------------------------------------------------------------------------------------------------------------------------------------------------


\begin{document}
%\pagestyle{fancy}
\begin{center}
{\Large  Worksheet 24: Pivot Tables}
\end{center}



\noindent \textbf{Group names:} \hrulefill \\
%-------------------------------------------------------------------------------------------------------------
%						Assignment
%-----------------------------------------------------------------------------------------------------
%


\be
\item Upload the data set {\tt PivotTableDataSets - JillLeahCourses.csv} into Google Sheets. This data set contains all the courses that your Math F113X instructors Jill Faudree and Leah Berman have taught since Spring 2015.

\item Select the cells, and create a Pivot Table.

\item Under {\tt Values}, add {\tt COURSE\_SECTION\_NUMBER}. What does this pivot table tell you?

\vfill

\item Under {\tt Columns}, add {\tt COURSE\_INSTRUCTOR}. What does this pivot table tell you? 

\vfill

Who taught the most sections over the time in the data set? \ans How many sections did she teach? \ans



\item Under {\tt Rows} add {\tt DISTINCT COURSE\_TERM\_CODE}. What does this pivot table tell you?

\vfill

\item Add {\tt COURSE\_TITLE} under rows. What does this pivot table tell you? How many sections were taught in Fall 2023, and of what?

\vfill

\item Delete {\tt COURSE\_SECTION\_NUMBER} from {\tt Values} and instead add {\tt COURSE\_ENROLLMENT}. What does this pivot table tell you?

\vfill

\item Modify the pivot table by deleting {\tt COURSE\_TITLE} from rows, and then answer the following questions:

\be
\item Which semester had the greatest enrollment? \ans

\item Which semester did Dr Faudree teach the most students? \ans

\item Which semester did Dr Berman teach the most students? \ans

\ee

\newpage

\item For the following questions, modify your pivot table (change what is in the rows and columns and values) to answer the following questions:

\be
\item Which courses have both Dr Berman and Dr Faudree taught? How many sections? (Hint: use {\tt COURSE\_SECTION\_NUMBER} in {\tt Values}).

\vfill

\item Change your pivot table as necessary. What changes did you make?

\vfill

How many total students\footnote{This counts the same student multiple times if they took multiple courses. That's fine.} did Dr Berman teach? \ans 

How many total students did Dr Faudree teach? \ans


\item Pivot tables can do different things with values. Make a pivot table that has {\tt COURSE\_INSTRUCTOR} followed by {\tt COURSE\_TITLE} in Rows, and {\tt DISTINCT COURSE\_TERM\_CODE} as {\tt Values}. Notice that you get some big, meaningless numbers. Change the ``Summarize By'' from {\tt SUM} to {\tt COUNTA}.

What is this pivot table now showing you?

\vfill

What is the course that Dr Berman has taught in the most semesters? \ans

What is the course that Dr Faudree has taught in the most semesters? \ans

\item Put {\tt DISTINCT COURSE\_TERM\_CODE} in Rows, {\tt COURSE\_INSTRUCTOR} in Columns, \\and {\tt COURSE\_COURSE\_CODE} in Values. Uncheck the boxes that say ``Show Totals.'' Select the Pivot Table, and then go to {\tt Insert > Chart} to make a chart. 

What do you see?

\vfill

\ee


\item Play around with putting different things into the rows, columns, and values in the pivot table. What is the most interesting way that you can summarize the data?

\vfill

\ee
\end{document}

%-------------------------------------------------------------------------------------------------------------------------------------------------------------------------------------------------------------------

%%% Local Variables:
%%% mode: latex
%%% TeX-master: t
%%% End:
