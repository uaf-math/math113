\documentclass[12pt]{article}

\usepackage[margin = .8in]{geometry}
\usepackage{amsmath}
\usepackage{graphicx}
\usepackage{multicol}

\usepackage{fancyhdr}
\pagestyle{fancy}

\lhead{Math F113X: Numbers and Society}
\rhead{January 22, 2025}

\usepackage{tikz}
\usetikzlibrary{calc,trees,positioning,arrows,fit,shapes,calc}
\usepackage{pgfplots}

\usepackage{longtable}
\usepackage{tabularx}

\newcommand{\ds}{\displaystyle}
\newcommand{\ans}[1][1in]{\rule{#1}{.5pt}}

\newcommand{\be}{\begin{enumerate}}
\newcommand{\ee}{\end{enumerate}}

\newcommand{\points}[1]{(#1 points.)}		% Trying to be lazy.

\usepackage{array}
\newcolumntype{L}[1]{>{\raggedright\let\newline\\\arraybackslash\hspace{0pt}}m{#1}}
\newcolumntype{C}[1]{>{\centering\let\newline\\\arraybackslash\hspace{0pt}}m{#1}}
\newcolumntype{R}[1]{>{\raggedleft\let\newline\\\arraybackslash\hspace{0pt}}m{#1}}
\newcommand{\red}[1]{\textcolor{red}{#1}}

%\topmargin -1in
%\textheight 9.5in
%\oddsidemargin -0.3in
%\evensidemargin \oddsidemargin
%\pagestyle{empty}
%%\marginparwidth 0.5in
%\textwidth 7in
%\parindent 0in

%--------------------------------------------------------------------------------------------------------------------------------------------------------------------------
%						Document
%--------------------------------------------------------------------------------------------------------------------------------------------------------------------------


\begin{document}
%\pagestyle{fancy}
\begin{center}
{\Large  Worksheet 4:  Weighted Voting Systems	 	}
\end{center}



\noindent \textbf{Group Names:} \hrulefill \\
%-------------------------------------------------------------------------------------------------------------
%						Assignment
%-----------------------------------------------------------------------------------------------------
\begin{enumerate}

\item Consider the weighted voting system [47: 10,9,9,5,4,4,3,2,2]
\be
\item How many players are there? \ans
\vspace{1cm}
\item What is the total number (weight) of votes? \ans
\vspace{1cm}
\item What is the quota in this system? \ans
\vspace{1cm}
\ee


\item Five friends decide to start a small side business in their
    free time. Wanting to create fairness in the decision-making proces,
    they decide to base their voting weight off the number of hours each
    one was working per week. Bill worked 15 hours, Tammy worked 8 hours,
    Dara worked 7 hours, Priyanka worked 3 hours, and Ross worked 2 hours. Any
    decision that their company makes requires a \emph{majority} of the votes.
    
 \be

  \item   Construct a weighted voting system to represent this scenario, and then determine if there are any dictators, dummies, or people with veto power. 
%Construct a weighted voting system to represent each of the following scenarios. Then, determine if there are any dictators, dummies, or people with veto power.

 

    \vfill
    
    \newpage
    
\item  In the previous scenario, construct a weighted voting system to represent this scenario,  if the decision required 2/3 of the votes. Then determine if there are any dictators, dummies, or people with veto power.
    
    \vfill
    
    \ee
%    \newpage

  \item A condo community is voting to approve a 10K loan to fix the
    structural and foundation issues with the building. If passed, the
    homeowners are responsible for paying off the loan over the next five
    years. This would result in an increase in the homeowners' monthly HOA
    fee until the loan is paid off. There are 2 homeowners and 2 board members. For this measure to pass both homeowners and one board member must vote yes. Determine a weighted  voting system to represent this scenario. Time permitting, decide if there are any dictators, dummies, or people with veto power.


    \vfill
    
    
\ee

\end{document}

%-------------------------------------------------------------------------------------------------------------------------------------------------------------------------------------------------------------------

%%% Local Variables:
%%% mode: latex
%%% TeX-master: t
%%% End:
