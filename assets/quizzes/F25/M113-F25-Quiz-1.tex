\documentclass[12pt]{article}

% Layout.
\usepackage[top=1in, bottom=0.75in, left=1in, right=1in, headheight=1in, headsep=6pt]{geometry}

% Fonts.
\usepackage{mathptmx}
\usepackage[scaled=0.86]{helvet}
\renewcommand{\emph}[1]{\textsf{\textbf{#1}}}

% TiKZ.
\usepackage{tikz, pgfplots}
\usetikzlibrary{calc}

% Misc packages.
\usepackage{amsmath,amssymb,latexsym}
\usepackage{graphicx}
\usepackage{array}
\usepackage{xcolor}
\usepackage{multicol}
\usepackage{fancyhdr}
\pagestyle{fancy}


% Misc.
\renewcommand{\d}{\displaystyle}
\newcommand{\ds}{\displaystyle}
\def\bc{\begin{center}}
\def\ec{\end{center}}
\def\be{\begin{enumerate}}
\def\ee{\end{enumerate}}

\newcommand{\ans}[1][1in]{\rule{#1}{.5pt}}


\lhead{Math F113X: Quiz 1}
\rhead{5 September, 2025}

\begin{document}

\vspace{1cm}

\noindent \textbf{Name:} \hrulefill \quad  score:\ans[1cm] \ / 10 \\


There are 10 points possible on this quiz. No aids (book, notes, etc.)
are permitted. You may use a non-programmable calculator.  {\bf Show all work and supporting calculations for full credit. Explain how you get your answers.}



\be
\item (6 points) The student government is holding elections for president. There are four candidates (A,B,C and D for convenience). The preference schedule is below.\\

\begin{tabular}{|c|| c|c|c|c|c|}
\hline
number of voters& 80&90&60&130&40\\
\hline \hline
1st choice&A&B&B&D&C\\ \hline
2nd choice&D&A&D&C&A\\ \hline
3rd choice&B&C&C&A&B\\ \hline
4th choice&C&D&A&B&D\\ \hline
\end{tabular}
\def\y{1.2}

For each of the following, provide supporting calculations.

\be
\item How many voters voted in this election? \ans
  \vfill
	
\item How many voters are needed for a majority? \ans
  \vfill
	
\item How many votes are needed for a plurality? \ans
  \vfill
  
\item Find a winner under the plurality method. Show some work. \ans
  \vfill
  \vfill

\item Did the winner under the plurality method also win a majority? \ans
  \vfill

\item Do you think the plurality winner in part (d) represents the
  will of the voters? Explain why or why not in a sentence.
  \vfill
	\ee
\newpage	
\item (4 points) Below is the same preference schedule. 

\begin{tabular}{|c|| c|c|c|c|c|}
\hline
number of voters& 80&90&60&130&40\\
\hline \hline
1st choice&A&B&B&D&C\\ \hline
2nd choice&D&A&D&C&A\\ \hline
3rd choice&B&C&C&A&B\\ \hline
4th choice&C&D&A&B&D\\ \hline
\end{tabular}

\be
\item In a one-to-one comparison, who is preferred, candidate A or
  candidate B? (You must show your calculation.)
	
  \vspace{1.5in}
	
  Candidate \ans\ is preferred.
\item Explain why candidate B cannot be the Condorcet winner.
  \vspace{1.5in}

\item Determine if there is a Condorcet winner. If so, who is
  it? Otherwise, explain why not. The results of each one-on-one
  comparison (excpet A vs B) are provided below.

  \begin{tabular}{|c||c|c|c|c|c|}
    \hline
    matchup & A vs C & A vs D & B vs C & B vs D & C vs D \\ \hline
    tally & A: 170 & A: 210 & B: 230 & B: 190 & C: 130 \\
            & C: 230 & D: 190 & C: 170 & D: 210 & D: 270 \\ \hline
    winner & C & A & B & D & D \\ \hline 
    \end{tabular}
    
	\vfill
	\ee	
\ee
\end{document}
%%% Local Variables:
%%% mode: LaTeX
%%% TeX-master: t
%%% End:
