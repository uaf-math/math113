\documentclass[12pt]{article}

% Layout.
\usepackage[top=1in, bottom=0.75in, left=1in, right=1in, headheight=1in, headsep=6pt]{geometry}

% Fonts.
\usepackage{mathptmx}
\usepackage[scaled=0.86]{helvet}
\renewcommand{\emph}[1]{\textsf{\textbf{#1}}}

% TiKZ.
\usepackage{tikz, pgfplots}
\usetikzlibrary{calc}

% Misc packages.
\usepackage{amsmath,amssymb,latexsym}
\usepackage{graphicx}
\usepackage{array}
\usepackage{xcolor}
\usepackage{multicol}
\usepackage{fancyhdr}
\pagestyle{fancy}


% Misc.
\renewcommand{\d}{\displaystyle}
\newcommand{\ds}{\displaystyle}
\newcommand{\ul}[1]{\underline{#1}}
\def\bc{\begin{center}}
\def\ec{\end{center}}
\def\be{\begin{enumerate}}
\def\ee{\end{enumerate}}

\newcommand{\ans}[1][1in]{\rule{#1}{.5pt}}


\lhead{Math F113X: Quiz 3}
\rhead{19 September, 2025}

\begin{document}

\vspace{1cm}

\noindent \textbf{Name:} \hrulefill \quad  score:\ans[1cm] \ / 10 \\


There are 10 points possible on this quiz. No aids (book, notes, etc.)
are permitted. You may use a non-programmable calculator.  {\bf Show all work and supporting calculations for full credit. Explain how you get your answers.}



\be

\item (3 points) Consider the weighted voting system $[25: 10, 8, 5,3,2,1]$
	\be
	\item Identify the dictators, if any. Explain your reasoning.
	\vfill
	\item Identify any players with veto power, if any. Explain your reasoning.
	\vfill
	\item Identify any dummies, if any. Explain your reasoning.
	\vfill
	\ee
\item (3 points) Consider the weighted voting system $[10:6,5,4,2,1]$,
	\be
	\item Does $\{P_3,P_4,P_5\}$ form a winning coalition? Explain.
	\vfill
	\item It is a fact that $\{P_1,P_2,P_4,P_5\}$ forms a winning coalition. \underline{Underline} the players that are \textbf{critical} to the coalition, and write/provide a computation that supports this.\\
	
	$$P_1,\:P_2,\:P_4,\:P_5$$
	\vfill
	\ee
	\newpage
\vfill
\item (4 points) 
 For the weighted voting system $[50:40,30,20,5]$, the winning coalitions are listed below. The critical players are underlined. 
 \be 
\item Using this information, determine the Banzhaf Power Distribution. Show your work.\\

\begin{tabular}{c}
winning coalitions\\
\hline \hline
\\
\ul{$P_1$} \ul{$P_2$}\\
\\
\ul{$P_1$} \ul{$P_3$}\\
\\
% \ul{$P_1$} \ul{$P_4$}\\
% \\
\ul{$P_2$} \ul{$P_3$}\\
\\
{$P_1$} {$P_2$} $P_3$\\
\\
\ul{$P_1$} {$P_2$} $P_4$\\
\\
\ul{$P_1$} {$P_3$} {$P_4$}\\
\\
\ul{$P_2$} \ul{$P_3$} {$P_4$}\\
\\
{$P_1$}{$P_2$} {$P_3$} {$P_4$}\\
\\
\end{tabular}
\vfill


\item Does this power distribution seem fair given the weighted voting system described above? Explain.

\vspace{1in}
\ee
%
%\textcolor{red}{I think we should leave problem 4 off and put it on the midterm. With it, I think the quiz is too long and these require a bit more thinking}
%\item (0 points) A law firm has 2 senior partners and 3 junior partners. For a motion to pass it must have at least 3 yes votes, one of which must be from a senior partner. Find a weighted voting system to represent this system. (Make sure to identify which weights go with the senior partners.)
%
%\vfill
 \ee
\end{document}
%%% Local Variables:
%%% mode: LaTeX
%%% TeX-master: t
%%% End:
