\documentclass[12pt]{article}

% Layout.
\usepackage[top=1in, bottom=0.75in, left=1in, right=1in, headheight=1in, headsep=6pt]{geometry}

% Fonts.
\usepackage{mathptmx}
\usepackage[scaled=0.86]{helvet}
\renewcommand{\emph}[1]{\textsf{\textbf{#1}}}

% TiKZ.
\usepackage{tikz, pgfplots}
\usetikzlibrary{calc}

% Misc packages.
\usepackage{amsmath,amssymb,latexsym}
\usepackage{graphicx}
\usepackage{array}
\usepackage{xcolor}
\usepackage{multicol}
\usepackage{fancyhdr}
\pagestyle{fancy}


% Misc.
\renewcommand{\d}{\displaystyle}
\newcommand{\ds}{\displaystyle}
\def\bc{\begin{center}}
\def\ec{\end{center}}
\def\be{\begin{enumerate}}
\def\ee{\end{enumerate}}

\newcommand{\ans}[1][1in]{\rule{#1}{.5pt}}


\lhead{Math F113X: Quiz 3}

\begin{document}

\ 

%\vspace{.25cm}

\noindent \textbf{Name:} \hrulefill \quad  score:\ans[1cm] \ / 10 \\


There are 10 points possible on this quiz. No aids (book, notes, etc.)
are permitted. You may use a non-programmable calculator.  {\bf Show
all work and supporting calculations for full credit. Explain how you
get your answers.}



\be

\item (4 pts. total -- 1 pt.~each) For the weighted voting system $[11; 7,5,2,2]$, 
\be
\item Why is $\{P_1,P_2,P_4\}$  a winning coalition?

\vskip .6in

\item  Is $P_2$ a critical member of $\{P_1,P_2,P_4\}$? Show how you decided.

\vskip .6in
\item  Are there any dictators for this voting system? If so, who? Explain your reasoning.

\vskip .6in

\item Which players, if any, have veto power? Explain  your reasoning.

\vskip .6in

\ee



\item  (3 pts.) For the weighted voting system $[14:9,6,6,2]$ there are 7 winning coalitions, listed here, with critical players underlined.


\begin{center}

\begin{minipage}{1.in}
$\{\underline{P_1},\underline{P_2}\}$\\
\ 

$\{\underline{P_1},\underline{P_3}\}$\\
\ 
\end{minipage}
\begin{minipage}{1.in}
$\{\underline{P_1},P_2,P_3\}$\\
\ 

$\{\underline{P_1},\underline{P_2},P_4\}$\\
\ 

\end{minipage}
\begin{minipage}{1.in}
$\{\underline{P_1},\underline{P_3},P_4\}$\\
\

$\{\underline{P_2},\underline{P_3},\underline{P_4}\}$ \\
\
\end{minipage}
\begin{minipage}{1.in}
$\{P_1,P_2,P_3,P_4\}$\\
\

\ \\
\ 

\end{minipage}
\end{center}


 Using this information, compute the Banzhaf Power Index of each player. You may leave your answers as fractions.

\ 


$P_1$:

\


$P_2$:

\ 


$P_3$:

\ 

$P_4$:

\ 

\item (3 pts.~total -- 1 pt.~each) A small business uses a weighted voting scheme $[q: 4,2,2,1]$ for its four owners, with weights in proportion to the amount each invested.
  
 \be
  \item One player suggests the quota be 2/3 of the total weight. What is the total weight, and what would be the quota?
  
  \ 
  
total weight $=$
  
  \ 
  
quota $=$
  
  \ 
  

  
  \item  $P_4$ responds, ``But that quota makes me a dummy!"  Explain what this means. (You do not need to show that statement is correct, although it is.)
  \vskip 1.2in
  
 \item  If instead the group sets the quota to be $5$, so the weighted system is $[5; 4,2,2,1]$, then $P_4$ is not a dummy. Give a coalition that shows this, and explain why. 
 \ee


\ee
\end{document}
%%% Local Variables:
%%% mode: LaTeX
%%% TeX-master: t
%%% End: