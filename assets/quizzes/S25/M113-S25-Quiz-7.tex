\documentclass[12pt]{article}

% Layout.
\usepackage[top=1in, bottom=0.75in, left=1in, right=1in, headheight=1in, headsep=6pt]{geometry}

% Fonts.
\usepackage{mathptmx}
\usepackage[scaled=0.86]{helvet}
\renewcommand{\emph}[1]{\textsf{\textbf{#1}}}

% TiKZ.
\usepackage{tikz, pgfplots,xcolor}

\usetikzlibrary{calc,trees,positioning,arrows,fit,shapes,through, backgrounds}
\usetikzlibrary{patterns}

\usetikzlibrary{decorations.markings}
\usetikzlibrary{arrows}

\usepackage{pgfplots}


% Misc packages.
\usepackage{amsmath,amssymb,latexsym}
\usepackage{graphicx}
\usepackage{array}
\usepackage{xcolor}
\usepackage{multicol}
\usepackage{fancyhdr}
\pagestyle{fancy}


% Misc.
\renewcommand{\d}{\displaystyle}
\newcommand{\ds}{\displaystyle}
\newcommand{\ul}[1]{\underline{#1}}
\def\bc{\begin{center}}
\def\ec{\end{center}}
\def\be{\begin{enumerate}}
\def\ee{\end{enumerate}}

\newcommand{\ans}[1][1in]{\rule{#1}{.5pt}}


\lhead{Math F113X: Quiz 7}
\rhead{April 11, 2025}

\begin{document}

\vspace{1cm}
\strut

\noindent \textbf{Name:} \hrulefill \quad  score:\ans[1cm] \ / 10 \\


\noindent There are 10 points possible on this quiz. No aids (book, notes, etc.)
are permitted. You may use a calculator.  {\bf Show all work and supporting calculations for full credit. Explain how you get your answers.}

A tabula recta is available.


\be

\item (2 points)  \textbf{Encrypt} the message IT'S FRIDAY using an alphabetic Caesar cipher with shift 5 (A to F).
%NYXKWNIFD

\vfill

\item (2 points)  \textbf{Decrypt} the message YFPJ F GWJFYM using an alphabetic Caesar cipher with shift 5 (A to F).
%TAKE A BREATH

\vfill

\item (2 points)  \textbf{Encrypt} the message APRIL 11 2025 using the substitution table below.\\

{\scriptsize
\begin{tabular}{|c|c|c|c|c|c|c|c|c|c|c|c|c|c|c|c|c|c|c|}
\hline
%&0&1&2&3&4&5&6&7&8&9&10&11&12&13&14&15&16&17\\ \hline \hline
original& A&B&C&D&E&F&G&H&I&J&K&L&M&N&O&P&Q&R\\  \hline
maps to&9&8&7&6&5&4&3&2&1&0&D&C&B&A&H&G&F&E\\
\hline
\end{tabular}

\begin{tabular}{|c|c|c|c|c|c|c|c|c|c|c|c|c|c|c|c|c|c|c|}
 \hline
%&18&19&20&21&22&23&24&25&26&27&28&29&30&31&32&33&34&35\\ \hline \hline
original &S&T&U&V&W&X&Y&Z&0&1&2&3&4&5&6&7&8&9\\ \hline
maps to&L&K&J&I&P&O&N&M&T&S&Q&R&V&U&Y&X&W&Z \\
%[12pt]
\hline
\end{tabular}
}

\vfill
\newpage
\item (2 points)  \textbf{Encrypt} the message WHICH ONE ARE YOU using a tabular transposition cipher with rows of length 6.

\vfill

\item (3 points)   \textbf{Decrypt} the message SUHKO GEOUH EXROC XDRHX OCAX using a tabular transposition cipher with rows of length 6.
%SOURDOUGH OR CHEECHAKO

\vfill
\item (1 point)  Describe at least one aspect of a Caesar cipher that makes it vulnerable to being hacked (that is, deciphered by an unwanted observer).
\vfill

\ee
\end{document}