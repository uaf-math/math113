\documentclass[12pt]{article}

% Layout.
\usepackage[top=1in, bottom=0.75in, left=1in, right=1in, headheight=1in, headsep=6pt]{geometry}

% Fonts.
\usepackage{mathptmx}
\usepackage[scaled=0.86]{helvet}
\renewcommand{\emph}[1]{\textsf{\textbf{#1}}}

% TiKZ.
\usepackage{tikz, pgfplots}
\usetikzlibrary{calc}

% Misc packages.
\usepackage{amsmath,amssymb,latexsym}
\usepackage{graphicx}
\usepackage{array}
\usepackage{xcolor}
\usepackage{multicol}
\usepackage{fancyhdr}
\pagestyle{fancy}


% Misc.
\renewcommand{\d}{\displaystyle}
\newcommand{\ds}{\displaystyle}
\def\bc{\begin{center}}
\def\ec{\end{center}}
\def\be{\begin{enumerate}}
\def\ee{\end{enumerate}}

\newcommand{\ans}[1][1in]{\rule{#1}{.5pt}}


\lhead{Math F113X: Quiz 1}
\rhead{Jan 24, 2025}

\begin{document}

\vspace{1cm}

\noindent \textbf{Name:} \hrulefill \quad  score:\ans[1cm] \ / 10 \\


There are 10 points possible on this quiz. No aids (book, notes, etc.)
are permitted. You may use a non-programmable calculator.  {\bf Show all work and supporting calculations for full credit. Explain how you get your answers.}



\be
\item (5 points) The student government is holding elections for president. There are four candidates (A,B,C and D for convenience). The preference schedule is below.\\

\begin{tabular}{|c|| c|c|c|c|c|}
\hline
number of voters& 70&60&50&110&10\\
\hline \hline
1st choice&C&B&C&D&A\\ \hline
2nd choice&D&A&A&B&C\\ \hline
3rd choice&B&D&B&C&B\\ \hline
4th choice&A&C&D&A&D\\ \hline
\end{tabular}
\def\y{1.2}

For each of the following, provide supporting calculations.

	\be
	\item How many voters voted in this election? \ans
	\vspace{\y cm}
	
	\item How many voters are needed for a majority? \ans
	
	\vspace{\y cm}
	
	\item How many votes are needed for a plurality? \ans
	
	\vspace{\y cm}
	
	\item Find a winner under the plurality method. \ans
	
	\vfill
	
	\vfill
	\ee
\newpage	
\item (5 points) Below is the same preference schedule. 

\begin{tabular}{|c|| c|c|c|c|c|}
\hline
number of voters& 70&60&50&110&10\\
\hline \hline
1st choice&C&B&C&D&A\\ \hline
2nd choice&D&A&A&B&C\\ \hline
3rd choice&B&D&B&C&B\\ \hline
4th choice&A&C&D&A&D\\ \hline
\end{tabular}

	\be
	\item In a one-to-one comparison, who is preferred, candidate A or candidate D? (You must show your calculation.)
	
	\vspace{1.5in}
	
	Candidate \ans\ is preferred.
	\item Explain why $A$ cannot be the Condorcet winner.
	\vspace{1.5in}

	\item Show that candidate $D$ is the Condorcet winner. (You must show your work.)
	\vfill
	\ee	
\ee
\end{document}