\documentclass[12pt]{article}

% Layout.
\usepackage[top=1in, bottom=0.75in, left=1in, right=1in, headheight=1in, headsep=6pt]{geometry}

% Fonts.
\usepackage{mathptmx}
\usepackage[scaled=0.86]{helvet}
\renewcommand{\emph}[1]{\textsf{\textbf{#1}}}

% TiKZ.
\usepackage{tikz, pgfplots}
\usetikzlibrary{calc}

% Misc packages.
\usepackage{amsmath,amssymb,latexsym}
\usepackage{graphicx}
\usepackage{array}
\usepackage{xcolor}
\usepackage{multicol}
\usepackage{fancyhdr}
\pagestyle{fancy}


% Misc.
\renewcommand{\d}{\displaystyle}
\newcommand{\ds}{\displaystyle}
\def\bc{\begin{center}}
\def\ec{\end{center}}
\def\be{\begin{enumerate}}
\def\ee{\end{enumerate}}

\newcommand{\ans}[1][1in]{\rule{#1}{.5pt}}


\lhead{Math F113X: Quiz 2}
\rhead{Jan 31, 2025}

\begin{document}

\vspace{1cm}
\strut

\noindent \textbf{Name:} \hrulefill \quad  score:\ans[1cm] \ / 10 \\


\noindent There are 10 points possible on this quiz. No aids (book, notes, etc.)
are permitted. You may use a calculator.  {\bf Show all work and supporting calculations for full credit. Explain how you get your answers.}



\be
\item (5 points) The student government is holding elections for president. There are four candidates (A,B,C and D for convenience). The preference schedule is below.  \\

\begin{tabular}{|c|| c|c|c|c|c|}
\hline
number of voters& 7&6&5&11&13\\
\hline \hline
1st choice&C&B&C&D&A\\ \hline
2nd choice&D&A&A&B&C\\ \hline
3rd choice&B&D&B&C&B\\ \hline
4th choice&A&C&D&A&D\\ \hline
\end{tabular}
\def\y{1.2}

	\be
	\item How many voters participated? \ans \\
	\item Does any candidate win a \emph{majority}? \emph{Justify} your answer.
	\vfill
	\item Find the winner under the Instant Runoff Voting / Ranked Choice Voting method. For each round, state the vote tally and identify the candidate eliminated.
	\vfill
	
	\vfill
	\ee
\newpage	
\item (5 points) A large family is voting on where to have the next family reunion. The preference schedule is below. 

\begin{tabular}{|c|| c|c|c|c|}
\hline
number of voters&30&10&10&5\\
\hline \hline
1st choice&C&B&D&D\\ \hline
2nd choice&D&D&A&B\\ \hline
3rd choice&B&A&B&A\\ \hline
4th choice&A&C&C&C\\ \hline
\end{tabular}

	\be
	\item  How many voters participated? \ans \\
	
	\item Find the winner under the Borda Count Method. (Show your calculations to earn full credit.)
	
	\vfill
	
	\vfill
	
	\item  Pick a fairness criterion that is violated in this case and provide a short explanation with supporting computation.\\
	
	Condorcet Criterion\\
	Monotonicity Criterion\\
	Majority Criterion\\
	Independence of Irrelevant Alternatives (IIA) Criterion\\
	
	\vfill
	
	
	\ee	
\ee
\end{document}