\documentclass[12pt]{article}
\usepackage[top=1in, bottom=1in, left=.75in, right=.75in]{geometry}
\usepackage{amsmath, enumerate}
\usepackage{fancyhdr}
\usepackage{graphicx, xcolor, setspace, array}
\usepackage{txfonts}
\usepackage{multicol,coordsys,pgfplots}
\usepackage[scaled=0.86]{helvet}
\renewcommand{\emph}[1]{\textsf{\textbf{#1}}}
\usepackage{anyfontsize}
% \usepackage{times}
% \usepackage[lf]{MinionPro}
\usepackage{tikz,pgfplots}
%\def\degC{{}^\circ{\rm C}}
\def\ra{\rightarrow}
\usetikzlibrary{calc,arrows.meta}
\pgfplotsset{compat = newest}
\newcommand{\blank}[1]{\rule{#1}{0.75pt}}

\pgfplotsset{my style/.append style={axis x line=middle, axis y line=
middle, xlabel={$x$}, ylabel={$y$}}}

%axis equal

%yticklabels={,,} , xticklabels={,,}

% \setmainfont{Times}
% \def\sansfont{Lucida Grande Bold}
\parindent 0pt
\parskip 4pt
\pagestyle{fancy}
\fancyfoot[C]{\emph{\thepage}}
\fancyfoot[R]{v3} %%%%%% <-- Version Info
\fancyhead[L]{\ifnum \value{page} > 1\relax\emph{Math F113X: Exam 1}\fi}
\fancyhead[R]{\ifnum \value{page} > 1\relax\emph{Spring 2025}\fi}
\headheight 15pt
\renewcommand{\headrulewidth}{0pt}
\renewcommand{\footrulewidth}{0pt}
\let\ds\displaystyle
\def\continued{{\emph {Continued....}}}
\def\continuing{{\emph {Problem \arabic{probcount} continued....}}\par\vskip 4pt}


\newcounter{probcount}
\newcounter{subprobcount}
\newcommand{\thesubproblem}{\emph{\alph{subprobcount}.}}
\def\problem#1{\setcounter{subprobcount}{0}%
\addtocounter{probcount}{1}{\emph{\arabic{probcount}.\hskip 1em(#1)}}\par}
\def\subproblem#1{\par\hangindent=1em\hangafter=0{%
\addtocounter{subprobcount}{1}\thesubproblem\emph{#1}\hskip 1em}}
\def\probskip{\vskip 10pt}
\def\medprobskip{\vskip 2in}
\def\subprobskip{\vskip 45pt}
\def\bigprobskip{\vskip 4in}


\newenvironment{subproblems}{%
\begin{enumerate}%
\setcounter{enumi}{\value{subprobcount}}%
\renewcommand{\theenumi}{\emph{\alph{enumi}}}}%
{\setcounter{subprobcount}{\value{enumi}}\end{enumerate}}


\newcommand{\be}{\begin{enumerate}}
\newcommand{\ee}{\end{enumerate}}


\newcommand{\ans}[1][1in]{\rule{#1}{.5pt}}



\begin{document}

{\emph{\fontsize{26}{28}\selectfont Spring 2025 \hfill
%{\fontsize{32}{36}\selectfont Calculus 1: Midterm 1}
\hfill Math F113X}}

\begin{center}
{\emph{%\fontsize{26}{28}\selectfont Spring 2024 
%%\hfill
{\fontsize{32}{36}\selectfont Exam 1}
%%\hfill Math F251X}
}}
\end{center}

%\vskip 2cm
\strut\vtop{\halign{\emph#\hskip 0.5em\hfil&#\hbox to 2in{\hrulefill}\cr
\emph{\fontsize{18}{22}\selectfont Name:}&\cr
%\noalign{\vskip 10pt}
%\emph{\fontsize{18}{22}\selectfont Student Id:}&\cr
%\noalign{\vskip 10pt}
%\emph{\fontsize{18}{22}\selectfont Calculator Model:}&\cr
}}
\hfill
\vtop{\halign{\emph{\fontsize{18}{22}\selectfont #}\hfil& \emph{\fontsize{18}{22}\selectfont\hskip 0.5ex $\square$ #}\hfil\cr
Section: & 10:30 am (Leah Berman)\cr
\noalign{\vskip 4pt}
         & 2:15pm (Jill Faudree)\cr
}}

\vfill
{\fontsize{18}{22}\selectfont\emph{Rules:}}

\begin{itemize}
\item Partial credit will be awarded, but you must show your work.

\item You may have 1/2 of a standard page of paper ($8.5''\times 5.5''$ or $11''\times 4.25''$)  of notes, both sides.

\item Calculators are allowed. 

\item Place a box around your  \fbox{FINAL ANSWER} to each question where appropriate.

\item Turn off anything that might go beep during the exam.

\end{itemize}

%If you need extra space, you can use the back sides of the pages.
%Please make it obvious  when you have done so.



Good luck!

\vfill
\def\emptybox{\hbox to 2em{\vrule height 16pt depth 8pt width 0pt\hfil}}
\def\tline{\noalign{\hrule}}
\centerline{\vbox{\offinterlineskip
{
\bf\sf\fontsize{18pt}{22pt}\selectfont
\hrule
\halign{
\vrule#&\strut\quad\hfil#\hfil\quad&\vrule#&\quad\hfil#\hfil\quad
&\vrule#&\quad\hfil#\hfil\quad&\vrule#\cr
height 3pt&\omit&&\omit&&\omit&\cr
&Problem&&Possible&&Score&\cr\tline
height 3pt&\omit&&\omit&&\omit&\cr
&1&&20 &&\emptybox&\cr\tline
&2&&12	&&\emptybox&\cr\tline
&3&&20	&&\emptybox&\cr\tline
&4&&12	&&\emptybox&\cr\tline
&5&&18	&&\emptybox&\cr\tline
&6&&18	&&\emptybox&\cr\tline
%&7&&	&&\emptybox&\cr\tline
%&8&&	&&\emptybox&\cr\tline
%&9&&	&&\emptybox&\cr\tline \tline
&Extra Credit&&(6)&&\emptybox&\cr\tline
&Total&&100&&\emptybox&\cr
}\hrule}}}

\newpage

%%Problem 1

\problem{20 points} 

A certain borough in Alaska has switched to using \emph{Instant Runoff Voting (Ranked Choice Voting) }to determine the winner of its mayoral races.

In a recent municipal election, the preference schedule for the race was as follows:

\begin{center}

\begin{tabular}{|l || c | c | c | c | c| c|}
\hline
& 47& 24& 15& 10& 14&10\\ \hline
%&Hopkins& Kassel & Sampson & Ward \\
1st choice & Sampson & Hopkins & Kassel & Sampson & Kassel & Ward\\
2nd choice & Hopkins & Kassel & Ward & Hopkins & Sampson& Kassel\\
3rd choice & Ward & Sampson & Sampson & Kassel & Ward & Sampson\\
4th choice & Kassel & Ward & Hopkins & Ward & Hopkins & Hopkins\\
\hline
\end{tabular}

\end{center}

\begin{subproblems}
\item How many voters voted in the election?  \ans %\ Provide your computation.

\vspace{.25in}

\item How many voters are needed to have a majority of the votes? \ans

\vspace{.25in}

\item Was there a winner after round 1 (that is, before anyone was eliminated)? Why or why not? Explain your answer.

\vspace{.5in}

\item Was anyone eliminated in round 1?  Explain your answer.

\vspace{.5in}

\item Determine the winner of the election. Show your work clearly, in a way that someone else can follow. If you require multiple rounds, show the computations clearly, and clearly state which candidate is eliminated.

\vfill


The winner of the election was \ans\ after \ans\ rounds.
\end{subproblems}
\newpage

%%%%%%%%%%%%%%%%%

%%Problem 2
%%%%%%%%%%%%%%%%%%%%%%%%%%%%%%


\problem{12 points} ASUAF is holding elections to decide who will represent UAF on the Board of Regents. There are four candidates (labeled A, B, C, and D for convenience). The preference schedule is in the table below.

\begin{tabular}{|c||c|c|c|c|c|c|}
\hline
number of voters&8&15&20&10&9&16\\
\hline \hline
1st choice&A&B&C&D&A&B\\
\hline
2nd choice&C&D&A&C&C&D\\
\hline
3rd choice&D&C&D&A&B&A\\
\hline
4th choice&B&A&B&B&D&C\\
\hline
\end{tabular}

	\begin{subproblems}
	\item Find the winner under the plurality method. Show the calculations that give your answer.
	\vfill
	\item Determine who would win if the only candidates were A and B. (That is, determine the winner in a head-to-head comparison of A and B.)  Show the calculations that give your answer.
	\vfill
	\item Based only on your calculation in part (b), \textbf{is it possible} for A to be the Condorcet Winner? Justify your answer.
	\vfill
	\item Determine the point value \emph{candidate C} would receive if the election were held using the Borda Count Method.
	\vfill
	\end{subproblems}
\newpage
%Problem 3
%Jill's weighted voting problem 2
\problem{20 points} Consider the weighted voting system $[q: 10,10,5,5,5,5,1]$
	\begin{enumerate}
	\item What is the smallest value $q$ can take? Justify your answer with a calculation.
	\vfill
	\item Explain why there is no choice of $q$ for which this voting system can have a dictator.
	\vfill
	\item Suppose $q$ is 36. So, the weighted voting system is $[36: 10,10,5,5,5,5,1]$.
		\begin{enumerate}
		\item Identify any players with veto power or state that none exist. Justify your answer.\\
		\vfill
		\item Identify any dummies or state that none exist. Justify your answer.\\
		\vfill
		\end{enumerate}
	\end{enumerate}
\newpage
%%%%%%%%%%%%%%%%%%%%%
\problem{12 points} Consider the weighted voting system $[17:13,9,5,2]$
	\begin{subproblems}
%	\item Determine all winning coalitions with exactly two players. \textcolor{red}{ Explain why both players must be critical in each of these coalitions.}
	
%	\vspace{2in}
 
 \item Determine all winning coalitions with one or two players. List of them in the space below (b).
 
\item  All winning coalitions using 3 or 4 players are listed below. Underline the players that are critical in each coalition (both the 2 player and the 3 or more players coalitions). %Use this along with your answer from part (a) to 
Then find the Banzhaf power distribution for this system.
	
	\bigskip
	{
	\begin{tabular}{p{2in} p{2in}}
	winning coalitions with & winning coalitions with\\
	1 or 2 players & 	 	3 or more players\\
	\hline
&	${P_1},{P_2},P_3$\\[12pt]
&	${P_1},{P_2},P_4$\\[12pt]
&	${P_1},{P_3},{P_4}$\\[12pt]
&	${P_1},P_2,{P_3},{P_4}$\\
	\end{tabular}
	}
	
\item \textbf{Based on your calculations in part (b)}, does this system contain any dummy players? Justify your conclusion.
\vspace{1in}
	\end{subproblems}
	\vfill

\newpage
%%%%%%%%%%%%%%%%%%%

\problem{18 points}

\begin{minipage}{.8\linewidth}Amanda and Bernadette pooled their money to buy a fancy box of handmade valentine's day heart-shaped truffles. The candy company makes boxes of truffles that contain three flavors of filling: caramel, raspberry, and hazelnut. Each box of candy contains 12 truffles, 3 each of caramel and raspberry and 6 of hazelnut. (See box right.) The box of truffles costs \$24.
\end{minipage}
\hspace{.2cm}
\begin{minipage}{.4\linewidth}
\begin{tabular}{|c c c c c c|}
\hline
c & c & c & h & h & h \\
r & r & r & h & h & h \\
\hline
\end{tabular}
\end{minipage}



\begin{subproblems}

\item What is the dollar value of a fair share? \ans 


\item Amanda will not eat caramel, and she likes raspberry twice as much as hazelnut. %Describe one possible fair share for her. Provide some words or a computation to support your answer.
\be[(i)]
\item What is the value of \emph{each} truffle to her?
\vfill
single caramel: \ans \ single raspberry: \ans \ single hazelnut: \ans \
\item Which of the following collections of truffles are a fair share for Amanda (if any)? Circle the answer, and write in the total value for Amanda.

\bigskip

{
\begin{tabular}{ c|| c | c | c }
&\begin{tabular}{| c c c c|}
\hline
c & c & c & h\\
\hline
\end{tabular}
&
\begin{tabular}{| c c c c|}
\hline
c & c &  c &h \\ 
r  & r & h & h\\
\hline
\end{tabular}
&
\begin{tabular}{| c c c  |}
\hline
r&r&r \\
\hline
\end{tabular}
\\ [12 pt] \hline
Value& & & 
\\[12 pt] \hline
Fair? & yes $\quad$ no & yes $\quad$ no & yes $\quad$ no

\end{tabular}
}

\vfill
\ee



\item Bernadette values a single caramel truffle at \$1, a single raspberry truffle at \$5, and a single hazelnut truffle at \$1.

\be[(i)]
\item If Bernadette is the divider, show a division of the box of candy that Bernadette might make, and explain why.  

\bigskip

 \begin{tabular}{|c c c c c c|}
\hline
c & c & c & h & h & h \\
r & r & r & h & h & h \\
\hline
\end{tabular} 

\bigskip

\item Which portion of chocolates would Amanda choose? Why? What is the total value to her of that portion?

\vfill
\ee

\end{subproblems}

\newpage

%%%%%%%%%%%%%%%%%%%%%%%%%%%%%%


%
%%%%%%%%%%%%%%%%%%%%%%%
\problem{18 points}

Alexis, Jamal, and Kasey buy a small cabin. They decide
to divide the time each one occupies the cabin using the lone divider
method. One person is chosen to be the divider and divides the year
into 3  parts that have equal value to the divider. The table
below represents the value of each section in each person's eyes.

\begin{center}
  \begin{tabular}{| c || c | c | c |}%| c|}
    \hline
    & {\bf Jan-May} & {\bf Jun-Sept} & {\bf Oct-Dec}\\% & total value \\
    \hline\hline
    {\bf Alexis} & \$25,000 & \$25,000 & \$25,000\\% & \$75,000\\
    \hline
    {\bf Jamal} & \$30,000 & \$27,000 & \$18,000 \\%& \$75,000 \\
    \hline
    {\bf Kasey} & \$33,000 & \$30,000 & \$12,000 \\%& \$75,000 \\
    \hline
  \end{tabular}
\end{center}

\begin{subproblems}
\item Identify who the divider was. \ans

\item %What is the value of a fair share for Alexis? \ans

%What is the value of a fair share for Jamal? \ans

%What is the value of a fair share for Kasey? \ans


 Circle the values in the table that represent a fair share to each person.
\item What happens in the next step of the lone divider process? Explain.
\vfill
\item Suppose they instead had submitted the following table of values.

\begin{center}
  \begin{tabular}{| c || c | c | c |}%| c|}
    \hline
    & {\bf Jan-May} & {\bf Jun-Sept} & {\bf Oct-Dec}\\% & total value \\
    \hline\hline
    {\bf Alexis} & \$25,000 & \$25,000 & \$25,000\\% & \$75,000\\
    \hline
    {\bf Jamal} & \$5,000 &  \$70,000 & \$5,000 \\%& \$75,000 \\
    \hline
    {\bf Kasey} & \$1,000 &  \$73,000 &\$1,000 \\%& \$75,000 \\
    \hline
  \end{tabular}
\end{center}
What happens in the next step of the lone divider process? Explain what the next step in the process should be and why (but you don't have to actually implement the step).

\vfill

\end{subproblems}
 
\newpage
%%%%%%%%%%%%%%%%%%%%%%%%%%%

%\fbox{Extra Credit} 
\problem{Extra Credit: 6 points} 

Andrea and Zeke are dividing up three items: an espresso maker, a cleaning robot, and a microwave. They submit the following sealed bids for the three items.

\begin{tabular}{| c | c | c | c| }\hline
& Espresso Maker & Cleaning robot & Microwave \\ \hline \hline
Andrea & \$60 & \$120 & \$30 \\ \hline
Zeke & \$100 & \$50 & \$75 \\ \hline
\end{tabular}
%\newpage

\be[a.]
\item Determine each person's fair share (in dollars).

\begin{tabular}{c|c}
Andrea's fair share & Zeke's fair share\\
\hline
&\\
&\\
&\\
\end{tabular}

\item  Determine which person gets each item.\\

\begin{tabular}{c|c|c}
Espresso Maker&Cleaning Robot&Microwave\\
\hline
&&\\
&&\\
\end{tabular}
\item Determine the \textbf{surplus}. Show your work!

\vfill
\item How many dollars does each person pay or receive in the end? Show your work!

Andrea:
\vfill
\vfill
Zeke:
\vfill
\vfill
\ee

%Determine the outcome of the sealed bids process. For each person, list which items they received and whether they got or owed money from the process. Show your work.
%\vfill

%\end{enumerate}
\end{document}

%%%%ENDDOCUMENT


