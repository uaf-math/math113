\documentclass[12pt]{article}
\usepackage[top=1in, bottom=1in, left=.75in, right=.75in]{geometry}
\usepackage{amsmath, enumerate}
\usepackage{fancyhdr}
\usepackage{graphicx, xcolor, setspace, array, adjustbox}
\usepackage{txfonts}
\usepackage{multicol,coordsys,pgfplots}
\usepackage[group-separator={,}]{siunitx}
\usepackage[scaled=0.86]{helvet}
\renewcommand{\emph}[1]{\textsf{\textbf{#1}}}
\usepackage{anyfontsize}
% \usepackage{times}
% \usepackage[lf]{MinionPro}
%\usepackage{tikz,pgfplots}

\usepackage{tikz}
\usetikzlibrary{calc,trees,positioning,arrows,fit,shapes,through, backgrounds}
\usetikzlibrary{patterns}

\usetikzlibrary{decorations.markings}
\usetikzlibrary{arrows}

%\def\degC{{}^\circ{\rm C}}
\def\ra{\rightarrow}
\usetikzlibrary{calc,arrows.meta}
\pgfplotsset{compat = newest}
\newcommand{\blank}[1]{\rule{#1}{0.75pt}}

\pgfplotsset{my style/.append style={axis x line=middle, axis y line=
middle, xlabel={$x$}, ylabel={$y$}}}

%axis equal

%yticklabels={,,} , xticklabels={,,}

% \setmainfont{Times}
% \def\sansfont{Lucida Grande Bold}
\parindent 0pt
\parskip 4pt
\pagestyle{fancy}
\fancyfoot[C]{\emph{\thepage}}
\fancyfoot[R]{} %%%%%% <-- Version Info
\fancyhead[L]{\ifnum \value{page} > 1\relax\emph{Math F113X: Exam 3}\fi}
\fancyhead[R]{\ifnum \value{page} > 1\relax\emph{Spring 2025}\fi}
\headheight 15pt
\renewcommand{\headrulewidth}{0pt}
\renewcommand{\footrulewidth}{0pt}
\let\ds\displaystyle
\def\continued{{\emph {Continued....}}}
\def\continuing{{\emph {Problem \arabic{probcount} continued....}}\par\vskip 4pt}


\newcounter{probcount}
\newcounter{subprobcount}
\newcommand{\thesubproblem}{\emph{\alph{subprobcount}.}}
\def\problem#1{\setcounter{subprobcount}{0}%
\addtocounter{probcount}{1}{\emph{\arabic{probcount}.\hskip 1em(#1)}}\par}
\def\subproblem#1{\par\hangindent=1em\hangafter=0{%
\addtocounter{subprobcount}{1}\thesubproblem\emph{#1}\hskip 1em}}
\def\probskip{\vskip 10pt}
\def\medprobskip{\vskip 2in}
\def\subprobskip{\vskip 45pt}
\def\bigprobskip{\vskip 4in}


\newenvironment{subproblems}{%
\begin{enumerate}%
\setcounter{enumi}{\value{subprobcount}}%
\renewcommand{\theenumi}{\emph{\alph{enumi}}}}%
{\setcounter{subprobcount}{\value{enumi}}\end{enumerate}}


\newcommand{\be}{\begin{enumerate}}
\newcommand{\ee}{\end{enumerate}}


\newcommand{\ans}[1][1in]{\rule{#1}{.5pt}}



\begin{document}

{\emph{\fontsize{20}{28}\selectfont Spring 2025 \hfill
%{\fontsize{32}{36}\selectfont Calculus 1: Midterm 1}
\hfill Math F113X}}

\begin{center}
{\emph{%\fontsize{26}{28}\selectfont Spring 2024 
%%\hfill
{\fontsize{24}{36}\selectfont Exam 3}
%%\hfill Math F251X}
}}
\end{center}

%\vskip 2cm
\strut\vtop{\halign{\emph#\hskip 0.5em\hfil&#\hbox to 2in{\hrulefill}\cr
\emph{\fontsize{14}{22}\selectfont Name:}&\cr
%\noalign{\vskip 10pt}
%\emph{\fontsize{18}{22}\selectfont Student Id:}&\cr
%\noalign{\vskip 10pt}
%\emph{\fontsize{18}{22}\selectfont Calculator Model:}&\cr
}}
\hfill
\vtop{\halign{\emph{\fontsize{14}{22}\selectfont #}\hfil& \emph{\fontsize{14}{22}\selectfont\hskip 0.5ex $\square$ #}\hfil\cr
Section: & 10:30 am (Leah Berman )\cr
\noalign{\vskip 4pt}
         & 2:15pm (Jill Faudree)\cr
}}

\vfill
{\fontsize{18}{22}\selectfont\emph{Rules:}}

\begin{itemize}
\item Partial credit will be awarded, but you must show your work.

\item You may have 1/2 of a standard page of paper ($8.5''\times 5.5''$)  of notes, both sides.

\item Calculators are allowed. 

\item Place a box around your  \fbox{FINAL ANSWER} to each question where appropriate.

\item Turn off anything that might go beep during the exam.

\item A formula sheet  (tabula recta, finance formulas) is included on the last page of the exam. You may tear it off if you need to.

\end{itemize}

%If you need extra space, you can use the back sides of the pages.
%Please make it obvious  when you have done so.



%Good luck!

\vfill
\def\emptybox{\hbox to 2em{\vrule height 16pt depth 8pt width 0pt\hfil}}
\def\tline{\noalign{\hrule}}
\centerline{\vbox{\offinterlineskip
{
\bf\sf\fontsize{18pt}{22pt}\selectfont
\hrule
\halign{
\vrule#&\strut\quad\hfil#\hfil\quad&\vrule#&\quad\hfil#\hfil\quad
&\vrule#&\quad\hfil#\hfil\quad&\vrule#\cr
height 3pt&\omit&&\omit&&\omit&\cr
&Problem&&Possible&&Score&\cr\tline
height 3pt&\omit&&\omit&&\omit&\cr
&1&& 12	&&\emptybox&\cr\tline
&2&& 16	&&\emptybox&\cr\tline
&3&& 10	&&\emptybox&\cr\tline
&4&& 16	&&\emptybox&\cr\tline
&5&& 5	&&\emptybox&\cr\tline
&6&& 5	&&\emptybox&\cr\tline
&7&& 20	&&\emptybox&\cr\tline
&8&& 6	&&\emptybox&\cr\tline \tline
&9&& 10	&&\emptybox&\cr\tline \tline
&Extra Credit&&(5)&&\emptybox&\cr\tline
&Total&&100&&\emptybox&\cr
}\hrule}}}

\vfill

\newpage
%Page 1%

%%%%%%%%%%%
%Simple Caesar Cipher
%%%%
\problem{12 points} 
	\begin{subproblems}
	\item Encrypt the message EAGLE using an alphabetic Caesar cipher 	with shift 8 (mapping A to I).
	\vfill
	\item Decrypt the message AYCQL if it was encrypted using an
	alphabetic Caesar cipher 	with shift 8 (mapping A to I).
	%SQUID
	\vfill
	\end{subproblems}

%%%%%%%%%%%
%Transposition Cipher with Key word
%%%%
\problem{16 points} 
	\begin{subproblems}
	\item Encrypt the message SPRING IS HERE using a tabular transposition cipher with the encryption keyword JUMBO.
	\vfill
	\item Decrypt the message NSEZM ENROM ZSBAB BOADA  if it was encrypted using a tabular transposition cipher with the encryption keyword JUMBO.
	%MOONBEAMS AND ZEBRAS
	\vfill
	\end{subproblems}
\newpage

%%%%%%%%%%%
%Shifting Ciphers
%%%%
\problem{10 points} Pick \textbf{ONE} of the following. Make sure to clearly identify which one you want graded by checking the appropriate box. 

{\bf Briefly explain} the technique you used to decrypt the message.
	\begin{subproblems}
	\item (Grade this one: {\Large{${\square}$}}) Decrypt the message VFBLV if it was encrypted using a 
	\textbf{shifting Caesar cipher} that started with a shift of 4 (mapping A to 
	E).
	%RAVEN
	\vfill
	\item  (Grade this one: {\Large{${\square}$}}) Decrypt the message MECZD if it was encrypted using a 			\textbf{Vigen\`{e}re cipher} with keyword FACE. %HEAVY
	\vfill
	\end{subproblems}
%%%%%%%%%%%
%Short Answers
%%%%	

\problem{16 points} %Short Answer
	\begin{subproblems}
	\item Which of the encryption methods below are vulnerable to being decrypted using frequency analysis? Check the box to indicate it is vulnerable.\\
	
	\begin{tabular}{lll}
	{\Large{$\square$}} Caesar cipher &{\Large{$\square$}}  tabular transposition & {\Large{$\square$}} Vigen\`{e}re \\
	&&\\
	{\Large{$\square$}} shifting Caesar cipher&{\Large{$\square$}} double transpositon&
	\end{tabular}
	\item Below is a table defining a \textbf{randomly assigned} substitution mapping. Describe one advantage of this encryption mechanism and one disadvantage.\\
	
\hspace*{-1cm}	{\scriptsize
\begin{tabular}{|c|c|c|c|c|c|c|c|c|c|c|c|c|c|c|c|c|c|c|c|c|c|c|c|c|c|c|}
\hline
original& A&B&C&D&E&F&G&H&I&J&K&L&M&N&O&P&Q&R&S&T&U&V&W&X&Y&Z\\  \hline
maps to&E&M&S&U&N&H&T&I&P&J&O&C&D&R&A&Y&V&B&Z&W&F&X&Q&G&K&L\\
\hline
\end{tabular}
	}
	\end{subproblems}
\vfill

\newpage

\problem{5 points} You want to encrypt the message YOUR COVER IS BLOWN. LEAVE SAFE HOUSE NOW. using a Vigen\`{e}re cipher. Pick a code word and explain why it is a good choice. Note: You are \textbf{not} asked to encrypt the message!
\vfill
%\newpage

%%%%%%%%%
% tip
%%%

%%%%%%%%%%%%%%%%	

\problem{5 points} Your bill at a restaurant is \$38.67 and you want to leave an 18\% tip. How much would you add to your bill? Show your calculation.
\vfill

\newpage
%%%%%%%%%
% Identify the formula
%%%



\problem{20 points} For each scenario below, \emph{identify} the formula you should use and then \emph{plug appropriate numbers into} that formula. You do \emph{not} need to calculate the the value.\\

	\begin{subproblems}
	\item You loan your friend \$400. They agree to pay an annual interest rate of 5\% simple interest. Eighteen months later, they repay the loan. How much did they pay you?
	\vfill
	\item You deposit \$1000 in an account that earns an annual interest rate of 4.8\% APR. The interest is compounding weekly. How much will the account be worth in 10 years?
	\vfill
	\item You want to take out a loan to buy a \$150,000 home. The bank offers a 30-year mortgage with an interest rate of 6.2\%. What will the monthly payments be? (Suppose the interest is compounded monthly.)
	\vfill
	\item You deposit \$4000 in an account the earns 2.75\% APR compounded daily for 7 years. How much interest did you earn? 
	\vfill
	\end{subproblems}
\newpage

%%%%%%
%Explain the calculation
%%%%%
\problem{6 points} It is a fact that \[\displaystyle \num{441488}=\num{10000}\left(1+\frac{0.05}{12}\right)^{(12) \cdot (42)}.\] Suppose this calculation is used to model a savings account. Explain what this calculation indicates about the account parameters.\\

	\begin{subproblems}
	\item How much was invested at the start? \blank{2in} \\
	\item What was the annual interest rate? \blank{2in} \\
	\item How frequently is the interest being compounded? \blank{2in} \\
	\item How long was the money invested? \blank{2in} \\
	\item Explain, in your own words, what the number \num{441488} represents.\\
	\vfill
	\vfill
	\end{subproblems}
	
	\problem{10 points} Suppose you charge \$2000 on a credit card with an annual percentage rate of 29\% compounded monthly and you plan to make monthly payments of \$40.

	\begin{subproblems}
	\item How much do you owe at the end of the first month? (Show your computation, and actually compute it.)
	\vfill
	\item What does the calculation in part (a) indicate about when you will pay off the loan?
	\vfill
	
	\vfill
	\end{subproblems}


\newpage


%%%%%
%loan basics
%%
%%%%%
%Extra Credit
%%
\textbf{Extra Credit (5 points)} 

Pick \textbf{one} of the two options. Indicate which one you want graded.
	\begin{enumerate}[(a)]
	\item (Grade this one: {\Large{${\square}$}}) 
	
	Suppose you have the following part of a spreadsheet, where \#\#\# represents numbers that have been entered in (but that could be changed).
	
	\begin{tabular}{c || c | c  | c| c|  p{2cm} | }
	 & A & B & C & D & E \\ \hline \hline
	1 & P & r & n & t & A \\ \hline
	2 & \#\#\# & \#\#\# & 1 & 1 & \\ \hline
	3 & &&&2& \\ \hline 
		4 & &&&3& \\ \hline 
			$\vdots$ & && & $\vdots$& \\ \hline 
	\end{tabular}
	
	
	\be[i.]
	\item Write a formula, using cell references (\verb`A1`, \verb`B2`, \verb`C3`, etc.), to put into cell \verb`E2` to compute the total amount of money at the end of the first year.
	
	\vspace{1in}
	
	\item You want to drag cell \verb`E3` down to automatically compute compound interest (annually compounded) for subsequent years. How would you change your previous formula to be able to do that?
		\vspace{1in}
	
	\ee
	
%	\vfill
	\item (Grade this one: {\Large{${\square}$}}) 
	
	Decrypt the text 
	\begin{quote}S2L53 \:\:CDIR8 \:\:TEOE4\:\:R3H  \end{quote}if it was encrypted using a double transposition cipher with first keyword JUMBO and second keyword KEYS.
	\vfill
	\end{enumerate}
	
	
	\newpage
	%%%%%%%%%%%%%%%
%	\begin{center}  Formulas \end{center}
	
	
%\textbf{Formulas}\\
%$$ A=P+I \quad \hspace{1cm} \quad A=P(1+rt) \quad \hspace{1cm} \quad A=P\left(1+\frac{r}{n}\right)^{(nt)} \quad \hspace{1cm} \quad P=\frac{A}{\left( 1+\frac{r}{n}\right)^{(nt)}} $$
%
%$$$$ P=\frac{d(1-\left(1+\frac{r}{n}\right)^{(-nt)}}{\left(\frac{r}{n} \right)} \hspace{1cm} \quad d= \frac{P\left(\frac{r}{n} \right)}{\left( 1- \left(1+\frac{r}{n}\right)^{(-nt)}\right)} \qquad
% d= \frac{P\left(\frac{r}{n} \right)}{\left( 1- \left(1+\frac{r}{n}\right)^{(-nt)}\right)}$$

\textbf{Formulas}\\
$$ A=P+I \quad \hspace{1cm} \quad A=P(1+rt) \quad \hspace{1cm} \quad A=P\left(1+\frac{r}{n}\right)^{(nt)} \quad \hspace{1cm} \quad P=\frac{A}{\left( 1+\frac{r}{n}\right)^{(nt)}} $$

$$ P=\frac{d(1-\left(1+\frac{r}{n}\right)^{(-nt)}}{\left(\frac{r}{n} \right)} \hspace{1cm} \quad d= \frac{P\left(\frac{r}{n} \right)}{\left( 1- \left(1+\frac{r}{n}\right)^{(-nt)}\right)}$$

\vfill

%		\newpage
	%%%%%%%%%%%%%

	
	\begin{center}
	
	
\makeatletter
\newcommand{\Letter}[1]{\@Alph{#1}}
\makeatother


	\def\r{.6}
\begin{tikzpicture}
\draw[step=\r] (0,0) grid (27*\r,27*\r);
%\foreach \i in {1,2,...,26}{\draw (0,0) -- (\i*\r, 1);}
\foreach \i in {1,2,...,26}{\path (\i*\r+1/2*\r, 27*\r-1/2*\r) node {\textbf{\Letter{\i}}};}%
\foreach \i in {1,2,...,26}{\path (1/2*\r, 26*\r-\i*\r+1/2*\r) node {\textbf{\Letter{\i}}};}%
\draw[line width = .75mm] (0,26*\r) -- (27*\r, 26*\r);
\draw[line width = .75mm] (\r,0) -- (\r, 27*\r);
\foreach \i in {1,2,...,26}
	\foreach \j in {1,2,...,26}{
		{\path let \n1 = {int(mod((\j-1)+(\i-1), 26)+1} in (\i*\r+1/2*\r, 27*\r-\j*\r - 1/2*\r) node {\Letter{\n1}};
		}}
\end{tikzpicture}
\end{center}

\end{document}

%%%%ENDDOCUMENT


