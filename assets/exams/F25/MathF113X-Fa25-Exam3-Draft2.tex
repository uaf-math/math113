\documentclass[12pt]{article}
\usepackage[top=1in, bottom=1in, left=.75in, right=.75in]{geometry}
\usepackage{amsmath, enumerate}
\usepackage{fancyhdr}
\usepackage{graphicx, xcolor, setspace, array, adjustbox}
\usepackage{txfonts}
\usepackage{multicol,coordsys,pgfplots}
\usepackage[group-separator={,}]{siunitx}
\usepackage[scaled=0.86]{helvet}
\renewcommand{\emph}[1]{\textsf{\textbf{#1}}}
\usepackage{anyfontsize}
% \usepackage{times}
% \usepackage[lf]{MinionPro}
%\usepackage{tikz,pgfplots}

\usepackage{tikz}
\usetikzlibrary{calc,trees,positioning,arrows,fit,shapes,through, backgrounds}
\usetikzlibrary{patterns}

\usetikzlibrary{decorations.markings}
\usetikzlibrary{arrows}

%\def\degC{{}^\circ{\rm C}}
\def\ra{\rightarrow}
\usetikzlibrary{calc,arrows.meta}
\pgfplotsset{compat = newest}
\newcommand{\blank}[1]{\rule{#1}{0.75pt}}

\pgfplotsset{my style/.append style={axis x line=middle, axis y line=
middle, xlabel={$x$}, ylabel={$y$}}}

%axis equal

%yticklabels={,,} , xticklabels={,,}

% \setmainfont{Times}
% \def\sansfont{Lucida Grande Bold}
\parindent 0pt
\parskip 4pt
\pagestyle{fancy}
\fancyfoot[C]{\emph{\thepage}}
\fancyfoot[R]{} %%%%%% <-- Version Info
\fancyhead[L]{\ifnum \value{page} > 1\relax\emph{Math F113X: Exam 3}\fi}
\fancyhead[R]{\ifnum \value{page} > 1\relax\emph{Fall 2025}\fi}
\headheight 15pt
\renewcommand{\headrulewidth}{0pt}
\renewcommand{\footrulewidth}{0pt}
\let\ds\displaystyle
\def\continued{{\emph {Continued....}}}
\def\continuing{{\emph {Problem \arabic{probcount} continued....}}\par\vskip 4pt}


\newcounter{probcount}
\newcounter{subprobcount}
\newcommand{\thesubproblem}{\emph{\alph{subprobcount}.}}
\def\problem#1{\setcounter{subprobcount}{0}%
\addtocounter{probcount}{1}{\emph{\arabic{probcount}.\hskip 1em(#1)}}\par}
\def\subproblem#1{\par\hangindent=1em\hangafter=0{%
\addtocounter{subprobcount}{1}\thesubproblem\emph{#1}\hskip 1em}}
\def\probskip{\vskip 10pt}
\def\medprobskip{\vskip 2in}
\def\subprobskip{\vskip 45pt}
\def\bigprobskip{\vskip 4in}


\newenvironment{subproblems}{%
\begin{enumerate}%
\setcounter{enumi}{\value{subprobcount}}%
\renewcommand{\theenumi}{\emph{\alph{enumi}}}}%
{\setcounter{subprobcount}{\value{enumi}}\end{enumerate}}


\newcommand{\be}{\begin{enumerate}}
\newcommand{\ee}{\end{enumerate}}


\newcommand{\ans}[1][1in]{\rule{#1}{.5pt}}



\begin{document}

{\emph{\fontsize{20}{28}\selectfont Fall 2025 \hfill
%{\fontsize{32}{36}\selectfont Calculus 1: Midterm 1}
\hfill Math F113X}}

\begin{center}
{\emph{%\fontsize{26}{28}\selectfont Spring 2024 
%%\hfill
{\fontsize{24}{36}\selectfont Exam 3}
%%\hfill Math F251X}
}}
\end{center}

%\vskip 2cm
\strut\vtop{\halign{\emph#\hskip 0.5em\hfil&#\hbox to 2in{\hrulefill}\cr
\emph{\fontsize{14}{22}\selectfont Name:}&\cr
%\noalign{\vskip 10pt}
%\emph{\fontsize{18}{22}\selectfont Student Id:}&\cr
%\noalign{\vskip 10pt}
%\emph{\fontsize{18}{22}\selectfont Calculator Model:}&\cr
}}
\hfill
\vtop{\halign{\emph{\fontsize{14}{22}\selectfont #}\hfil& \emph{\fontsize{14}{22}\selectfont\hskip 0.5ex $\square$ #}\hfil\cr
Section: & 10:30 am (Leah Berman )\cr
\noalign{\vskip 4pt}
         & 11:45pm (Kevin Meek)\cr
         \noalign{\vskip 4pt}
         & online (Kevin Meek)\cr
}}

\vfill
{\fontsize{18}{22}\selectfont\emph{Rules:}}

\begin{itemize}
\item Partial credit will be awarded, but you must show your work.

\item You may have 1/2 of a standard page of paper ($8.5''\times 5.5''$)  of notes, both sides.

\item Calculators are allowed. 

\item Place a box around your  \fbox{FINAL ANSWER} to each question where appropriate.

\item Turn off anything that might go beep during the exam.

\item A formula sheet  (tabula recta, finance formulas) is included on the last page of the exam. You may tear it off if you need to.

\end{itemize}

%If you need extra space, you can use the back sides of the pages.
%Please make it obvious  when you have done so.



%Good luck!

\vfill
\def\emptybox{\hbox to 2em{\vrule height 16pt depth 8pt width 0pt\hfil}}
\def\tline{\noalign{\hrule}}
\centerline{\vbox{\offinterlineskip
{
\bf\sf\fontsize{18pt}{22pt}\selectfont
\hrule
\halign{
\vrule#&\strut\quad\hfil#\hfil\quad&\vrule#&\quad\hfil#\hfil\quad
&\vrule#&\quad\hfil#\hfil\quad&\vrule#\cr
height 3pt&\omit&&\omit&&\omit&\cr
&Problem&&Possible&&Score&\cr\tline
height 3pt&\omit&&\omit&&\omit&\cr
&1&& 20	&&\emptybox&\cr\tline
&2&& 18	&&\emptybox&\cr\tline
&3&& 12	&&\emptybox&\cr\tline
&4&& 5	&&\emptybox&\cr\tline
&5&& 5	&&\emptybox&\cr\tline
&6&& 20	&&\emptybox&\cr\tline
&7&& 10	&&\emptybox&\cr\tline
&8&& 10	&&\emptybox&\cr\tline \tline
&Extra Credit&&(5)&&\emptybox&\cr\tline
&Total&&100&&\emptybox&\cr
}\hrule}}}

\vfill

\newpage

\problem{20 points}
\begin{subproblems}

\item \emph{Encrypt} the message \textsf{GARBANZO} using a \emph{Caesar cipher}
with
shift 7 (mapping A to H).
\vfill

\item \emph{Decrypt} the message \textsf{DXPQE} using a \emph{ progressive Caesar
cipher / sequential shift cipher} starting with a shift of 3 (mapping A to C).
\vfill

\item \emph{Encrypt} the message \textsf{GOOSE} using a \emph{Vigen\`ere cipher}
with keyword {\tt BONK}
\vfill
\end{subproblems}

\newpage

\problem{18 points}
\begin{subproblems}
\item  \emph{Encrypt} the message \textsf{THE CAT IS OUT OF THE BAG} using a
\emph{tabular transposition cipher} with no keyword and rows of length
5.
\vfill

\item \emph{Decrypt} the message \begin{quote}\textsf{RTECZ WEOTE AULUY EOFTX} \end{quote} using a \emph{
tabular transposition cipher} with keyword {\tt CURB}.
\vfill
\end{subproblems}

\newpage
\problem{12 points} For each of the following encryption methods, list at
least one advantage and at least one disadvantage of the encryption
system.
\begin{subproblems}
\item Caesar cipher
  \begin{itemize}
  \item Advantage: %\hrulefill%\underline{\hspace{5cm}}
    \vspace{1.5cm}
  \item Disadvantage: %\hrulefill%\underline{\hspace{5cm}}
      \vspace{1.5cm}
  \end{itemize}
%  \vfill
\item Double transposition
  \begin{itemize}
  \item Advantage: %\hrulefill%\underline{\hspace{5cm}}
    \vspace{1.5cm}
  \item Disadvantage: %\hrulefill%\underline{\hspace{5cm}}
      \vspace{1.5cm}
  \end{itemize}
%  \vfill
\end{subproblems}

\problem{5 points} What feature(s) of the word {\tt ALMOST} make it be a bad keyword for a \emph{transposition cipher}? Write your answer in a sentence or two.
\vfill

%%%%%%%%%%%%%%%%



\problem{5 points} Suppose you loan a friend \$1000 and they agree that they will pay back the entire amount they borrowed, plus 6\% \emph{simple interest}. How much interest will they pay you? Show your computation, and provide a numeric answer.
\vfill


%%%%%%%%%
% Identify the formula
%%%
\newpage	


\problem{20 points} For each scenario below, %\emph{identify} the formula you should use
 \emph{plug in numbers into the appropriate formula} needed to calculate the value. Do \emph{not} actually calculate the the value.\\

	\begin{subproblems}
		\item You deposit \$1500 in an account with an APR of 2.4\%. The interest is compounded quarterly. How much will the account be worth in 5 years?
	\vfill
	\item You loan your friend \$600. They agree to pay an annual interest rate of 5\% \emph{simple interest}. Fifteen months later, they repay the loan. How much did they pay you in total?
	\vfill
	\item You want to take out a loan to buy a \$250,000 home. The bank offers a 30-year mortgage with an interest rate of 6.2\%. What will the monthly payments be? (Suppose the interest is compounded monthly.)
	\vfill
	\item You deposit \$6000 in an account the earns 2.75\% APR compounded \emph{daily} for 5 years. How much \emph{interest} did you earn? 
	\vfill
		\item If you have a mortgage and you pay exactly \$1200 a month for 30 years, how much did you pay over the life of the mortgage? %You want to end up with \$10,000 at the end of 7 years. If you invest at an annual interest rate 
	\vfill
	\end{subproblems}
\newpage

%%%%%%
%Explain the calculation
%%%%%
\problem{10 points} It is a fact that \[ \num{185078.84}=
\frac{\num{1500}\left(1 - \left(1 + \frac{0.05375}{12}\right)^{-12\cdot15}\right)}{\frac{0.05375}{12}}\]
Suppose this calculation is used to model a loan in which payments are made regularly. Explain what this calculation indicates about the account parameters.

	\begin{subproblems}
	\item What was the annual interest rate (as a percent)? \blank{2in} \\
	\item How long was the loan for? \blank{2in} \\
	\item How frequently are regular payments being made? \blank{2in} \\
	\item How much is each regular payment?\blank{2in} \\
	\item Explain, in your own words, what the number \num{185078.84} represents.\\
	\vfill
	\vfill
	\end{subproblems}
	
	\problem{10 points} Suppose you have \$2500 in charges on a credit card with an annual percentage rate of 29.99\%, compounded monthly. In order to pay down your balance, you plan to stop charging anything onto the card, and you plan to make monthly payments of \$50.

	\begin{subproblems}
	\item How much do you owe at the end of the first month? (Show your computation, and actually compute it.)
	\vfill
	\item Is this a reasonable payment plan? Why or why not?
	\vfill
	
	\vfill
	\end{subproblems}


\newpage


%%%%%
%loan basics
%%
%%%%%
%Extra Credit
%%
\emph{Extra Credit (5 points)} 

Pick \emph{one} of the two options. Indicate which one you want graded.
	\begin{enumerate}[(a)]
	\item (Grade this one: {\Large{${\square}$}}) 
	
	Suppose you have the following part of a spreadsheet, where \#\#\# represents specific numbers that have been typed in (but which could be changed).
	
	\begin{tabular}{c || c | c  | c| c|  p{2cm} | }
	 	& A 	 & B 	  & C 	& D & E \\ \hline \hline
	1 	& P 	 & r 	  & n 	& t & A \\ \hline
	2 	& \#\#\# & \#\#\# & 1 	& 1 & \\ \hline
	3 	& 		 &		  &		& 2	& \\ \hline 
	4 	& 		 &		  &		& 3	& \\ \hline 
		$\vdots$ & 		  &		& 	& $\vdots$& \\ \hline 
	\end{tabular}
	
	
	\be[i.]
	\item Write a formula, using cell references (\verb`A1`, \verb`B2`, \verb`C3`, etc.), to put into cell \verb`E2` to compute the total amount of money you have at the end of the first year.
	
	\vspace{1cm}
	
	\item %You want to drag cell \verb`E3` down to automatically compute compound interest (compounded annually) for subsequent years. How would you change your previous formula to be able to do that?
	If you want to compute the compound interest for year 2, what cell reference should you enter into cell {\tt A3}?		\vspace{1cm}
	
	\item What should you enter into cell {\tt E3} to be able to fill down row 3 to compute compound interest for subsequent years?
		\vspace{1cm}
	
	\ee
	
%	\vfill
	\item (Grade this one: {\Large{${\square}$}}) 
	
	Decrypt the text 
	\begin{quote}
	\textsf{EHACC LEOTP PAIOE CN}
	  \end{quote}if it was encrypted using a double transposition cipher with first keyword {\tt BLUE} and second keyword {\tt HOUSE}.
	\vfill
	\end{enumerate}
	
	
	\newpage
	%%%%%%%%%%%%%%%
%	\begin{center}  Formulas \end{center}
	
	
%\textbf{Formulas}\\
%$$ A=P+I \quad \hspace{1cm} \quad A=P(1+rt) \quad \hspace{1cm} \quad A=P\left(1+\frac{r}{n}\right)^{(nt)} \quad \hspace{1cm} \quad P=\frac{A}{\left( 1+\frac{r}{n}\right)^{(nt)}} $$
%
%$$$$ P=\frac{d(1-\left(1+\frac{r}{n}\right)^{(-nt)}}{\left(\frac{r}{n} \right)} \hspace{1cm} \quad d= \frac{P\left(\frac{r}{n} \right)}{\left( 1- \left(1+\frac{r}{n}\right)^{(-nt)}\right)} \qquad
% d= \frac{P\left(\frac{r}{n} \right)}{\left( 1- \left(1+\frac{r}{n}\right)^{(-nt)}\right)}$$

\textbf{Formulas}\\
$$ A=P+I \quad \hspace{1cm} \quad A=P(1+rt) \quad \hspace{1cm} \quad A=P\left(1+\frac{r}{n}\right)^{(nt)} \quad \hspace{1cm} \quad P=\frac{A}{\left( 1+\frac{r}{n}\right)^{(nt)}} $$

$$ P=\frac{d\left(1-\left(1+\frac{r}{n}\right)^{(-nt)}\right)}{\left(\frac{r}{n} \right)} \hspace{1cm} \quad d= \frac{P\left(\frac{r}{n} \right)}{\left( 1- \left(1+\frac{r}{n}\right)^{(-nt)}\right)}$$

\vfill

%		\newpage
	%%%%%%%%%%%%%

\begin{center}
	
	
\makeatletter
\newcommand{\Letter}[1]{\@Alph{#1}}
\makeatother

\def\r{.6}
\begin{tikzpicture}
\draw[step=\r] (0,0) grid (28*\r,28*\r);
%\foreach \i in {1,2,...,26}{\draw (0,0) -- (\i*\r, 1);}
\foreach \i in {1,2,...,26}{\path let \n1 = {int(\i-1)} in (\i*\r+1/2*\r+1*\r, 28*\r-1/2*\r) node {\n1};}
\foreach \i in {1,2,...,26}{\path (\i*\r+1/2*\r +1*\r, 27*\r-1/2*\r) node {\textbf{\Letter{\i}}};}%
\foreach \i in {1,2,...,26}{\path (1/2*\r+1*\r, 26*\r-\i*\r+1/2*\r) node {\textbf{\Letter{\i}}};}%
\foreach \i in {1,2,...,26}{\path let \n1 = {int(\i-1)} in  (1/2*\r, 26*\r-\i*\r+1/2*\r) node {\n1};}%
\draw[line width = .75mm] (0,26*\r) -- (28*\r, 26*\r);
\draw[line width = .75mm] (\r+1*\r,0) -- (\r+1*\r, 28*\r);
\foreach \i in {1,2,...,26}
	\foreach \j in {1,2,...,26}{
		{\path let \n1 = {int(mod((\j-1)+(\i-1), 26)+1} in (\i*\r+1/2*\r+1*\r, 27*\r-\j*\r - 1/2*\r) node {\Letter{\n1}};
		}}
\end{tikzpicture}\end{center}

\end{document}

%%%%ENDDOCUMENT



%%% Local Variables:
%%% mode: LaTeX
%%% TeX-master: t
%%% End:
